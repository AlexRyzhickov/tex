\documentclass{beamer}
\usepackage{beamerthemesplit}
\usepackage{wrapfig}
\usetheme{SPbGU}
\usepackage{pdfpages}
\usepackage{amsmath}
\usepackage{cmap} 
\usepackage[T2A]{fontenc} 
\usepackage[utf8]{inputenc}
\usepackage[english,russian]{babel}
\usepackage{indentfirst}
\usepackage{amsmath}
\usepackage{tikz}
\usepackage{multirow}
\usepackage[noend]{algpseudocode}
\usepackage{algorithm}
\usepackage{algorithmicx}
\usetikzlibrary{shapes,arrows}
\usepackage{fancyvrb}
\newtheorem{rutheorem}{Теорема}
\newtheorem{ruproof}{Доказательство}
\newtheorem{rudefinition}{Определение}
\newtheorem{rulemma}{Лемма}
\beamertemplatenavigationsymbolsempty

\title[]{Парсер-комбинаторы и левая рекурсия}
\subtitle[]{введение}
% То, что в квадратных скобках, отображается в левом нижнем углу. 
\institute[]{
Лаборатория языковых инструментов JetBrains \\
Санкт-Петербургский государственный университет \\
Математико-механический факультет }

% То, что в квадратных скобках, отображается в левом нижнем углу.
\author[Екатерина Вербицкая]{Екатерина Вербицкая}

\date{2 ноября 2015г.}

\definecolor{orange}{RGB}{179,36,31}

\begin{document}
{
\begin{frame}[fragile]
  \titlepage
\end{frame}
}

\begin{frame}[fragile]
  \transwipe[direction=90]
  \frametitle{Синтаксический анализ}
  Сопоставление строки терминалов формальной грамматике языка

  \begin{itemize}
    \item Восходящие
    \item Нисходящие
  \end{itemize}

  \begin{itemize}
    \item Написанные руками
    \item Генерируемые по спецификации грамматики
    \item Написанные руками, но с использованием некого инструментария
  \end{itemize}
\end{frame}
            
\begin{frame}[fragile]
  \transwipe[direction=90]
  \frametitle{Парсер-комбинаторы: общая идея}
  \begin{itemize}
    \item Написать примитивные парсеры (например, для обработки отдельных 
символов)
    \item Написать комбинаторы для составления из примитивных парсеров более 
сложных
    \begin{itemize}
      \item Последовательности
      \item Альтернативы
      \item Повторение
      \item Опциональный парсер
      \item \dots
    \end{itemize}
    \item С использованием этого инструментария описать необходимый парсер
  \end{itemize}
\end{frame}

\begin{frame}
  \transwipe[direction=90]
  \frametitle{Парсер-комбинаторы: тип парсера}
$$
\begin{array}{crcl}
&Parser & = & String \rightarrow Tree \\ \pause
&Parser & = & String \rightarrow (Tree, String) \\ \pause
&Parser & = & String \rightarrow [(Tree, String)] \\ \\ \pause
&Parser \, a & = & String \rightarrow [(a, String)]
\end{array}
$$
\end{frame}

\begin{frame}[fragile]
  \transwipe[direction=90]
  \frametitle{Примитивные парсеры}
\begin{verbatim}
result :: a -> Parser a
result v = \inp -> [(v,inp)]

zero :: Parser a
zero = \inp -> []

item :: Parser Char
item = \inp -> case inp of
                 [] -> []
                 (x:xs) -> [(x,xs)]
\end{verbatim}  
\end{frame}

\begin{frame}[fragile]
  \transwipe[direction=90]
  \frametitle{Парсер-комбинаторы}
\begin{verbatim}
bind :: Parser a -> (a -> Parser b) -> Parser b
p ‘bind‘ f = \inp -> concat [f v inp' | (v,inp') <- p inp]

p ‘seq‘ q = p ‘bind‘ \x ->
            q ‘bind‘ \y ->
            result (x,y)

sat :: (Char -> Bool) -> Parser Char
sat p = item ‘bind‘ \x ->
        if p x then result x else zero

plus :: Parser a -> Parser a -> Parser a
p ‘plus‘ q = \inp -> (p inp ++ q inp)
\end{verbatim}
\end{frame}

\begin{frame}[fragile]
  \transwipe[direction=90]
  \frametitle{Маленькие полезные парсеры}
\begin{verbatim}
char :: Char -> Parser Char
char x = sat (\y -> x == y)

digit :: Parser Char
digit = sat (\x -> ’0’ <= x && x <= ’9’)

lower :: Parser Char
lower = sat (\x -> ’a’ <= x && x <= ’z’)

upper :: Parser Char
upper = sat (\x -> ’A’ <= x && x <= ’Z’)
\end{verbatim}
\end{frame}

\begin{frame}[fragile]
  \transwipe[direction=90]
  \frametitle{Менее маленькие полезные парсеры}
\begin{verbatim}
letter :: Parser Char
letter = lower ‘plus‘ upper

alphanum :: Parser Char
alphanum = letter ‘plus‘ digit

word :: Parser String
word = neWord ‘plus‘ result ""
       where
         neWord = letter ‘bind‘ \x ->
                  word ‘bind‘ \xs ->
                  result (x:xs)

word "Yes!" = [("Yes","!"), ("Ye","s!"), 
               ("Y","es!"), ("","Yes!")]
\end{verbatim}
\end{frame}

\begin{frame}[fragile]
  \transwipe[direction=90]
  \frametitle{Монадические парсер-комбинаторы}
\begin{verbatim}
class Monad m where
  result :: a -> m a
  bind :: m a -> (a -> m b) -> m b

instance Monad Parser where
  -- result :: a -> Parser a
  result v = \inp -> [(v,inp)]

  -- bind :: Parser a -> (a -> Parser b) -> Parser b
  p ‘bind‘ f = \inp -> concat [f v out | (v,out) <- p inp]
\end{verbatim}
\end{frame}

\begin{frame}[fragile]
  \transwipe[direction=90]
  \frametitle{Монадические парсер-комбинаторы}
\begin{verbatim}
class Monad m => Monad0Plus m where
  zero :: m a
  (++) :: m a -> m a -> m a

instance Monad0Plus Parser where
  -- zero :: Parser a
  zero = \inp -> []
  
  -- (++) :: Parser a -> Parser a -> Parser a
  p ++ q = \inp -> (p inp ++ q inp)
\end{verbatim}
\end{frame}

\begin{frame}[fragile]
  \transwipe[direction=90]
  \frametitle{Другая форма записи}
\begin{verbatim}
p1 ‘bind‘ \x1 ->
p2 ‘bind‘ \x2 ->
...
pn ‘bind‘ \xn ->
result (f x1 x2 ... xn)


[ f x1 x2 ... xn | x1 <- p1
                 , x2 <- p2
                 , ...
                 , xn <- pn ]

string :: String -> Parser String
string "" = [""]
string (x:xs) = [x:xs | _ <- char x, _ <- string xs]
\end{verbatim}
\end{frame}

\begin{frame}[fragile]
  \transwipe[direction=90]
  \frametitle{Монадические парсер-комбинаторы}
\begin{verbatim}
sat :: (Char -> Bool) -> Parser Char
sat p = [x | x <- item, p x]

many :: Parser a -> Parser [a]
many p = [x:xs | x <- p, xs <- many p] ++ [[]]



ident :: Parser String
ident = [x:xs | x <- lower, xs <- many alphanum]
\end{verbatim}
\end{frame}

\begin{frame}[fragile]
  \transwipe[direction=90]
  \frametitle{Примеры парсеров}
\begin{verbatim}
many1 :: Parser a -> Parser [a]
many1 p = [x:xs | x <- p, xs <- many p]

nat :: Parser Int
nat = [eval xs | xs <- many1 digit]
      where
        eval xs = foldl1 op [ord x - ord ’0’ | x <- xs]
        m ‘op‘ n = 10*m + n

int :: Parser Int
int = [-n | _ <- char ’-’, n <- nat] ++ nat

int :: Parser Int
int = [f n | f <- op, n <- nat]
      where
        op = [negate | _ <- char ’-’] ++ [id]
\end{verbatim}
\end{frame}


\begin{frame}[fragile]
  \transwipe[direction=90]
  \frametitle{Пример: список чисел}
\begin{verbatim}
ints :: Parser [Int]
ints = [n:ns | _ <- char ’[’
             , n <- int
             , ns <- many [x | _ <- char ’,’, x <- int]
             , _ <- char ’]’]


sepby1 :: Parser a -> Parser b -> Parser [a]
p ‘sepby1‘ sep = [x:xs | x <- p
                       , xs <- many [y | _ <- sep, y <- p]]

ints = [ns | _ <- char ’[’
           , ns <- int ‘sepby1‘ char ’,’
           , _ <- char ’]’]

\end{verbatim}
\end{frame}


\begin{frame}[fragile]
  \transwipe[direction=90]
  \frametitle{Список чисел: еще короче}
\begin{verbatim}
bracket :: Parser a -> Parser b -> Parser c -> Parser b
bracket open p close = [x | _ <- open, x <- p, _ <- close]

ints = bracket (char ’[’)
               (int ‘sepby1‘ char ’,’)
               (char ’]’)
\end{verbatim}
\end{frame}


\begin{frame}[fragile]
  \transwipe[direction=90]
  \frametitle{Пример: арифметические выражения}
$$
\begin{array}{crcl}
&expr & ::= & expr \, addop \, factor \, | \, factor \\ 
&addop & ::= & + \, | \, - \\ 
&factor & ::= & nat \, | \, ( \, expr \, ) \\
\end{array}
$$ 

\begin{verbatim}

expr :: Parser Int
addop :: Parser (Int -> Int -> Int)
factor :: Parser Int

expr = [f x y | x <- expr
              , f <- addop
              , y <- factor] ++ factor

addop = [(+) | _ <- char ’+’] ++ [(-) | _ <- char ’-’]

factor = nat ++ bracket (char ’(’) expr (char ’)’)
\end{verbatim}
\end{frame}

\begin{frame}[fragile]
  \transwipe[direction=90]
  \frametitle{Преимущества монадических парсер-комбинаторов}
  \begin{itemize}
    \item Простота
    \item Гибкость
    \item Выразительность
    \item Возможность откатываться (backtracking)
    \item Лексический анализ не нужно выделять в отдельный шаг
    \item Можно считать семантику во время синтаксического анализа
  \end{itemize}
\end{frame}

\begin{frame}[fragile]
  \transwipe[direction=90]
  \frametitle{Недостатки монадических парсер-комбинаторов}
  \begin{itemize}
    \item Если использовать неграмотно, можно получить непредсказуемое время 
работы и легко исчерпать всю доступную память
    \begin{itemize}
      \item Наличие общих префиксов у нескольких правил. Решение: факторизация грамматики
      \item Вычисление промежуточных результатов, где не было надо. Решение: использование ленивости (например, $p +++ q = first (p ++ q)$)
    \end{itemize}
  \end{itemize}
\end{frame}

\begin{frame}[fragile]
  \transwipe[direction=90]
  \frametitle{Packrat-парсер: не совсем парсер-комбинаторы}
  \begin{itemize}
    \item Нисходящий
    \item Использует преимущества ленивости и функционального стиля
    \item Unlimited lookahed + backtracking 
    \item Гарантирует линейное время работы на LL(k) и LR(k) грамматиках
    \item Использует запоминание
  \end{itemize}
\end{frame}
         
\begin{frame}[fragile]
  \transwipe[direction=90]
  \frametitle{Packrat-парсер: пример}
Грамматика языка арифметических выражений
$$
\begin{array}{crcl}
&E & :: & M \, + \, E \, | \, M \\ 
&M & :: & P \, * \, M \, | \, P \\ 
&P & :: & ( \, E \, ) \, | \, D \\
&D & :: & 0 \, | \, 1 \, | \, \dots \, | \, 9
\end{array}
$$   	                         
\end{frame}


\begin{frame}[fragile]
  \transwipe[direction=90]
  \frametitle{Типы классического нисходящего парсера}
\begin{verbatim}  
data Result v = Parsed v String
              | NoParse
\end{verbatim}

Типы для правил грамматики выражений:

\begin{verbatim}
pE :: String -> Result Int
pM :: String -> Result Int
pP :: String -> Result Int
pD :: String -> Result Int
\end{verbatim}

\end{frame}

\begin{frame}[fragile]
  \transwipe[direction=90]
  \frametitle{Классический нисходящий парсер выражений}
\begin{verbatim}  
pE s = alt1 where
    alt1 = case pM s of
             Parsed vleft s' ->
               case s' of
                 ('+':s'') ->
                   case pE s'' of
                     Parsed vright s''' ->
                       Parsed (vleft + vright) s'''
                     _ -> alt2
                _ -> alt2
            _ -> alt2

    alt2 = case pM s of
             Parsed v s' -> Parsed v s'
             NoParse -> NoParse
\end{verbatim}
\end{frame}

\begin{frame}[fragile]
  \transwipe[direction=90]
  \frametitle{Проблемы классического нисходящего парсинга}
  \begin{itemize}
    \item При откатах одно и то же может вычисляться несколько раз
    \item В худшем случае экспоненциальное время работы
    \item Факторизация грамматики -- выход, но не очень хороший
  \end{itemize}

  \begin{itemize}
    \item Решение: запоминать промежуточные результаты в таблице
  \end{itemize}

\end{frame}

\begin{frame}[fragile]
  \transwipe[direction=90]
  \frametitle{Таблица}
\begin{verbatim}
data Derivs = Derivs { dvE :: Result Int,
                       dvM :: Result Int,
                       dvP :: Result Int,
                       dvD :: Result Int,
                       dvC :: Result Char }
\end{verbatim}
\end{frame}

\begin{frame}[fragile]
  \transwipe[direction=90]
  \frametitle{Типы}
\begin{verbatim}
data Result v = Parsed v Derivs
              | NoParse


pE :: Derivs -> Result Int
pM :: Derivs -> Result Int
pP :: Derivs -> Result Int
pD :: Derivs -> Result Int
\end{verbatim}
\end{frame}

\begin{frame}[fragile]
  \transwipe[direction=90]
  \frametitle{Парсер для выражения}
\begin{verbatim}
pE d = alt1 where
    alt1 = case dvM d of
             Parsed vleft d' ->
               case dvC d' of
                 Parsed '+' d'' ->
                   case dvE d'' of
                     Parsed vright d''' ->
                        Parsed (vleft + vright) d'''
                     _ -> alt2
                 _ -> alt2
             _ -> alt2

    alt2 = dvM d
\end{verbatim}
\end{frame}

\begin{frame}[fragile]
  \transwipe[direction=90]
  \frametitle{Парсер}
\begin{verbatim}
parse :: String -> Derivs
parse s = d where
    d    = Derivs add mult prim dec chr
    add  = pE d
    mult = pM d
    prim = pP d
    dec  = pD d
    chr  = case s of
             (c:s') -> Parsed c (parse s')
             [] -> NoParse
\end{verbatim}
\end{frame}

\begin{frame}[fragile]
  \transwipe[direction=90]
  \frametitle{Монадический Packrat-парсер}
\begin{verbatim}
newtype Parser v = Parser (Derivs -> Result v)             

instance Monad Parser where
  
  (Parser p1) >>= f2 = Parser pre
          where pre d = post (p1 d)
                post (Parsed v d’) = p2 d’
                  where Parser p2 = f2 v
                post (NoParse) = NoParse
  
  return x = Parser (\d -> Parsed x d)
  
  fail msg = Parser (\d -> NoParse)
\end{verbatim}
\end{frame}

\begin{frame}[fragile]
  \transwipe[direction=90]
  \frametitle{Монадический Packrat-парсер}
Альтернатива: 
\begin{verbatim}
(<|>) :: Parser v -> Parser v -> Parser v
(Parser p1) <|> (Parser p2) = Parser pre
                where pre d = post d (p1 d)
                      post d NoParse = p2 d
                      post d r = r

Parser pE = (do vleft <- Parser dvM
                char ’+’
                vright <- Parser dvE
                return (vleft + vright))
        <|> (do Parser dvM)

\end{verbatim}
\end{frame}


\begin{frame}[fragile]
  \transwipe[direction=90]
  \frametitle{Преимущества и недостатки}
\begin{itemize}
  \item Линейное время работы, простота, выразительность, но:
\end{itemize}
\begin{itemize}
  \item Можно реализовать только детерминированные парсеры
  \item Нет состояний
  \item Потребляет достаточно много памяти
  \item Проблемы с левой рекурсией
\end{itemize}
\end{frame}

\begin{frame}[fragile]
  \transwipe[direction=90]
  \frametitle{Борьба с левой рекурсией: Warth et al}
\begin{itemize}
  \item Первый раз, когда наткнулись на левую рекурсию, запомнить в таблице 
ошибку анализа --- посадить семечко
  \item Откатиться во входном потоке в начало правила и применить правило еще раз --- прорастить семечко
\end{itemize}

\begin{itemize}
  \item Определение левой рекурсии:
  \begin{itemize}
    \item В таблице наряду с вычисленным значением храним галочку: 
леворекурсивно ли это правило
    \item Перед применением правила, выставляем галочку в false
    \item Если в таблице нет вычисленного значения, значит попали в левую 
рекурсию: выставляем галочку в true, садим семечко
  \end{itemize}
\end{itemize}
\end{frame}

\begin{frame}[fragile]
  \transwipe[direction=90]
  \frametitle{Warth et al: неявная рекурсия}
Пример: разбираем "$4-3$" относительно грамматики
\begin{verbatim}
X :: E
E :: X - Num | Num
\end{verbatim}

При обработке X посадили семечко: $X \rightarrow E \rightarrow 4$

При проращивании семечка (повторном вызове парсера X) получаем уже лежащее в 
таблице значение ($X \rightarrow E \rightarrow 4$)
                        
Решение: 
\begin{itemize}
  \item Храним стек вызовов правил
  \item При проращивании семечка проращиваем все вовлеченные в левую рекурсию 
правила
\end{itemize}
\end{frame}


\begin{frame}[fragile]
  \transwipe[direction=90]
  \frametitle{Проблемы с подходом Warth et al}
Проблемы с ассоциативностью, если есть и левая, и правая рекурсия

\begin{verbatim}
E :: E - E | Num
\end{verbatim}

"$1-2-3$" разберется как "$1-(2-(3))$". Однако если переписать грамматику 
следующим образом, ассоциативность будет правильной

\begin{verbatim}
E :: E - E (- E)? | Num
\end{verbatim}

Частичное решение: ограничить правую рекурсию максимум одним шагом в глубину

\end{frame}

\begin{frame}[fragile]
  \transwipe[direction=90]
  \frametitle{Литература}
\begin{itemize}
  \item Monadic Parser Combinators: \url{http://www.cs.nott.ac.uk/~pszgmh/monparsing.pdf}
  \item Packrat Parsing: \url{http://www.brynosaurus.com/pub/lang/packrat-icfp02.pdf}
  \item Packrat Parsers Can Support Left Recursion: \url{http://www.vpri.org/pdf/tr2007002_packrat.pdf}
  \item Проблемы с предыдущим подходом (Direct Left-Recursive Parsing Expression Grammars): \url{http://tratt.net/laurie/research/pubs/papers/tratt__direct_left_recursive_parsing_expression_grammars.pdf}
\end{itemize}

\end{frame}
\end{document}
