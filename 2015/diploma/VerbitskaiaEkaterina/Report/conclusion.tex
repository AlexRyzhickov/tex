% У заключения нет номера главы
\clearpage

\section*{Заключение}
В ходе данной работы получены следующие результаты. 
\begin{itemize}
  \item Изучена предметная область: методы обработки встроенных языков и алгоритм обобщённого синтаксического анализа RNGLR.
  \item Разработан алгоритм синтаксического анализа динамически формируемых выражений, поддерживающий работу с произвольными входными графами.
  \item Доказана корректность алгоритма:
  \begin{itemize}
    \item алгоритм завершит работу для любых входных данных;
          %для любой входной детерминированной контекстно-свободной грамматики и 
          %произвольного входного графа алгоритм завершит свою работу;
    \item для любой цепочки из входного множества, выводимой в эталонной грамматике G, в SPPF содержится её дерево вывода в G; при этом никакие другие деревья не содержатся в SPPF.
          %для любой цепочки, которую может породить автомат (которая содержится 
          %в регулярном множестве), выводимой в рассматриваемой грамматике G, в 
          %SPPF содержится её дерево вывода в G, при этом не содержится никаких 
          %других деревьев.
  \end{itemize}
  \item Выполнена реализация алгоритма на языке программирования F\# в рамках исследовательского проекта YaccConstructor.
  \item Проведена апробация: регрессионное тестирование, тестирование производительности и тестирование на реальных данных.
  \item Исходный код проекта YaccConstructor можно найти на сайте \url{https://github.com/YaccConstructor/YaccConstructor}, автор принимал участие под учётной записью kajigor.
\end{itemize}

В дальнейшем планируется изменить алгоритм таким образом, чтобы помимо 
построения леса разбора всех корректных выражений осуществлялся бы также поиск ошибочных выражений и сообщение о них. Также необходимо произвести теоретическую оценку сложности алгоритма. Предложенный алгоритм планируется внедрить в инструмент по реинжинирингу информационных систем.  