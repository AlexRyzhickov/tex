\section{Conclusion}

In this paper, we discussed some issues which arise in partial deduction of a relational programming language, \mk{}.
We presented a novel approach to partial deduction which uses less-branching heuristic to select the most suitable relation call to unfold at each step of driving.
We compared this approach to the earlier implementation of conjunctive partial deduction and the implementation of CPD equipped with the new less-branching heuristic.

The conservative partial deduction improved the execution time of all relations while the implementations of CPD degraded the performance of some of them.
However, CPD equipped with the less-branching heuristic improved the execution time of one relation the most compared with the other specializers.
We conclude that there is still no one good technique which definitely speeds up every relational program.
More research is needed to develop models to predict performance of relations, these models can further be used in specialization.




