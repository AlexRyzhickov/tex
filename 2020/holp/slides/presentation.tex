\documentclass{beamer}
\usepackage{beamerthemesplit}
\usepackage{wrapfig}
\usetheme{SPbGU}
\usepackage{pdfpages}
\usepackage{amsmath}
\usepackage{mathtools}
\usepackage{cmap}
\usepackage[T2A]{fontenc}
\usepackage[utf8]{inputenc}
\usepackage[english,russian]{babel}
\usepackage{indentfirst}
\usepackage{amsmath}
\usepackage{tikz}
\usepackage{multirow}
\usepackage[noend]{algpseudocode}
\usepackage{algorithm}
\usepackage{algorithmicx}
\usepackage{ stmaryrd }
\usepackage{fancyvrb}
\usepackage{qtree}
\usepackage{verbatim}
\usepackage{ulem}
\usepackage{proof}

\beamertemplatenavigationsymbolsempty

\setbeamertemplate{itemize item}[circle]
\setbeamertemplate{enumerate item}[circle]
\newcommand{\derives}[1][*]{\xRightarrow[]{#1}}

\def\To{\derives[]}
\def\iff{\Leftrightarrow}

\usetikzlibrary{shapes,arrows}
\usetikzlibrary{positioning,automata}
\tikzset{every state/.style={minimum size=0.2cm},
initial text={}
}

\graphicspath{
  {pics/}
}

\newcommand{\incimage}[2][0.8]{
  \begin{center}
    \includegraphics[width=\textwidth, height=#1\textheight, keepaspectratio]{#2}
  \end{center}
  }

\title[]{Логическое программирование высшего порядка}
\subtitle[]{}
\institute[]{
Лаборатория языков инструментов JetBrains\\
}

\author[]{Екатерина Вербицкая}

\date{14 сентября 2020}

\definecolor{orange}{RGB}{179,36,31}

\begin{document}
{
  \begin{frame}
    \titlepage
  \end{frame}
}


\begin{frame}[fragile]
  \frametitle{Логическое программирование}
\begin{center}
  Декларативное программирование, основанное на формальной логике
\end{center}
\end{frame}

\begin{frame}[fragile]
  \frametitle{Логическое программирование: пример}
\[
  \forall x. human(x) \to mortal(x)
\]

\[
  human(Socrates)
\]

\vspace{1cm}

\begin{center}
  Смертен ли Сократ?
\end{center}
\end{frame}

\begin{frame}[fragile]
  \frametitle{Логическое программирование: пример}
\[
  \forall x z. \exists y. child (x,y) \wedge child(y,z) \to grandchild(x,z)
\]
\end{frame}

\begin{frame}[fragile]
  \frametitle{Логическое программирование: пример}
\[
  \forall h x y z. (append \ x \ y \ x \vee (append \ x \ y \ z \to append \ (h :: x) \  y \ (h :: z)))
\]
\end{frame}

\begin{frame}[fragile]
  \frametitle{Как это работает: дизъюнкты Хорна}

  \begin{align*}
    G &::= \top \mid A \mid G \wedge G \mid G \vee G \mid \exists_{\tau} x. G \\
    D &::= A \mid G \to D \mid D \wedge D \mid \forall_{\tau} x. D
  \end{align*}

  \vspace{1cm}

  \begin{itemize}
    \item $\top$ --- истина
    \item $A$ --- атом
    \item $G$ --- цель
    \item $D$ --- дизъюнкт Хорна (правило логического вывода)
  \end{itemize}
\end{frame}


\begin{frame}[fragile]
  \frametitle{Как это работает: логический вывод (backchaining)}
\begin{columns}
  \begin{column}{0.3\textwidth}
    \infer[]{\Sigma; \mathcal{P} \longrightarrow \top}{}
  \end{column}

  \begin{column}{0.3\textwidth}
    \infer[]{\Sigma; \mathcal{P} \longrightarrow B_1 \vee B_2}{\Sigma; \mathcal{P} \longrightarrow B_1}
  \end{column}

  \begin{column}{0.3\textwidth}
    \infer[]{\Sigma; \mathcal{P} \longrightarrow B_1 \vee B_2}{\Sigma; \mathcal{P} \longrightarrow B_2}
  \end{column}
\end{columns}

\vspace{1cm}

\begin{columns}
  \begin{column}{0.5\textwidth}
    \infer[]{\Sigma; \mathcal{P} \longrightarrow B_1 \wedge B_2}{\Sigma; \mathcal{P} \longrightarrow B_1 \ \ \ \ \ \Sigma; \mathcal{P} \longrightarrow B_2}
  \end{column}

  \begin{column}{0.4\textwidth}
    \infer[]{\Sigma; \mathcal{P} \longrightarrow B_1 \to B_2}{\Sigma; \mathcal{P}, B_1 \longrightarrow B_2}
  \end{column}

\end{columns}

\vspace{1cm}

\begin{columns}
  \begin{column}{0.5\textwidth}
    \infer[]{\Sigma; \mathcal{P} \longrightarrow \exists_{\tau} x. B}{\Sigma; \mathcal{P} \longrightarrow B[t / x] \ \ \ \ \ \Sigma; \varnothing \vdash t : \tau}
  \end{column}

  \begin{column}{0.4\textwidth}
    \infer[]{\Sigma; \mathcal{P} \longrightarrow \forall_{\tau} x. B}{c : \tau, \Sigma; \mathcal{P} \longrightarrow B[c / x]}
  \end{column}

\end{columns}
\end{frame}

\begin{frame}[fragile]
  \frametitle{Как это работает: логический вывод (backchaining)}
\begin{columns}
  \begin{column}{0.25\textwidth}
    \infer[]{\Sigma; \mathcal{P} \longrightarrow A}{\Sigma; \mathcal{P} \overset{D}{\longrightarrow} A}
  \end{column}

  \begin{column}{0.25\textwidth}
    \infer[]{\Sigma; \mathcal{P} \overset{A}{\longrightarrow} A}{}
  \end{column}

  \begin{column}{0.4\textwidth}
    \infer[]{\Sigma; \mathcal{P} \overset{G \to D}{\longrightarrow} A}{\Sigma; \mathcal{P} \overset{D}{\longrightarrow} A \ \ \ \ \  \Sigma; \mathcal{P} \longrightarrow G}
  \end{column}
\end{columns}

\vspace{1cm}


\begin{columns}
  \begin{column}{0.25\textwidth}
    \infer[]{\Sigma; \mathcal{P} \overset{D_1 \wedge D_2}{\longrightarrow} A}{\Sigma; \mathcal{P} \overset{D_1}{\longrightarrow} A}
  \end{column}

  \begin{column}{0.25\textwidth}
    \infer[]{\Sigma; \mathcal{P} \overset{D_1 \wedge D_2}{\longrightarrow} A}{\Sigma; \mathcal{P} \overset{D_2}{\longrightarrow} A}
  \end{column}

  \begin{column}{0.4\textwidth}
    \infer[]{\Sigma; \mathcal{P} \overset{\forall_{\tau} x. D}{\longrightarrow} A}{\Sigma; \mathcal{P} \overset{D[t / x]}{\longrightarrow} A \ \ \ \ \ \Sigma; \varnothing \vdash t : \tau}
  \end{column}
\end{columns}

\vspace{1cm}


\begin{center}
  Или если $D$ имеет вид $\forall_{\tau_1} x_1, \dots, \forall_{\tau_m} x_m. \ A_1 \wedge \dots \wedge A_n \to A_0$:
\end{center}

\begin{center}
  \begin{minipage}{0.5\textwidth}
    \infer[, A = A_0 \theta]{\Sigma; \mathcal{P} \overset{D}{\longrightarrow} A}{\Sigma; \mathcal{P} \longrightarrow A_1 \theta \ \ \ \ \ \dots \ \ \ \ \ \Sigma; \mathcal{P} \longrightarrow A_n \theta}

  \end{minipage}
\end{center}
\end{frame}

\begin{frame}[fragile]
  \frametitle{Унификация}

\begin{center}
  Даны два терма $t, s$
\end{center}

\begin{center}
  Задача: найти подстановку на свободных переменных термов (унификатор) $\theta$, такую что
\end{center}

\[
  t \theta = s \theta
\]

\end{frame}

\begin{frame}[fragile]
  \frametitle{Алгоритм унификации}

\begin{center}
  Будем искать подстановку как множество уравнений $\mathcal{E} = \{ t_i = s_i \}$
\end{center}

\begin{itemize}
  \item Упрощение термов: $(f \ t_1 \dots t_n = g \ s_1 \dots s_m) \in \mathcal{E}$
  \begin{itemize}
    \item Если $f, g$ --- различные константы, то $\mathcal{E} = \bot$
    \item Иначе заменяем уравнение в $\mathcal{E}$ на множество $t_1 = s_1, \dots, t_n = s_n$
  \end{itemize}
  \item Переориентация: $(t = x) \in \mathcal{E}$
  \begin{itemize}
    \item Если $t$ --- терм, $x$ --- переменная, заменяем в $\mathcal{E}$ уравнение на $x = t$
  \end{itemize}
  \item Элиминация переменных: $(x = t) \in \mathcal{E}$, $x$ входит в какое-то уравнение
  \begin{itemize}
    \item Если $x$ входит в $t$, $t \equiv x$, то удаляем уравнение из $\mathcal{E}$
    \item Иначе, если $x$ входит в $t$, то $\mathcal{E} = \bot$
    \item Иначе, подставляем $t$ вместо $x$ во всех уравнениях в $\mathcal{E}$
  \end{itemize}
\end{itemize}

\end{frame}

\begin{frame}[fragile]
  \frametitle{Унификация: пример}
\[
  \{ node \ El \ T \ T = node \ 1 \ (node \ 2 \ emp \ emp) \ (node \ 2 \ emp \ emp) \}
\]

\[
  \{ El = 1, T = node \ 2 \ emp \ emp, T = node \ 2 \ emp \ emp \}
\]

\[
  \{ El = 1, T = node \ 2 \ emp \ emp, node \ 2 \ emp \ emp = node \ 2 \ emp \ emp \}
\]

\[
  \{ El = 1, T = node \ 2 \ emp \ emp, 2 = 2, emp = emp, emp = emp \}
\]

\[
  \{ El = 1, T = node \ 2 \ emp \ emp \}
\]
\end{frame}


\begin{frame}[fragile]
  \frametitle{Унификация: пример}
\[
  \{ node \ El \ T \ T = node \ 1 \ (node \ 2 \ emp \ emp) \ (node \ 3 \ emp \ emp) \}
\]

\[
  \{ El = 1, T = node \ 2 \ emp \ emp, T = node \ 3 \ emp \ emp \}
\]

\[
  \{ El = 1, T = node \ 2 \ emp \ emp, node \ 2 \ emp \ emp = node \ 3 \ emp \ emp \}
\]

\[
  \{ El = 1, T = node \ 2 \ emp \ emp, 2 = 3, emp = emp, emp = emp \}
\]

\[
  \bot
\]
\end{frame}

\begin{frame}[fragile]
  \frametitle{Логическое программирование над абстрактным синтаксом}

  \begin{columns}
    \begin{column}{0.25\textwidth}
      \infer[]{\Gamma \vdash x : \tau}{\langle x, \tau \rangle \in \Gamma}
    \end{column}

    \begin{column}{0.3\textwidth}
      \infer[]{\Gamma \vdash \lambda x. B : \tau_1 \to \tau}{\langle x, \tau_1 \rangle, \Gamma \vdash B : \tau}
    \end{column}

    \begin{column}{0.45\textwidth}
      \infer[]{\Gamma \vdash m \ n : \tau}{\Gamma \vdash m : \tau_1 \to \tau \ \ \ \ \ \Gamma \vdash n : \tau_1}
    \end{column}
  \end{columns}

\end{frame}

\begin{frame}[fragile]
  \frametitle{FOAS vs HOAS}

  \begin{itemize}
    \item First-order abstract syntax
    \begin{itemize}
      \item Структурное представление термов
      \item Переменные представляются конкретными значениями (строками, числами, ...)
      \item Необходимо реализовывать capture-avoiding substitution
      \begin{itemize}
        \item $\lambda x. y [x / y] \to \lambda x. x$ --- неправильно
        \item $\lambda x. y [x / y] \to \lambda z. x$ --- сложно
      \end{itemize}
      \item Не очень сложно реализовать сравнение термов на равенство
    \end{itemize}
    \item Higher-order abstract syntax
    \begin{itemize}
      \item Структурное представление термов
      \item Переменные и связывания представляются силами метаязыка
      \item Нет необходимости в capture-avoiding substitution: об этом позаботится метаязык
      \item Проверка термов на равенство затруднена
    \end{itemize}
  \end{itemize}
\end{frame}

\begin{frame}
  \frametitle{HOAS в логическом программировании}
  \begin{itemize}
    \item Использование предикатов высшего порядка
    \item Использование функций высшего порядка
  \end{itemize}
\end{frame}

\begin{frame}[fragile]
  \frametitle{Миграция связываний}

  \begin{verbatim}
type app tm -> (tm -> tm)
type abs (tm -> tm) -> tm
  \end{verbatim}

\vspace{-1cm}

\begin{align*}
  & \forall M. term(M) \to \forall N. term(N) \to term(app(M,N)) \\
  & \forall B. (\forall x. term(x) \to term(B \ x)) \to term(abs(B))
\end{align*}

\vspace{0.3cm}

\begin{verbatim}
?- term (abs y\ app y y).
?- pi x\ term x => term ((y \ app y y) x).
?- pi x \ term x => term (app x x)
?- term (app c c).
\end{verbatim}

\end{frame}

\begin{frame}[fragile]
  \frametitle{Идиома миграции связываний}

  Чтобы продолжить анализировать под связыванием:
  \begin{enumerate}
    \item Примени связывание к новой переменной под квантором всеобщности
    \item Используй импликацию, в которой посылка использует эту новую переменную, чтобы сделать выводы о терме
  \end{enumerate}
\end{frame}

\begin{frame}[fragile]
  \frametitle{Higher-order hereditary Harrop formulas}

  \begin{align*}
    G &::= \top \mid A \mid G \wedge G \mid G \vee G \mid \exists_{\tau} x. G \mid D \to G \mid \forall_{\tau} x. G \\
    D &::= A_r \mid G \to D \mid D \wedge D \mid \forall_{\tau} x. D
  \end{align*}

  \vspace{0.5cm}

  \begin{itemize}
    \item $\top$ --- истина
    \item $A$ --- атом
    \item $A_r$ --- жесткий (rigid) атом
    \begin{itemize}
      \item Rigid atom: $h \ t_1 \dots t_n$, где $h$ --- (не логическая) константа
      \item Flexible atom: $h \ t_1 \dots t_n$, где $h$ --- переменная
    \end{itemize}
    \item $G$ --- цель
    \item $D$ --- дизъюнкт Хорна (правило логического вывода)
  \end{itemize}
\end{frame}

\begin{frame}[fragile]
  \frametitle{Унификация высшего порядка: пример}

  \[
    \forall a \ \exists F \ [F \ a = g \ a \ a]
  \]


  \[
    F \in \{ \lambda x. g \ a \ a, \ \lambda x. g \ a \ x, \ \lambda x. g \ x \ a, \ \lambda x. g \ x \ x\}
  \]

  \pause

  \vspace{2cm}

  \[
    \exists F \ \forall a \ [F \ a = g \ a \ a]
  \]


  \[
    F \in \{ \lambda a. g \ a \ a \}
  \]
\end{frame}

\begin{frame}[fragile]
  \frametitle{Унификация высшего порядка: пример}

Предположим, в мире есть только $u : i \to i$, $v : i \to i$

$\lambda w . w$ соответствует пустой строке, $\lambda w. u (v (u \ w))$ --- строке ``uvu''

  \[
    \exists F \ \exists G [\lambda w. F (G w) = \lambda w. u (v (u \ w))]
  \]


\begin{align*}
  F = \lambda w. u (v (u \ w)), & \ G = \lambda w. w \\
  F = \lambda w. u (v \ w), & \ G = \lambda w. u \ w \\
  F = \lambda w. u \ w, & \ G = \lambda w. v (u \ w) \\
  F = \lambda w. w, & \ G = u (v (u \ w))
\end{align*}

\end{frame}


\begin{frame}[fragile]
  \frametitle{Унификация высшего порядка: пример}

Предположим, в мире есть только $u : i \to i$, $v : i \to i$

$\lambda w . w$ соответствует пустой строке, $\lambda w. u (v (u \ w))$ --- строке ``uvu''

  \[
    \exists F [\lambda w. u (F (u w)) = \lambda w. u (v (v (u \ w)))]
  \]


\[
  F = \lambda w. v (v \ w)
\]

\end{frame}

\begin{frame}[fragile]
  \frametitle{Унификация высшего порядка: пример}

Предположим, в мире есть только $u : i \to i$, $v : i \to i$

$\lambda w . w$ соответствует пустой строке, $\lambda w. u (v (u \ w))$ --- строке ``uvu''

  \[
    \exists F [\lambda w. u (F \ w) = \lambda w. F (u \ w)]
  \]


\[
  F \in \{\lambda w. w, \lambda w. u \ w, \lambda w. u (u \ w), \dots \}
\]

\end{frame}

\begin{frame}[fragile]
  \frametitle{Унификация высшего порядка: пример}
\begin{center}
У следующих задач унификаторов нет

$c, d$ --- константы, $F, G$ --- переменные
\end{center}

\[ \lambda x. d (c (F \ x)) = \lambda x. c (d (G \ x)) \]

\[ \lambda x. x (F \ x) = \lambda x. (c (G \ x)) \]

\[ \lambda x. \lambda y. x (F \ x \ y) = \lambda x . \lambda y. y (G \ x \ y)\]

\[\lambda x. \lambda y. x (F \ x \ y) = \lambda x. \lambda y. G \ y \ y\]
\end{frame}


\begin{frame}[fragile]
  \frametitle{Унификация высшего порядка: неразрешимость}
Унификация высшего порядка неразрешима в общем случае: задача соответствия Поста может быть сведена к унификации
\end{frame}

\begin{frame}[fragile]
  \frametitle{Унификация высшего порядка: rigid-rigid уравнения}
Rigid терм --- $\lambda x_1 \dots \lambda x_n. h \ t_1 \dots t_m$, где $h = x_i$ или $h$ находится под квантором всеобщности

\begin{itemize}
  \item Если $c \ t_1  \dots t_m = d \ s_1 \dots s_n$, где $c \neq d$, то унификация невозможна
  \item Если $c = c$, то заменяем это выражение на $\top$
  \item Если $c \ t_1 \dots t_n = c \ s_1 \dots s_n$, то заменяем уравнение на $t_1 = s_1 \wedge \dots \wedge t_n = s_n$
\end{itemize}
\end{frame}

\begin{frame}[fragile]
  \frametitle{Унификация высшего порядка: flexible-rigid уравнения}
Flexible терм --- $\lambda x_1 \dots \lambda x_n. F \ t_1 \dots t_m$, где $F$ находится под квантором существования

\[ F \ t_1  \dots t_n = c \ s_1 \dots s_m \]

\begin{itemize}
  \item Подстановка-имитация
  \begin{itemize}
    \item $F$ связан в скоупе, где связан $c$
    \item $F = \lambda x_1 \dots \lambda x_n. c \ (H_1 \ x_1 \dots x_n) \dots (H_m \ x_1 \dots x_n)$
    \item Добавляем кванторы для $H_1 \dots H_m$ на место квантора для $F$
  \end{itemize}
  \item Подстановка-проекция
  \begin{itemize}
    \item $F = \lambda x_1 \dots \lambda x_n. x_i \ (H_1 \ x_1 \dots x_n) \dots (H_m \ x_1 \dots x_n)$
  \end{itemize}
\end{itemize}
\end{frame}


\begin{frame}[fragile]
  \frametitle{Унификация высшего порядка: пример}
\[ \forall a. \forall g. \exists F [F \ a = g \ a \ a] \]
\[ F \in \{ \lambda x. g \ (H_1 \ x) \ (H_2 \ x), \  \lambda x. x \} \]
\begin{center}
  Подставляем второй вариант, нормализуем, получается неунифицирующееся уравнение
\end{center}
\[\forall a. \forall g. [a = g \ a \ a ]\]
\end{frame}

\begin{frame}[fragile]
  \frametitle{Унификация высшего порядка: пример}
\[ \forall a. \forall g. \exists F [F \ a = g \ a \ a] \]
\[ F \in \{ \lambda x. g \ (H_1 \ x) \ (H_2 \ x), \  \lambda x. x \} \]
\begin{center}
  Подставляем первый вариант
\end{center}
\[ \forall a. \forall g. \exists H_1. \exists H_2. [g \ (H_1 \ a) \ (H_2 \ a) = g \ a \ a]\]
\begin{center}
  Упрощаем
\end{center}
\[ \forall a. \forall g. \exists H_1. \exists H_2. [H_1 \ a = a \wedge H_2 \ a = a] \]
\end{frame}

\begin{frame}[fragile]
  \frametitle{Унификация высшего порядка: пример}
\[ \forall a. \forall g. \exists F [F \ a = g \ a \ a] \]

\[ \forall a. \forall g. \exists H_1. \exists H_2. [H_1 \ a = a \wedge H_2 \ a = a] \]

\[ H_1 \in \{ \lambda x. x, \ \lambda x. a \}\]
\begin{center}
  Подставляем любой из двух вариантов, получаем
\end{center}
\[\forall a. \forall g. \exists H_2. [\top \wedge H_2 \ a = a]\]


\begin{center}
  Собираем все вместе, получаем
\end{center}
\[F \in \{\ \lambda x. g \ a \ a, \ \lambda x. g \ a \ x, \ \lambda x. g \ x \ a, \ \lambda x. g \ x \ x \}\]

\end{frame}


\begin{frame}[fragile]
  \frametitle{Унификация высшего порядка: flexible-flexible}
\[ \forall a . \forall g. \exists X. \exists Y. [X \ a = g \ (F \ a) ]\]
\[ X = \lambda x. g \ (H \ x)\]
\[ \forall a. \forall g. \exists H. \exists Y. [H \ a = Y \ a] \]

\begin{itemize}
  \item Для примитивного типа $\tau$ cоздать уникальную переменную $H^{\tau}$
  \item Для каждого типа $\tau_1 \to \dots \to \tau_n \to \sigma$ назовем каноническим термом $\lambda x_1 \dots \lambda x_n. H^{\sigma}$
  \item Используем канонический терм подходящего типа для flexible термов
\end{itemize}

\end{frame}


\begin{frame}[fragile]
  \frametitle{Унификация высшего порядка: нетерминируемость}
\[ \forall g. \exists F. \forall x. [F \ (g \ x) = g \ (F \ x) ]\]

\[ \forall g. \exists H. \forall x. [H \ (g \ x) = g \ (H \ x) ]\]

\[ F \in \{ \lambda x. x, \ \lambda x. g \ x, \ \lambda x. g \ (g \ x), \dots\}\]
\end{frame}

\begin{frame}[fragile]
  \frametitle{Язык $L_{\lambda}$: кошмарные определения}
Вхождение подформулы $C$ в формулу $B$ называется \textit{положительным}, если оно находится слева от четного количества импликаций в $B$. Иначе --- \textit{негативным}

\begin{itemize}
  \item Вхождение связанной переменной в цель $G$ называется \textit{essentially universal}, если оно связано позитивным квантором всеобщности, негативным квантором существования или $\lambda$-абстракцией
  \item Вхождение связанной переменной в цель $G$ называется \textit{essentially existential}, если оно не является essentially universal
  \item Вхождение связанной переменной в $D$ называется \textit{essentially universal}, если оно связано негативным квантором всеобщности, позитивным квантором существования или $\lambda$-абстракцией
  \item Вхождение связанной переменной в $D$ называется \textit{essentially existential}, если оно не является essentially universal
\end{itemize}
\end{frame}

\begin{frame}[fragile]
  \frametitle{Язык $L_{\lambda}$: основная характеристика}

Запрещена квантификация над предикатами, а также:
в каждом подтерме $x \ t_1 \dots t_n, n \geq 0$, в котором $x$ является essentially existential, все $t_i$ должны быть различными existential universal переменными, связанными внутри скоупа $x$

\vspace{1cm}

Это означает, что если $x$ будет инстанциировано термом $t$, то результирующие $\beta$-редексы будут иметь вид $t \ y_1 \dots y_n$, где все $y_i$ будут связанными, а значит $t = \lambda y_1 \dots y_n. t'$, и $(\lambda y_1 \dots y_n. t') \ y_1  \dots y_n = t'$
\end{frame}


\begin{frame}[fragile]
  \frametitle{Higher-order pattern unification}
Отслеживаем выполнение основной характеристики $L_{\lambda}$ в процессе выполнения унификации

\vspace{1cm}

Работа с rigid-rigid уравнениями не изменяется

\vspace{1cm}

В случае flexible-rigid уравнений все, кроме одной, подстановки сразу же ломаются

\end{frame}

\begin{frame}[fragile]
  \frametitle{Higher-order pattern unification: flexible-rigid}
\[ F \ c_1 \dots c_n = c \ t_1 \dots t_m \]

\begin{itemize}
  \item Если $F$ связано в скоупе квантора, который связывает $c$, то $c$ не совпадает ни с одним из $c_1, \dots, c_n$: ни одна подстановка-проекция не завершится успехом
  \item Если $с$ связано в скоупе квантора, который связывает $F$, тогда подстановка-имитация не завершится успехом, и только $i$-ая подстановка-проекция может выжить: когда $c = c_i$
\end{itemize}

\end{frame}

\begin{frame}[fragile]
  \frametitle{Higher-order pattern unification: flexible-flexible}
\[ F \ c_1 \dots c_n = G \ d_1 \dots d_m \]

\begin{itemize}
  \item $c_i, d_j$, все связаны в скоупе $F, G$
  \item Если $F$ и $G$ --- разные переменные, то $F = \lambda c_1. \dots \lambda c_n. H \ e_1 \dots e_l$, $G = \lambda d_1. \dots \lambda d_n. H \ e_1 \dots e_l$, где $e_1, \dots, e_l$ --- общие переменные
  \item Если $F$ и $G$ --- одинаковые переменные, то $n = m$, $F = \lambda c_1. \dots \lambda c_n. H \ e_1 \dots e_l$, $G = \lambda d_1. \dots \lambda d_n. H \ e_1 \dots e_l$, где $e_1, \dots, e_l$ --- такие переменные, что $c_i = d_i = e_i$
\end{itemize}

В результате получится mgu.
\end{frame}

\begin{frame}[fragile]
  \frametitle{Higher-order pattern unification: терминируемость}
  \[ \forall f. \exists X [X = f \ X] \]
  \[ \forall f. \exists H [H = f \ H] \]

  \begin{center}
    В этом случае спасет occurs-check
  \end{center}

\end{frame}

\begin{frame}[fragile]
  \frametitle{Higher-order pattern unification: примеры}
  \[\forall f. \forall g. \exists U. \exists V. \forall w. \forall x. \forall y. [f \ (U \ x \ y) = f \ (g (V \ y \ w))]\]
  \[\forall g. \exists U. \exists V. \forall w. \forall x. \forall y. [U \ x \ y = g (V \ y \ w)]\]
  \[U = \lambda x. \lambda y. g \ (V' \ y), V = \lambda y. \lambda w. V' \ y\]

  \begin{center}
    Следующая задача не имеет унификатора, потому что срабатывает occurs-check
  \end{center}
  \[\forall g. \exists U. \forall w. \forall x. \forall y. [U \ x \ y = g (U \ y \ w)]\]

  \begin{center}
    Следующая задача не имеет унификатора, потому что $w$ не встречается слева, а $g$ --- под квантором всеобщности
  \end{center}
  \[\forall g. \exists U. \forall w. \forall x. \forall y. [U \ x \ y = g \ w]\]
\end{frame}
\end{document}