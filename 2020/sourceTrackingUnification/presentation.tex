\documentclass[aspectratio=169]{beamer}
\usepackage{beamerthemesplit}
\usepackage{wrapfig}
\usetheme{SPbGU}
\usepackage{pdfpages}
\usepackage{amsmath}
\usepackage{mathtools}
\usepackage{cmap}
\usepackage[T2A]{fontenc}
\usepackage[utf8]{inputenc}
\usepackage[english,russian]{babel}
\usepackage{indentfirst}
\usepackage{amsmath}
\usepackage{tikz}
\usetikzlibrary{shapes,arrows,automata,positioning,quotes,backgrounds,decorations.text,decorations.pathmorphing,calc}
\usepackage{multirow}
\usepackage[noend]{algpseudocode}
\usepackage{algorithm}
\usepackage{algorithmicx}
\usepackage{ stmaryrd }
\usepackage{fancyvrb}
\usepackage{verbatim}
\usepackage{ulem}
\usepackage{proof}
\usepackage{xspace}
\usepackage{bm}
\usepackage{proof}
\usepackage{extarrows}

\beamertemplatenavigationsymbolsempty

\setbeamertemplate{itemize item}[circle]
\setbeamertemplate{enumerate item}[circle]
\newcommand{\derives}[1][*]{\xRightarrow[]{#1}}
\newcommand{\gEdge}[3]{#1 \xlongrightarrow{#2} #3}
\newcommand{\pu}[4]{G \vdash #1 : \gEdge{#2}{#3}{#4}}
\newcommand{\ok}[1]{\textcolor{teal}{#1}}
\newcommand{\notok}[1]{\textcolor{magenta}{#1}}

\setlength{\inferLineSkip}{3pt}

\def\To{\derives[]}
\def\iff{\Leftrightarrow}

\tikzset{every state/.style={minimum size=0.2cm},
initial text={}
}

\tikzset{fnode/.style={shape=circle}}

\tikzset{eqedge/.style={thick, dashed}}
\tikzset{funedge/.style={thick}}
\tikzset{deredge/.style={thick, dashed, teal}}

\tikzstyle{processTree} = [
  ->,
  sibling distance=5em,
  scale=0.9,
  every node/.style = {
    shape=rectangle,
    rounded corners=0.05cm,
    draw,
    align=center,
    minimum size=5mm,
    scale=0.9,
    },
  ]

\newcommand{\iseq}{\stackrel{?}{=}}

\graphicspath{
  {pics/}
}

\newcommand{\incimage}[2][0.8]{
  \begin{center}
    \includegraphics[width=\textwidth, height=#1\textheight, keepaspectratio]{#2}
  \end{center}
  }

\title[]{Унификация посредством поиска путей с контекстно-свободными ограничениями в графе \\ Source-tracking unification}
\subtitle[]{}
\institute[]{
Лаборатория языковых инструментов JetBrains\\
}

\author[]{Екатерина Вербицкая}

\date{16 ноября 2020}

\definecolor{orange}{RGB}{179,36,31}

\begin{document}
{
  \begin{frame}
    \titlepage
  \end{frame}
}


\begin{frame}[fragile]
  \frametitle{TLDR}
\begin{center}
  Задачу унификации можно свести к \\ поиску путей с КС ограничениями в графе\footnote{Choppella, V., and Haynes, C. T. (2005). Source-tracking unification.}

  \bigskip

  Путь --- доказательство (не)унифицируемости термов
\end{center}
\end{frame}

\begin{frame}[fragile]
  \frametitle{План докалада}
\begin{itemize}
  \item Что такое унификация
  \item Как задача унификации представима в виде графа
  \item Какой язык будем использовать в качестве ограничений
  \item Почему это работает
  \item Какую дополнительную информацию можно получить из пути
\end{itemize}

\end{frame}

\begin{frame}[fragile]
  \frametitle{Унификация}

  \begin{center}
    Даны два терма $t, s$
  \end{center}

  \begin{center}
    Задача: найти подстановку на свободных переменных термов (унификатор) $\theta$, такую что
  \end{center}
  \[
    t \theta = s \theta
  \]
\end{frame}


\begin{frame}[fragile]
  \frametitle{Подстановка}
  \[
    \text{Терм: } \mathcal{T} :: \mathcal{V} \mid \mathcal{F}^n \ \mathcal{T}_1 \dots \mathcal{T}_n
  \]

  \[
    \text{Подстановка: } \theta :: \mathcal{V} \to \mathcal{T}
  \]


  \begin{center}
    Применение подстановки $t\{x_1 \mapsto t_1, \dots, x_k \mapsto t_k\}$: \\ одновременно заменить свободные переменные $x_i$ терма $t$ на $t_i$
  \end{center}
  \[
     (f \ x \ a \ (g \ z) \ y)\{x \mapsto h \ a \ y, z \mapsto y\} = f \ (h \ a  \ y) \ a \ (g \ y) \ y
  \]
\end{frame}

\begin{frame}[fragile]
  \frametitle{Применение унификации}
\begin{verbatim}
apply :: (a -> b) -> a -> b
apply f x = f x

f :: Int -> Int
f x = x + 1

apply_f :: ?
apply_f = apply f
\end{verbatim}

\bigskip

Унифицируем \verb!a -> b! и \verb!Int -> Int!, получаем \verb!a == Int!, \verb!b == Int!

\bigskip

\verb!apply_f :: Int -> Int!
\end{frame}



\begin{frame}[fragile]
  \frametitle{Простой алгоритм унификации}

\begin{center}
  Будем искать подстановку как множество уравнений $\mathcal{E} = \{ t_i = s_i \}$
\end{center}

\begin{itemize}
  \item Упрощение термов: $(f \ t_1 \dots t_n = g \ s_1 \dots s_m) \in \mathcal{E}$
  \begin{itemize}
    \item Если $f, g$ --- различные константы, то $\mathcal{E} = \bot$
    \item Иначе заменяем уравнение в $\mathcal{E}$ на множество $t_1 = s_1, \dots, t_n = s_n$
  \end{itemize}
  \item Переориентация: $(t = x) \in \mathcal{E}$
  \begin{itemize}
    \item Если $t$ --- терм, $x$ --- переменная, заменяем в $\mathcal{E}$ уравнение на $x = t$
  \end{itemize}
  \item Элиминация переменных: $(x = t) \in \mathcal{E}$, $x$ входит в какое-то уравнение
  \begin{itemize}
    \item Если $x$ входит в $t$, $t \equiv x$, то удаляем уравнение из $\mathcal{E}$
    \item Иначе, если $x$ входит в $t$, то $\mathcal{E} = \bot$
    \item Иначе, подставляем $t$ вместо $x$ во всех уравнениях в $\mathcal{E}$
  \end{itemize}
\end{itemize}

\end{frame}

\begin{frame}[fragile]
  \frametitle{Унификация: пример}
\[
  \{ node \ El \ T \ T = node \ 1 \ (node \ 2 \ emp \ emp) \ (node \ 2 \ emp \ emp) \}
\]

\[
  \{ El = 1, T = node \ 2 \ emp \ emp, T = node \ 2 \ emp \ emp \}
\]

\[
  \{ El = 1, T = node \ 2 \ emp \ emp, node \ 2 \ emp \ emp = node \ 2 \ emp \ emp \}
\]

\[
  \{ El = 1, T = node \ 2 \ emp \ emp, 2 = 2, emp = emp, emp = emp \}
\]

\[
  \{ El = 1, T = node \ 2 \ emp \ emp \}
\]
\end{frame}


\begin{frame}[fragile]
  \frametitle{Унификация: пример}
\[
  \{ node \ El \ T \ T = node \ 1 \ (node \ 2 \ emp \ emp) \ (node \ 3 \ emp \ emp) \}
\]

\[
  \{ El = 1, T = node \ 2 \ emp \ emp, T = node \ 3 \ emp \ emp \}
\]

\[
  \{ El = 1, T = node \ 2 \ emp \ emp, node \ 2 \ emp \ emp = node \ 3 \ emp \ emp \}
\]

\[
  \{ El = 1, T = node \ 2 \ emp \ emp, 2 = 3, emp = emp, emp = emp \}
\]

\[
  \bot
\]
\end{frame}


\begin{frame}[fragile]
  \frametitle{Чем плох простой алгоритм}
  \begin{itemize}
    \item Не очень эффективный
    \item Не говорит, почему унификация не завершилась успехом
  \end{itemize}
\end{frame}


\begin{frame}[fragile]
  \frametitle{Граф унификации}
\begin{center}
  $a \to b \iseq Int \to Int$
\end{center}

\bigskip

\begin{center}
\begin{tikzpicture}[
  processTree,
  anchor=center]
  \node (0) [fnode] {$\to$}
    child { node (00) {a} }
    child { node (01) {b} };
  \node (1) [fnode] [right of=0, xshift=3cm] {$\to$}
    child { node [fnode] (10) {Int} }
    child { node [fnode] (11) {Int} };
  level 1/.style={sibling distance=5em},
  \draw [eqedge] (0.east) to node[above, draw=none, black] {$\varepsilon$} (1.west);
  \path
  (0) edge [funedge, left]  node [draw=none] {$1$}  (00)
      edge [funedge, right] node [draw=none] {$2$}  (01)
  (1) edge [funedge, left]  node [draw=none] {$1$}  (10)
      edge [funedge, right] node [draw=none] {$2$}  (11);
  \pause
  \draw [deredge] (00.south east) [out=315,in=225] to (10.south west);
  \draw [deredge] (01.south east) [out=315,in=225] to (11.south west);
\end{tikzpicture}
\end{center}
\end{frame}

\begin{frame}[fragile]
  \frametitle{Отношение эквивалентности на вершинах}
Отношение на вершинах $R$ \textit{замкнуто вниз}, если для любой метки на ребре $\delta$ и двух вершин в отношении $uRu'$ с ребрами $u \xrightarrow{\delta} v$ и $u' \xrightarrow{\delta} v'$ верно $vRv'$

\begin{center}
  \begin{tikzpicture}[
    processTree,
    anchor=center]
    \node (0) {$u$}
      child { node (00) {$v$} };
    \node (1) [right of=0, xshift=1.5cm] {$u'$}
      child { node (10) {$v'$} };
    \path
    (0) edge [deredge, above] node [draw=none] {$R$} (1)
    (0) edge [funedge, left]  node [draw=none] {$\delta$}  (00)
    (1) edge [funedge, right]  node [draw=none] {$\delta$}  (10);
    \pause
    \path
    (00) edge [deredge, above] node [draw=none] {$R$} (10);
    % \draw [deredge] (00.east) to (10.west);
  \end{tikzpicture}
\end{center}

\bigskip

\textit{Замыкание унификации отношения} $R$ это наименьшее замкнутое вниз отношение на вершинах, содержащее $R$
\end{frame}

\begin{frame}[fragile]
  \frametitle{Факторграф унификации}
Вершины \textit{равны}, если связаны $\varepsilon$-ребром

\bigskip

\textit{Факторизуем} граф унификации по отношению эквивалентности на вершинах, которое построено как замыкание унификации отношения равенства вершин

\pause

\begin{center}
  \begin{tikzpicture}[
  processTree,
  anchor=center,
  decoration = {
    snake,
    pre length=2pt,
    post length=2pt
  },]
  \node (0) [fnode] {$\to$}
    child { node (00) {a} }
    child { node (01) {b} };
  \node (1) [fnode] [right of=0, xshift=3cm] {$\to$}
    child { node [fnode] (10) {Int} }
    child { node [fnode] (11) {Int} };
  level 1/.style={sibling distance=5em},
  \draw [eqedge] (0.east) to node[above, draw=none, black] {$\varepsilon$} (1.west);
  \path
  (0) edge [funedge, left]  node [draw=none] {$1$}  (00)
      edge [funedge, right] node [draw=none] {$2$}  (01)
  (1) edge [funedge, left]  node [draw=none] {$1$}  (10)
      edge [funedge, right] node [draw=none] {$2$}  (11);
  \draw [deredge] (00.south east) [out=315,in=225] to (10.south west);
  \draw [deredge] (01.south east) [out=315,in=225] to (11.south west);

  \node (12) [fnode, right of=11, xshift=3cm, yshift=1cm] {$a \ b \ int$};
  \draw [decorate] ($(1.east)+(1,-0.5)$) to node [draw=none,above] {факторизация} ($(12.west)+(-0.5,0)$);
\end{tikzpicture}
\end{center}

Унификация невозможна, тогда и только тогда, когда в факторграфе есть цикл или вершина с разными функциональными символами
\end{frame}


\begin{frame}[fragile]
  \frametitle{Большой пример}

  \begin{columns}
    \begin{column}{0.3\textwidth}
      \begin{align*}
  t_0 &\iseq t_1 \to t_2 \\
  t_2 &\iseq t_4 \\
  t_3 &\iseq bool \\
  t_4 &\iseq t_5 \\
  t_3 &\iseq t_1 \\
  t_6 &\iseq t_7 \to t_4 \\
  t_5 &\iseq t_1 \\
  t_6 &\iseq int \to int \\
  t_7 &\iseq t_1
\end{align*}
      \pause
    \end{column}
    \begin{column}{0.7\textwidth}
      \begin{center}
        \begin{tikzpicture}[
  processTree,
  anchor=center,
  node distance=2cm]
  \node (t0) {$t_0$};
  \node (w1) [fnode, below of=t0] {$\to$};
  \path
    (t0) edge [eqedge]  node[left,  draw=none] {$\varepsilon$} (w1);
  \pause

  \node (t2) [right of=t0] {$t_2$};
  \node (t1) [below of=w1] {$t_1$};
  \path
    (w1) edge [funedge] node[left,  draw=none] {$\to_1$}           (t1)
         edge [funedge] node[left,  draw=none] {$\to_2$}           (t2);
  \pause

  \node (t4) [right of=t2] {$t_4$};
  \path (t2) edge [eqedge]  node[above, draw=none] {$\varepsilon$} (t4.west);
  \pause

  \node (t5) [right of=w1] {$t_5$};

  \node (t3) [below of=t1] {$t_3$};
  \node (w5) [fnode, right of=t3] {bool};
  \node (w2) [fnode, right of=t5] {$\to$};
  \node (t7) [below of=w2] {$t_7$};

  \node (t6) [right of=w2] {$t_6$};
  \node (w3) [fnode, below of=t6] {$\to$};
  \node (w4) [fnode, below of=w3] {int};

  \path
    (t3) edge [eqedge]  node[below, draw=none] {$\varepsilon$} (w5)
         edge [eqedge]  node[left,  draw=none] {$\varepsilon$} (t1)
    (w3) edge [funedge, bend left]  node[right, draw=none] {$\to_2$} (w4.north east)
         edge [funedge, bend right] node[left,  draw=none] {$\to_1$} (w4.north west)
    (t2) edge [eqedge]  node[above, draw=none] {$\varepsilon$} (t4.west)
    (t4) edge [eqedge]  node[above, draw=none] {$\varepsilon$} (t5)
    (t5) edge [eqedge]  node[above, draw=none] {$\varepsilon$} (t1)
    (w2) edge [funedge] node[right, draw=none] {$\to_2$}           (t4)
         edge [funedge] node[right, draw=none] {$\to_1$}           (t7)
    (t7) edge [eqedge]  node[above, draw=none] {$\varepsilon$} (t1)
    (t6) edge [eqedge]  node[above, draw=none] {$\varepsilon$} (w2)
         edge [eqedge]  node[right, draw=none] {$\varepsilon$} (w3)
         ;
\end{tikzpicture}
      \end{center}
    \end{column}
    \end{columns}

\end{frame}



\begin{frame}[fragile]
  \frametitle{Большой пример: $t_7 = int$}

  \begin{columns}
    \begin{column}{0.3\textwidth}
        \begin{align*}
  t_0 &\iseq t_1 \to t_2 \\
  t_2 &\iseq t_4 \\
  t_3 &\iseq bool \\
  t_4 &\iseq t_5 \\
  t_3 &\iseq t_1 \\
  t_6 &\iseq t_7 \to t_4 \\
  t_5 &\iseq t_1 \\
  t_6 &\iseq int \to int \\
  t_7 &\iseq t_1
\end{align*}
    \end{column}
    \begin{column}{0.7\textwidth}
      \begin{center}
        \begin{tikzpicture}[
  processTree,
  anchor=center,
  node distance=2cm]
  \node (t0) {$t_0$};
  \node (w1) [fnode, below of=t0] {$\to$};
  \node (t5) [right of=w1] {$t_5$};
  \node (t2) [above of=t5] {$t_2$};
  \node (t1) [below of=w1] {$t_1$};
  \node (t3) [below of=t1] {$t_3$};
  \node (w5) [fnode, right of=t3] {bool};
  \node (w2) [fnode, right of=t5] {$\to$};
  \node (t7) [below of=w2] {$t_7$};
  \node (t4) [above of=w2] {$t_4$};
  \node (t6) [right of=w2] {$t_6$};
  \node (w3) [fnode, below of=t6] {$\to$};
  \node (w4) [fnode, below of=w3] {int};

  \path
    (t0) edge [eqedge]  node[left,  draw=none] {$\varepsilon$} (w1)
    (w1) edge [funedge] node[left,  draw=none] {$\to_1$}           (t1)
         edge [funedge] node[left,  draw=none] {$\to_2$}           (t2)
    (t3) edge [eqedge]  node[below, draw=none] {$\varepsilon$} (w5)
         edge [eqedge]  node[left,  draw=none] {$\varepsilon$} (t1)
    (w3) edge [funedge, bend left]  node[right, draw=none] {$\to_2$} (w4.north east)
         edge [funedge, bend right, purple] node[left,  draw=none] {$\to_1$} (w4.north west)
    (t2) edge [eqedge]  node[above, draw=none] {$\varepsilon$} (t4.west)
    (t4) edge [eqedge]  node[above, draw=none] {$\varepsilon$} (t5)
    (t5) edge [eqedge]  node[above, draw=none] {$\varepsilon$} (t1)
    (w2) edge [funedge] node[right, draw=none] {$\to_2$}           (t4)
         edge [funedge, purple] node[right, draw=none] {$\to_1$}   (t7)
    (t7) edge [eqedge]  node[above, draw=none] {$\varepsilon$} (t1)
    (t6) edge [eqedge, purple] node[above, draw=none] {$\varepsilon$} (w2)
         edge [eqedge, purple] node[right, draw=none] {$\varepsilon$} (w3)
         ;
  \draw [deredge] (t7.south east) [out=315, in=180] to (w4.west);
\end{tikzpicture}
      \end{center}
    \end{column}
    \end{columns}
\end{frame}


\begin{frame}[fragile]
  \frametitle{Большой пример: $t_4 = int$}

  \begin{columns}
    \begin{column}{0.3\textwidth}
        \begin{align*}
  t_0 &\iseq t_1 \to t_2 \\
  t_2 &\iseq t_4 \\
  t_3 &\iseq bool \\
  t_4 &\iseq t_5 \\
  t_3 &\iseq t_1 \\
  t_6 &\iseq t_7 \to t_4 \\
  t_5 &\iseq t_1 \\
  t_6 &\iseq int \to int \\
  t_7 &\iseq t_1
\end{align*}
    \end{column}
    \begin{column}{0.7\textwidth}
      \begin{center}
        \begin{tikzpicture}[
  processTree,
  anchor=center,
  node distance=2cm]
  \node (t0) {$t_0$};
  \node (w1) [fnode, below of=t0] {$\to$};
  \node (t5) [right of=w1] {$t_5$};
  \node (t2) [above of=t5] {$t_2$};
  \node (t1) [below of=w1] {$t_1$};
  \node (t3) [below of=t1] {$t_3$};
  \node (w5) [fnode, right of=t3] {bool};
  \node (w2) [fnode, right of=t5] {$\to$};
  \node (t7) [below of=w2] {$t_7$};
  \node (t4) [above of=w2] {$t_4$};
  \node (t6) [right of=w2] {$t_6$};
  \node (w3) [fnode, below of=t6] {$\to$};
  \node (w4) [fnode, below of=w3] {int};

  \path
    (t0) edge [eqedge]  node[left,  draw=none] {$\varepsilon$} (w1)
    (w1) edge [funedge] node[left,  draw=none] {$\to_1$}           (t1)
         edge [funedge] node[left,  draw=none] {$\to_2$}           (t2)
    (t3) edge [eqedge]  node[below, draw=none] {$\varepsilon$} (w5)
         edge [eqedge]  node[left,  draw=none] {$\varepsilon$} (t1)
    (w3) edge [funedge, bend left, purple]  node[right, draw=none] {$\to_2$} (w4.north east)
         edge [funedge, bend right] node[left,  draw=none] {$\to_1$} (w4.north west)
    (t2) edge [eqedge]  node[above, draw=none] {$\varepsilon$} (t4.west)
    (t4) edge [eqedge]  node[above, draw=none] {$\varepsilon$} (t5)
    (t5) edge [eqedge]  node[above, draw=none] {$\varepsilon$} (t1)
    (w2) edge [funedge, purple] node[right, draw=none] {$\to_2$}           (t4)
         edge [funedge] node[right, draw=none] {$\to_1$}           (t7)
    (t7) edge [eqedge]  node[above, draw=none] {$\varepsilon$} (t1)
    (t6) edge [eqedge, purple]  node[above, draw=none] {$\varepsilon$} (w2)
         edge [eqedge, purple]  node[right, draw=none] {$\varepsilon$} (w3)
         ;
  \draw [deredge] (t4.east) [out=0, in=0] to (w4.east);
\end{tikzpicture}
      \end{center}
    \end{column}
    \end{columns}
\end{frame}

\begin{frame}[fragile]
  \frametitle{Большой пример: $t_7 = bool$}

  \begin{columns}
    \begin{column}{0.3\textwidth}
        \begin{align*}
  t_0 &\iseq t_1 \to t_2 \\
  t_2 &\iseq t_4 \\
  t_3 &\iseq bool \\
  t_4 &\iseq t_5 \\
  t_3 &\iseq t_1 \\
  t_6 &\iseq t_7 \to t_4 \\
  t_5 &\iseq t_1 \\
  t_6 &\iseq int \to int \\
  t_7 &\iseq t_1
\end{align*}
    \end{column}
    \begin{column}{0.7\textwidth}
      \begin{center}
        \begin{tikzpicture}[
  processTree,
  anchor=center,
  node distance=2cm]
  \node (t0) {$t_0$};
  \node (w1) [fnode, below of=t0] {$\to$};
  \node (t5) [right of=w1] {$t_5$};
  \node (t2) [above of=t5] {$t_2$};
  \node (t1) [below of=w1] {$t_1$};
  \node (t3) [below of=t1] {$t_3$};
  \node (w5) [fnode, right of=t3] {bool};
  \node (w2) [fnode, right of=t5] {$\to$};
  \node (t7) [below of=w2] {$t_7$};
  \node (t4) [above of=w2] {$t_4$};
  \node (t6) [right of=w2] {$t_6$};
  \node (w3) [fnode, below of=t6] {$\to$};
  \node (w4) [fnode, below of=w3] {int};

  \path
    (t0) edge [eqedge]  node[left,  draw=none] {$\varepsilon$} (w1)
    (w1) edge [funedge] node[left,  draw=none] {$\to_1$}           (t1)
         edge [funedge] node[left,  draw=none] {$\to_2$}           (t2)
    (t3) edge [eqedge, purple]  node[below, draw=none] {$\varepsilon$} (w5)
         edge [eqedge, purple]  node[left,  draw=none] {$\varepsilon$} (t1)
    (w3) edge [funedge, bend left]  node[right, draw=none] {$\to_2$} (w4.north east)
         edge [funedge, bend right] node[left,  draw=none] {$\to_1$} (w4.north west)
    (t2) edge [eqedge]  node[above, draw=none] {$\varepsilon$} (t4.west)
    (t4) edge [eqedge]  node[above, draw=none] {$\varepsilon$} (t5)
    (t5) edge [eqedge]  node[above, draw=none] {$\varepsilon$} (t1)
    (w2) edge [funedge] node[right, draw=none] {$\to_2$}           (t4)
         edge [funedge] node[right, draw=none] {$\to_1$}           (t7)
    (t7) edge [eqedge, purple]  node[above, draw=none] {$\varepsilon$} (t1)
    (t6) edge [eqedge]  node[above, draw=none] {$\varepsilon$} (w2)
         edge [eqedge]  node[right, draw=none] {$\varepsilon$} (w3)
         ;
  \draw [deredge] (t7.south west) to (w5.north east);
\end{tikzpicture}
      \end{center}
    \end{column}
    \end{columns}
\end{frame}


\begin{frame}[fragile]
  \frametitle{Большой пример: $t_4 = bool$}

  \begin{columns}
    \begin{column}{0.3\textwidth}
        \begin{align*}
  t_0 &\iseq t_1 \to t_2 \\
  t_2 &\iseq t_4 \\
  t_3 &\iseq bool \\
  t_4 &\iseq t_5 \\
  t_3 &\iseq t_1 \\
  t_6 &\iseq t_7 \to t_4 \\
  t_5 &\iseq t_1 \\
  t_6 &\iseq int \to int \\
  t_7 &\iseq t_1
\end{align*}
    \end{column}
    \begin{column}{0.7\textwidth}
      \begin{center}
        \begin{tikzpicture}[
  processTree,
  anchor=center,
  node distance=2cm]
  \node (t0) {$t_0$};
  \node (w1) [fnode, below of=t0] {$\to$};
  \node (t5) [right of=w1] {$t_5$};
  \node (t2) [above of=t5] {$t_2$};
  \node (t1) [below of=w1] {$t_1$};
  \node (t3) [below of=t1] {$t_3$};
  \node (w5) [fnode, right of=t3] {bool};
  \node (w2) [fnode, right of=t5] {$\to$};
  \node (t7) [below of=w2] {$t_7$};
  \node (t4) [above of=w2] {$t_4$};
  \node (t6) [right of=w2] {$t_6$};
  \node (w3) [fnode, below of=t6] {$\to$};
  \node (w4) [fnode, below of=w3] {int};

  \path
    (t0) edge [eqedge]  node[left,  draw=none] {$\varepsilon$} (w1)
    (w1) edge [funedge] node[left,  draw=none] {$\to_1$}           (t1)
         edge [funedge] node[left,  draw=none] {$\to_2$}           (t2)
    (t3) edge [eqedge, purple]  node[below, draw=none] {$\varepsilon$} (w5)
         edge [eqedge, purple]  node[left,  draw=none] {$\varepsilon$} (t1)
    (w3) edge [funedge, bend left]  node[right, draw=none] {$\to_2$} (w4.north east)
         edge [funedge, bend right] node[left,  draw=none] {$\to_1$} (w4.north west)
    (t2) edge [eqedge]  node[above, draw=none] {$\varepsilon$} (t4.west)
    (t4) edge [eqedge, purple]  node[above, draw=none] {$\varepsilon$} (t5)
    (t5) edge [eqedge, purple]  node[above, draw=none] {$\varepsilon$} (t1)
    (w2) edge [funedge] node[right, draw=none] {$\to_2$}           (t4)
         edge [funedge] node[right, draw=none] {$\to_1$}   (t7)
    (t7) edge [eqedge]  node[above, draw=none] {$\varepsilon$} (t1)
    (t6) edge [eqedge] node[above, draw=none] {$\varepsilon$} (w2)
         edge [eqedge] node[right, draw=none] {$\varepsilon$} (w3)
         ;
  \draw [deredge] (t4.south) [out=225, in=90] to (w5.north);
\end{tikzpicture}
      \end{center}
    \end{column}
    \end{columns}
\end{frame}

\begin{frame}[fragile]
  \frametitle{Факторграф для примера: унификация невозможна}

  \begin{columns}
    \begin{column}{0.3\textwidth}
      \begin{center}
        \begin{tikzpicture}[
  processTree,
  anchor=center,
  node distance=3cm]
  \node (0) [fnode] {bool \\ int \\ $t_1 t_2 t_3$ \\ $t_4 t_5 t_7$ };
  \node (1) [fnode, above of=0] {$\to$ \\ $t_0$};
  \node (2) [fnode, below of=0] {$\to$ \\ $\to$ \\ $t_6$};

  \path (1) edge [funedge] node[draw=none] {} (0)
        (2) edge [funedge] node[draw=none] {} (0);

\end{tikzpicture}
      \end{center}
    \end{column}
    \begin{column}{0.7\textwidth}
      \begin{center}
        \begin{tikzpicture}[
  processTree,
  anchor=center,
  node distance=2cm]
  \node (t0) {$t_0$};
  \node (w1) [fnode, below of=t0] {$\to$};
  \path
    (t0) edge [eqedge]  node[left,  draw=none] {$\varepsilon$} (w1);

  \node (t2) [right of=t0] {$t_2$};
  \node (t1) [below of=w1] {$t_1$};
  \path
    (w1) edge [funedge] node[left,  draw=none] {$\to_1$}           (t1)
         edge [funedge] node[left,  draw=none] {$\to_2$}           (t2);

  \node (t4) [right of=t2] {$t_4$};
  \path (t2) edge [eqedge]  node[above, draw=none] {$\varepsilon$} (t4.west);

  \node (t5) [right of=w1] {$t_5$};

  \node (t3) [below of=t1] {$t_3$};
  \node (w5) [fnode, right of=t3] {bool};
  \node (w2) [fnode, right of=t5] {$\to$};
  \node (t7) [below of=w2] {$t_7$};

  \node (t6) [right of=w2] {$t_6$};
  \node (w3) [fnode, below of=t6] {$\to$};
  \node (w4) [fnode, below of=w3] {int};

  \path
    (t3) edge [eqedge]  node[below, draw=none] {$\varepsilon$} (w5)
         edge [eqedge]  node[left,  draw=none] {$\varepsilon$} (t1)
    (w3) edge [funedge, bend left]  node[right, draw=none] {$\to_2$} (w4.north east)
         edge [funedge, bend right] node[left,  draw=none] {$\to_1$} (w4.north west)
    (t2) edge [eqedge]  node[above, draw=none] {$\varepsilon$} (t4.west)
    (t4) edge [eqedge]  node[above, draw=none] {$\varepsilon$} (t5)
    (t5) edge [eqedge]  node[above, draw=none] {$\varepsilon$} (t1)
    (w2) edge [funedge] node[right, draw=none] {$\to_2$}           (t4)
         edge [funedge] node[right, draw=none] {$\to_1$}           (t7)
    (t7) edge [eqedge]  node[above, draw=none] {$\varepsilon$} (t1)
    (t6) edge [eqedge]  node[above, draw=none] {$\varepsilon$} (w2)
         edge [eqedge]  node[right, draw=none] {$\varepsilon$} (w3)
         ;
\end{tikzpicture}
      \end{center}
    \end{column}
    \end{columns}
\end{frame}

\begin{frame}[fragile]
  \frametitle{Полудиков язык: соотношения отмены}
\[
  \Sigma = \{\delta_1, \dots, \delta_n \}
\]
\[
  \Sigma^{-1} = \{\delta_1^{-1}, \dots, \delta_n^{-1} \}
\]
\[
  \Sigma \cap \Sigma^{-1} = \varnothing
\]

\[
  \bm{S}(\Sigma) = \{\delta^{-1} \delta \approx \varepsilon \mid \delta \in \Sigma \}
\]
% \[
%   \bm{S_2}(\Sigma) = \{\delta^{-1} \delta \approx \varepsilon, \delta \delta^{-1} \approx \varepsilon \mid \delta \in \Sigma \}
% \]

\[
  u \stackrel{\bm{S}(\Sigma)}{\approx} v, \text{ если } u \text{ можно получить из } v \text{ применением равенств из } \bm{S}(\Sigma)
\]
% \[
%   \text{ (или наоборот, } v \text{ из } u \text{; аналогично для } u \stackrel{\bm{S_2}(\Sigma)}{\approx} v)
% \]
\end{frame}

\begin{frame}[fragile]
  \frametitle{Полудиков\footnote{Определение полудикова языка может показаться неожиданным} язык}
\[
  \mu_{\bm{S}(\Sigma)}(l) \text{ --- нормальная форма } l \text{ относительно равенств из } \bm{S}(\Sigma)
\]
% \[
%   \mu_{\bm{S_2}(\Sigma)}(l) \text{ --- нормальная форма } l \text{ относительно равенств из } \bm{S_2}(\Sigma)
% \]

\[
  S(\Sigma, L) \stackrel{def}{=} \{ l \in (\Sigma \cup \Sigma^{-1})^* \mid \mu_{\bm{S}(\Sigma)}(l) \in L\}
\]
% \[
%   S_2(\Sigma, L) \stackrel{def}{=} \{ l \in (\Sigma \cup \Sigma^{-1})^* \mid \mu_{\bm{S_2}(\Sigma)}(l) \in L\}
% \]

\begin{align*}
  S^0(\Sigma) &= S(\Sigma, {\varepsilon}) \\
  S^+(\Sigma) &= S(\Sigma, \Sigma^+) \\
  S^*(\Sigma) &= S(\Sigma, \Sigma^*)
\end{align*}

% \begin{columns}ё
\bigskip

  \[
    S^0(\Sigma) \text{ --- полудиков язык (semi-Dyck set)}
  \]

  \bigskip

\end{frame}

\begin{frame}[fragile]
  \frametitle{Примеры строк полудикова языка}
\[
  \Sigma = \{ (, [ \}
\]
\[
  \Sigma^{-1} = \{ ), ] \}
\]

\[
  \ok{\varepsilon}, \ \ok{))((}, \ \ok{][][}, \ \ok{)][()(} \in S^0(\Sigma)
\]
\[
  \notok{()}, \ \notok{))(}, \ \notok{])[(} \notin S^0{\Sigma}
\]

\[
  \ok{(}, \ \ok{)[(}, \ \ok{)(()[(} \in S^*(\Sigma) \setminus S^0(\Sigma)
\]
\[
  \notok{)}, \ \notok{)()} \notin S^*(\Sigma)
\]
\end{frame}


\begin{frame}[fragile]
  \frametitle{Пути унификации: обращение ребер}

  \begin{center}
    \begin{tikzpicture}[
      processTree,
      anchor=center,
      node distance=2cm]
      \node (0) {$u$};
      \node (00) [below of=0] {$v$};
      \node (1) [right of=0] {$u$};
      \node (10) [below of=1] {$v$};
      \path
      (0) edge [funedge, left]  node [draw=none] {$\delta$}  (00)
      (10) edge [funedge, right]  node [draw=none] {$\delta^{-1}$}  (1);
    \end{tikzpicture}
  \end{center}

\pause

\bigskip
\[
  \varepsilon^{-1} = \varepsilon
\]

  \begin{center}
    \begin{tikzpicture}[
      processTree,
      anchor=center,
      node distance=2cm]
      \node (0) {$u$};
      \node (00) [below of=0] {$v$};
      \node (1) [right of=0] {$u$};
      \node (10) [below of=1] {$v$};
      \path
      (0) edge [funedge, left]  node [draw=none] {$\varepsilon$}  (00)
      (10) edge [funedge, right]  node [draw=none] {$\varepsilon$}  (1);
    \end{tikzpicture}
  \end{center}
\end{frame}

\begin{frame}[fragile]
  \frametitle{Пути унификации}
  \begin{center}
    \textit{Путь унификации} --- такой путь $p$ в графе $G \cup G^{-1}$, что нормальная форма его метки $\mu_{\bm{S}(\Sigma)}(l(p)) \in \Sigma^*$

\bigskip

$G$ --- граф унификации, $G^{-1}$ --- граф с обращенными ребрами
  \end{center}

\end{frame}

\begin{frame}[fragile]
  \frametitle{Путь унификации: пример}
  \vspace{-0.5cm}
  \[
    \mu_{\bm{S}(\Sigma)}(l(p^{-1}f^{-1}hr)) = \mu_{\bm{S}(\Sigma)}((\to_1)^{-1} \to_1) = \varepsilon \in \Sigma^*
  \]
  \vspace{-0.5cm}
  \begin{columns}
    \begin{column}{0.3\textwidth}
      \begin{align*}
  a &: t_0 \iseq t_1 \to t_2 \\
  b &: t_2 \iseq t_4 \\
  c &: t_3 \iseq bool \\
  d &: t_4 \iseq t_5 \\
  e &: t_3 \iseq t_1 \\
  f &: t_6 \iseq t_7 \to t_4 \\
  g &: t_5 \iseq t_1 \\
  h &: t_6 \iseq int \to int \\
  i &: t_7 \iseq t_1
\end{align*}
    \end{column}
    \begin{column}{0.7\textwidth}
      \begin{center}
        \begin{tikzpicture}[
     processTree,
     anchor=center,
     node distance=2cm]
     \node (t0) {$t_0$};
     \node (w1) [fnode, below of=t0] {$\to$};
     \node (t5) [right of=w1] {$t_5$};
     \node (t2) [above of=t5] {$t_2$};
     \node (t1) [below of=w1] {$t_1$};
     \node (t3) [below of=t1] {$t_3$};
     \node (w5) [fnode, right of=t3] {bool};
     \node (w2) [fnode, right of=t5] {$\to$};
     \node (t7) [below of=w2] {$t_7$};
     \node (t4) [above of=w2] {$t_4$};
     \node (t6) [right of=w2] {$t_6$};
     \node (w3) [fnode, below of=t6] {$\to$};
     \node (w4) [fnode, below of=w3] {int};

     \path
       (t0) edge [eqedge]  node[left,  draw=none] {$\varepsilon$} (w1)
       (w1) edge [funedge] node[left,  draw=none] {$\to_1$}           (t1)
            edge [funedge] node[left,  draw=none] {$\to_2$}           (t2)
       (t3) edge [eqedge]  node[below, draw=none] {$\varepsilon$} (w5)
            edge [eqedge]  node[left,  draw=none] {$\varepsilon$} (t1)
       (w3) edge [funedge, bend left]  node[right, draw=none] {$\to_2$} (w4.north east)
            edge [funedge, bend right, purple] node[left,  draw=none] {$r: \ \to_1$} (w4.north west)
       (t2) edge [eqedge]  node[above, draw=none] {$\varepsilon$} (t4.west)
       (t4) edge [eqedge]  node[above, draw=none] {$\varepsilon$} (t5)
       (t5) edge [eqedge]  node[above, draw=none] {$\varepsilon$} (t1)
       (w2) edge [funedge] node[right, draw=none] {$\to_2$}           (t4)
            edge [funedge, purple] node[right, draw=none] {$p: \ \to_1$}   (t7)
       (t7) edge [eqedge]  node[above, draw=none] {$\varepsilon$} (t1)
       (t6) edge [eqedge, purple] node[above, draw=none] {$f: \varepsilon$} (w2)
            edge [eqedge, purple] node[right, draw=none] {$h: \varepsilon$} (w3)
            ;
   \end{tikzpicture}
      \end{center}
    \end{column}
    \end{columns}
\end{frame}

\begin{frame}[fragile]
  \frametitle{Путь унификации: пример}
  \vspace{-0.5cm}
  \[
    \mu_{\bm{S}(\Sigma)}(l(s^{-1}f^{-1}hq)) = \mu_{\bm{S}(\Sigma)}((\to_2)^{-1} \to_2) = \varepsilon \in \Sigma^*
  \]
  \vspace{-1cm}
  \begin{columns}
    \begin{column}{0.3\textwidth}
      \begin{align*}
  a &: t_0 \iseq t_1 \to t_2 \\
  b &: t_2 \iseq t_4 \\
  c &: t_3 \iseq bool \\
  d &: t_4 \iseq t_5 \\
  e &: t_3 \iseq t_1 \\
  f &: t_6 \iseq t_7 \to t_4 \\
  g &: t_5 \iseq t_1 \\
  h &: t_6 \iseq int \to int \\
  i &: t_7 \iseq t_1
\end{align*}
    \end{column}
    \begin{column}{0.7\textwidth}
      \begin{center}
        \begin{tikzpicture}[
  processTree,
  anchor=center,
  node distance=2cm]
  \node (t0) {$t_0$};
  \node (w1) [fnode, below of=t0] {$\to$};
  \node (t5) [right of=w1] {$t_5$};
  \node (t2) [above of=t5] {$t_2$};
  \node (t1) [below of=w1] {$t_1$};
  \node (t3) [below of=t1] {$t_3$};
  \node (w5) [fnode, right of=t3] {bool};
  \node (w2) [fnode, right of=t5] {$\to$};
  \node (t7) [below of=w2] {$t_7$};
  \node (t4) [above of=w2] {$t_4$};
  \node (t6) [right of=w2] {$t_6$};
  \node (w3) [fnode, below of=t6] {$\to$};
  \node (w4) [fnode, below of=w3] {int};

  \path
    (t0) edge [eqedge]  node[left,  draw=none] {$\varepsilon$} (w1)
    (w1) edge [funedge] node[left,  draw=none] {$\to_1$}           (t1)
         edge [funedge] node[left,  draw=none] {$\to_2$}           (t2)
    (t3) edge [eqedge]  node[below, draw=none] {$\varepsilon$} (w5)
         edge [eqedge]  node[left,  draw=none] {$\varepsilon$} (t1)
    (w3) edge [funedge, bend left, purple]  node[right, draw=none] {$q: \ \to_2$} (w4.north east)
         edge [funedge, bend right] node[left,  draw=none] {$\to_1$} (w4.north west)
    (t2) edge [eqedge]  node[above, draw=none] {$\varepsilon$} (t4.west)
    (t4) edge [eqedge]  node[above, draw=none] {$\varepsilon$} (t5)
    (t5) edge [eqedge]  node[above, draw=none] {$\varepsilon$} (t1)
    (w2) edge [funedge, purple] node[right, draw=none] {$s: \ \to_2$}           (t4)
         edge [funedge] node[right, draw=none] {$\to_1$}           (t7)
    (t7) edge [eqedge]  node[above, draw=none] {$\varepsilon$} (t1)
    (t6) edge [eqedge, purple]  node[above, draw=none] {$f : \varepsilon$} (w2)
         edge [eqedge, purple]  node[right, draw=none] {$h : \varepsilon$} (w3)
         ;
\end{tikzpicture}
      \end{center}
    \end{column}
    \end{columns}
\end{frame}

\begin{frame}[fragile]
  \frametitle{Путь унификации в факторграфе}

\begin{theorem}
  Если в графе унификации существует путь $p$ из вершины $u$ в вершину $v$, то в факторграфе существует путь из класса эквивалентности $[u]$ в класс эквивалентности $[v]$ с меткой $\mu_{\bm{S}(\Sigma)}(l(p))$
\end{theorem}


\begin{theorem}
  Если в факторграфе унификации существует путь $p'$ из класса эквивалентности $[u]$ в класс эквивалентности $[v]$, то в графе унификации существует путь $p$ из $u$ в вершину $v$ с меткой, чья нормальная форма $\mu_{\bm{S}(\Sigma)}(l(p)) = p'$
\end{theorem}


\begin{center}
  \begin{tikzpicture}[
    processTree,
    anchor=center,
    ]
    \node (0) {$u$}
      child { node (00) {$v$} };
    \node (1) [right of=0, xshift=1.5cm] {$u'$}
      child { node (10) {$v'$} };
    level 1/.style={sibling distance=5em},
    \draw [deredge] (0.east) to (1.west);
    \path
    (0) edge [funedge, left]  node [draw=none] {$\delta$}  (00)
    (1) edge [funedge, right]  node [draw=none] {$\delta$}  (10);
    \draw [deredge] (00.east) to (10.west);
  \end{tikzpicture}
\end{center}

\end{frame}

\begin{frame}[fragile]
  \frametitle{Критерий унифицируемости термов}

\begin{center}
  Два терма унифицируются тогда и только тогда, когда между ними есть путь унификации, нормальная форма которого --- $\varepsilon$
\end{center}

\begin{center}
  Два терма унифицируются тогда и только тогда, когда между ними есть путь унификации, метка которого в полудиковом языке $S^0(\Sigma)$
\end{center}
\end{frame}

\begin{frame}[fragile]
  \frametitle{Логика над путями}
\begin{columns}
  \begin{column}{0.55\textwidth}
    \begin{center}
      \begin{minipage}{\textwidth}
        \infer[, c : \gEdge{u}{\eta}{v} \in E(G)]{\pu{c}{u}{\eta}{v}}{}
      \end{minipage}
    \end{center}
  \end{column}
  \begin{column}{0.45\textwidth}
    \begin{center}
      \begin{minipage}{\textwidth}
        \infer[, u \in V(G)]{\pu{\varepsilon}{u}{\varepsilon}{u}}{}
      \end{minipage}
    \end{center}
  \end{column}
\end{columns}

\bigskip

\begin{columns}
  \begin{column}{0.45\textwidth}
    \begin{center}
      \begin{minipage}{\textwidth}
        \infer[]{\pu{p^{-1}}{u}{\varepsilon}{v}}{\pu{p}{v}{\varepsilon}{u}}
      \end{minipage}
    \end{center}
  \end{column}
  \begin{column}{0.55\textwidth}
    \begin{center}
      \begin{minipage}{\textwidth}
        \infer[]{\pu{pq}{u}{ll'}{v}}{\pu{p}{u}{l}{v'} \ \ \ \ \pu{q}{v'}{l'}{v}}
      \end{minipage}
    \end{center}
  \end{column}
\end{columns}

\bigskip

\begin{center}
  \begin{minipage}{\textwidth}
    \infer[, b : \gEdge{u'}{\delta}{u} \in G, b' : \gEdge{v'}{\delta}{v} \in G]{\pu{b^{-1} p b'}{u}{\varepsilon}{v}}{\pu{p}{u'}{\varepsilon}{v'}}
  \end{minipage}
\end{center}
\end{frame}

\begin{frame}[fragile]
  \frametitle{Вывод успешного пути}

  \begin{columns}
    \begin{column}{0.5\textwidth}
      \begin{center}
        \begin{minipage}{\textwidth}
          \infer[]
          {\pu{p^{-1}f^{-1}hr}{t_7}{\varepsilon}{w_4}}
          {\infer[]
            {\pu{f^{-1}h}{w_2}{\varepsilon}{w_3}}
            {
              \infer[]{\pu{f^{-1}}{w_2}{\varepsilon}{t_6}}{\pu{f}{t_6}{\varepsilon}{w_2}} \ \ \ \ \pu{h}{t_6}{\varepsilon}{w_3}
            }
          }
        \end{minipage}
      \end{center}
    \end{column}
    \begin{column}{0.5\textwidth}
      \begin{center}
        \begin{tikzpicture}[
     processTree,
     anchor=center,
     node distance=2cm]
     \node (t0) {$t_0$};
     \node (w1) [fnode, below of=t0] {$\to$};
     \node (t5) [right of=w1] {$t_5$};
     \node (t2) [above of=t5] {$t_2$};
     \node (t1) [below of=w1] {$t_1$};
     \node (t3) [below of=t1] {$t_3$};
     \node (w5) [fnode, right of=t3] {bool};
     \node (w2) [fnode, right of=t5] {$\to$};
     \node (t7) [below of=w2] {$t_7$};
     \node (t4) [above of=w2] {$t_4$};
     \node (t6) [right of=w2] {$t_6$};
     \node (w3) [fnode, below of=t6] {$\to$};
     \node (w4) [fnode, below of=w3] {int};

     \path
       (t0) edge [eqedge]  node[left,  draw=none] {$\varepsilon$} (w1)
       (w1) edge [funedge] node[left,  draw=none] {$\to_1$}           (t1)
            edge [funedge] node[left,  draw=none] {$\to_2$}           (t2)
       (t3) edge [eqedge]  node[below, draw=none] {$\varepsilon$} (w5)
            edge [eqedge]  node[left,  draw=none] {$\varepsilon$} (t1)
       (w3) edge [funedge, bend left]  node[right, draw=none] {$\to_2$} (w4.north east)
            edge [funedge, bend right, purple] node[left,  draw=none] {$r: \ \to_1$} (w4.north west)
       (t2) edge [eqedge]  node[above, draw=none] {$\varepsilon$} (t4.west)
       (t4) edge [eqedge]  node[above, draw=none] {$\varepsilon$} (t5)
       (t5) edge [eqedge]  node[above, draw=none] {$\varepsilon$} (t1)
       (w2) edge [funedge] node[right, draw=none] {$\to_2$}           (t4)
            edge [funedge, purple] node[right, draw=none] {$p: \ \to_1$}   (t7)
       (t7) edge [eqedge]  node[above, draw=none] {$\varepsilon$} (t1)
       (t6) edge [eqedge, purple] node[above, draw=none] {$f: \varepsilon$} (w2)
            edge [eqedge, purple] node[right, draw=none] {$h: \varepsilon$} (w3)
            ;
   \end{tikzpicture}
      \end{center}
    \end{column}
    \end{columns}



\end{frame}

\begin{frame}[fragile]
  \frametitle{Вывод успешного пути}

  \begin{columns}
    \begin{column}{0.55\textwidth}
      \begin{center}
        \footnotesize
        \begin{minipage}{\textwidth}
          \infer[]
          {\pu{fqb^{-1}}{t_6}{\to_2}{t_2}}
          {
            \pu{f}{t_6}{\varepsilon}{w_2} \ \
            \infer[]
              {\pu{qb^{-1}}{w_2}{\to_2}{t_2}}
              {\pu{q}{w_2}{\to_2}{t_4} \ \
               \infer[]{\pu{b^{-1}}{t_4}{\varepsilon}{t_2}}{\pu{b}{t_2}{\varepsilon}{t_4}}
              }
          }
        \end{minipage}
      \end{center}
    \end{column}
    \begin{column}{0.45\textwidth}
      \begin{center}
        \begin{tikzpicture}[
     processTree,
     anchor=center,
     node distance=2cm,
     scale=0.8,
     every node/.style = {
          shape=rectangle,
          rounded corners=0.05cm,
          draw,
          align=center,
          minimum size=5mm,
          scale=0.8,
     }]
     \node (t0) {$t_0$};
     \node (w1) [fnode, below of=t0] {$\to$};
     \node (t5) [right of=w1] {$t_5$};
     \node (t2) [above of=t5] {$t_2$};
     \node (t1) [below of=w1] {$t_1$};
     \node (t3) [below of=t1] {$t_3$};
     \node (w5) [fnode, right of=t3] {bool};
     \node (w2) [fnode, right of=t5] {$\to$};
     \node [draw=none, left of=w2, xshift=1.3cm] {$w_2$};

     \node (t7) [below of=w2] {$t_7$};
     \node (t4) [above of=w2] {$t_4$};
     \node (t6) [right of=w2] {$t_6$};
     \node (w3) [fnode, below of=t6] {$\to$};
     \node (w4) [fnode, below of=w3] {int};

     \path
       (t0) edge [eqedge]  node[left,  draw=none] {$\varepsilon$} (w1)
       (w1) edge [funedge] node[left,  draw=none] {$\to_1$}           (t1)
            edge [funedge] node[left,  draw=none] {$\to_2$}           (t2)
       (t3) edge [eqedge]  node[below, draw=none] {$\varepsilon$} (w5)
            edge [eqedge]  node[left,  draw=none] {$\varepsilon$} (t1)
       (w3) edge [funedge, bend left]  node[right, draw=none] {$\to_2$} (w4.north east)
            edge [funedge, bend right] node[left,  draw=none] {$\to_1$} (w4.north west)
       (t2) edge [eqedge, purple]  node[above, draw=none] {$b: \varepsilon$} (t4.west)
       (t4) edge [eqedge]  node[above, draw=none] {$\varepsilon$} (t5)
       (t5) edge [eqedge]  node[above, draw=none] {$\varepsilon$} (t1)
       (w2) edge [funedge, purple] node[right, draw=none] {$q: \ \to_2$} (t4)
            edge [funedge] node[right, draw=none] {$\to_1$}   (t7)
       (t7) edge [eqedge]  node[above, draw=none] {$\varepsilon$} (t1)
       (t6) edge [eqedge, purple] node[above, draw=none] {$f: \varepsilon$} (w2)
            edge [eqedge] node[right, draw=none] {$\varepsilon$} (w3)
            ;
   \end{tikzpicture}
      \end{center}
    \end{column}
    \end{columns}
\end{frame}

\begin{frame}[fragile]
  \frametitle{Вывод неуспешного пути}

  \begin{columns}
    \begin{column}{0.48\textwidth}
      \begin{center}
        \begin{minipage}{\textwidth}
          \infer[]
          {\pu{mm^{-1}}{w_1}{\to_1 \to_{1}^{-1}}{w_1}}
          {
            \pu{m}{w_1}{\to_1}{t_1} \ \ \ \
            \infer[]
              {\pu{m^{-1}}{t_1}{\to_1^{-1}}{w_1}}
              {\pu{m}{w_1}{\to_1 }{t_1}}
          }
        \end{minipage}
      \end{center}
    \end{column}
    \begin{column}{0.52\textwidth}
      \begin{center}
        \begin{tikzpicture}[
     processTree,
     anchor=center,
     node distance=2cm]
     \node (t0) {$t_0$};
     \node (w1) [fnode, below of=t0] {$\to$};
     \node [draw=none, left of=w1, xshift=1.3cm] {$w_1$};
     \node (t5) [right of=w1] {$t_5$};
     \node (t2) [above of=t5] {$t_2$};
     \node (t1) [below of=w1] {$t_1$};
     \node (t3) [below of=t1] {$t_3$};
     \node (w5) [fnode, right of=t3] {bool};
     \node (w2) [fnode, right of=t5] {$\to$};
     \node (t7) [below of=w2] {$t_7$};
     \node (t4) [above of=w2] {$t_4$};
     \node (t6) [right of=w2] {$t_6$};
     \node (w3) [fnode, below of=t6] {$\to$};
     \node (w4) [fnode, below of=w3] {int};

     \path
       (t0) edge [eqedge]  node[left,  draw=none] {$\varepsilon$} (w1)
       (w1) edge [funedge,purple] node[left,  draw=none] {$m: \ \to_1$}           (t1)
            edge [funedge] node[left,  draw=none] {$\to_2$}           (t2)
       (t3) edge [eqedge]  node[below, draw=none] {$\varepsilon$} (w5)
            edge [eqedge]  node[left,  draw=none] {$\varepsilon$} (t1)
       (w3) edge [funedge, bend left]  node[right, draw=none] {$\to_2$} (w4.north east)
            edge [funedge, bend right] node[left,  draw=none] {$\to_1$} (w4.north west)
       (t2) edge [eqedge]  node[above, draw=none] {$\varepsilon$} (t4.west)
       (t4) edge [eqedge]  node[above, draw=none] {$\varepsilon$} (t5)
       (t5) edge [eqedge]  node[above, draw=none] {$\varepsilon$} (t1)
       (w2) edge [funedge] node[right, draw=none] {$\to_2$}           (t4)
            edge [funedge] node[right, draw=none] {$\to_1$}   (t7)
       (t7) edge [eqedge]  node[above, draw=none] {$\varepsilon$} (t1)
       (t6) edge [eqedge] node[above, draw=none] {$\varepsilon$} (w2)
            edge [eqedge] node[right, draw=none] {$\varepsilon$} (w3)
            ;
   \end{tikzpicture}
      \end{center}
    \end{column}
    \end{columns}
\end{frame}


\begin{frame}[fragile]
  \frametitle{Алгоритмы унификации через поиск путей}
\begin{itemize}
  \item Один алгоритм ищет кратчайшие пути, работает за $O(n^3)$
  \item Второй алгоритм интегрируется в стандартный алгоритм унификации, работает  на константу хуже и строит не самые маленькие доказательства унифицируемости \\ (но и не самые плохие)
\end{itemize}
\end{frame}

\end{document}