\documentclass{beamer}
\usepackage{beamerthemesplit}
\usepackage{wrapfig}
\usetheme{SPbGU}
\usepackage{pdfpages}
\usepackage{amsmath}
\usepackage{mathtools}
\usepackage{cmap}
\usepackage[T2A]{fontenc}
\usepackage[utf8]{inputenc}
\usepackage[english,russian]{babel}
\usepackage{indentfirst}
\usepackage{amsmath}
\usepackage{tikz}
\usepackage{multirow}
\usepackage[noend]{algpseudocode}
\usepackage{algorithm}
\usepackage{algorithmicx}
\usepackage{ stmaryrd }
\usepackage{fancyvrb}
\usepackage{qtree}
\usepackage{verbatim}
\usepackage{ulem}
\usepackage{proof}

\beamertemplatenavigationsymbolsempty

\setbeamertemplate{itemize item}[circle]
\setbeamertemplate{enumerate item}[circle]
\newcommand{\derives}[1][*]{\xRightarrow[]{#1}}

\def\To{\derives[]}
\def\iff{\Leftrightarrow}

\usetikzlibrary{shapes,arrows}
\usetikzlibrary{positioning,automata}
\tikzset{every state/.style={minimum size=0.2cm},
initial text={}
}

\graphicspath{
  {pics/}
}

\newcommand{\incimage}[2][0.8]{
  \begin{center}
    \includegraphics[width=\textwidth, height=#1\textheight, keepaspectratio]{#2}
  \end{center}
  }

\title[]{Унификация посредством поиска путей с контекстно-свободными ограничениями в графе \\ Source-tracking unification}
\subtitle[]{}
\institute[]{
Лаборатория языковых инструментов JetBrains\\
}

\author[]{Екатерина Вербицкая}

\date{6 ноября 2020}

\definecolor{orange}{RGB}{179,36,31}

\begin{document}
{
  \begin{frame}
    \titlepage
  \end{frame}
}


\begin{frame}[fragile]
  \frametitle{TLDR}
\begin{center}
  Задачу унификации можно свести к \\ поиску путей с КС ограничениями в графе\footnote{Choppella, V., and Haynes, C. T. (2005). Source-tracking unification.}
\end{center}
\end{frame}

\begin{frame}[fragile]
  \frametitle{План докалада}
\begin{itemize}
  \item Что такое унификация
  \item Как задача унификации представима в виде графа
  \item Какой язык будем использовать в качестве ограничений
  \item Почему это работает
  \item Какую дополнительную информацию можно получить из пути
\end{itemize}

\end{frame}

\begin{frame}[fragile]
  \frametitle{Унификация}

  \begin{center}
    Даны два терма $t, s$
  \end{center}

  \begin{center}
    Задача: найти подстановку на свободных переменных термов (унификатор) $\theta$, такую что
  \end{center}
  \[
    t \theta = s \theta
  \]
\end{frame}


\begin{frame}[fragile]
  \frametitle{Подстановка}
  \[
    \text{Терм: } \mathcal{T} :: \mathcal{V} \mid \mathcal{F}^n \ \mathcal{T}_1 \dots \mathcal{T}_n
  \]

  \[
    \text{Подстановка: } \theta :: \mathcal{V} \to \mathcal{T}
  \]


  \begin{center}
    Применение подстановки $t\{x_1 \mapsto t_1, \dots, x_k \mapsto t_k\}$: \\ одновременно заменить свободные переменные $x_i$ терма $t$ на $t_i$
  \end{center}
  \[
     (f \ x \ a \ (g \ z) \ y)\{x \mapsto h \ a \ y, z \mapsto y\} = f \ (h \ a  \ y) \ a \ (g \ y) \ y
  \]
\end{frame}

\begin{frame}[fragile]
  \frametitle{Применение унификации}
\begin{verbatim}
apply :: (a -> b) -> a -> b
apply f x = f x

f :: Int -> Int
f x = x + 1

apply_f :: ?
apply_f = apply f
\end{verbatim}

\bigskip

Унифицируем \verb!a -> b! и \verb!Int -> Int!, получаем \verb!a == Int!, \verb!b == Int!

\bigskip

\verb!apply_f :: Int -> Int!
\end{frame}



\begin{frame}[fragile]
  \frametitle{Простой алгоритм унификации}

\begin{center}
  Будем искать подстановку как множество уравнений $\mathcal{E} = \{ t_i = s_i \}$
\end{center}

\begin{itemize}
  \item Упрощение термов: $(f \ t_1 \dots t_n = g \ s_1 \dots s_m) \in \mathcal{E}$
  \begin{itemize}
    \item Если $f, g$ --- различные константы, то $\mathcal{E} = \bot$
    \item Иначе заменяем уравнение в $\mathcal{E}$ на множество $t_1 = s_1, \dots, t_n = s_n$
  \end{itemize}
  \item Переориентация: $(t = x) \in \mathcal{E}$
  \begin{itemize}
    \item Если $t$ --- терм, $x$ --- переменная, заменяем в $\mathcal{E}$ уравнение на $x = t$
  \end{itemize}
  \item Элиминация переменных: $(x = t) \in \mathcal{E}$, $x$ входит в какое-то уравнение
  \begin{itemize}
    \item Если $x$ входит в $t$, $t \equiv x$, то удаляем уравнение из $\mathcal{E}$
    \item Иначе, если $x$ входит в $t$, то $\mathcal{E} = \bot$
    \item Иначе, подставляем $t$ вместо $x$ во всех уравнениях в $\mathcal{E}$
  \end{itemize}
\end{itemize}

\end{frame}

\begin{frame}[fragile]
  \frametitle{Унификация: пример}
\[
  \{ node \ El \ T \ T = node \ 1 \ (node \ 2 \ emp \ emp) \ (node \ 2 \ emp \ emp) \}
\]

\[
  \{ El = 1, T = node \ 2 \ emp \ emp, T = node \ 2 \ emp \ emp \}
\]

\[
  \{ El = 1, T = node \ 2 \ emp \ emp, node \ 2 \ emp \ emp = node \ 2 \ emp \ emp \}
\]

\[
  \{ El = 1, T = node \ 2 \ emp \ emp, 2 = 2, emp = emp, emp = emp \}
\]

\[
  \{ El = 1, T = node \ 2 \ emp \ emp \}
\]
\end{frame}


\begin{frame}[fragile]
  \frametitle{Унификация: пример}
\[
  \{ node \ El \ T \ T = node \ 1 \ (node \ 2 \ emp \ emp) \ (node \ 3 \ emp \ emp) \}
\]

\[
  \{ El = 1, T = node \ 2 \ emp \ emp, T = node \ 3 \ emp \ emp \}
\]

\[
  \{ El = 1, T = node \ 2 \ emp \ emp, node \ 2 \ emp \ emp = node \ 3 \ emp \ emp \}
\]

\[
  \{ El = 1, T = node \ 2 \ emp \ emp, 2 = 3, emp = emp, emp = emp \}
\]

\[
  \bot
\]
\end{frame}


\begin{frame}[fragile]
  \frametitle{Чем плох простой алгоритм}
  \begin{itemize}
    \item Не очень эффективный
    \item Не говорит, почему унификация не завершилась успехом
  \end{itemize}
\end{frame}


\begin{frame}[fragile]
  \frametitle{Граф унификации}

\end{frame}

\end{document}