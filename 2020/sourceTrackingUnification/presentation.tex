\documentclass{beamer}
\usepackage{beamerthemesplit}
\usepackage{wrapfig}
\usetheme{SPbGU}
\usepackage{pdfpages}
\usepackage{amsmath}
\usepackage{mathtools}
\usepackage{cmap}
\usepackage[T2A]{fontenc}
\usepackage[utf8]{inputenc}
\usepackage[english,russian]{babel}
\usepackage{indentfirst}
\usepackage{amsmath}
\usepackage{tikz}
\usetikzlibrary{shapes,arrows,automata,positioning,quotes,backgrounds,decorations.text,decorations.pathmorphing,calc}
\usepackage{multirow}
\usepackage[noend]{algpseudocode}
\usepackage{algorithm}
\usepackage{algorithmicx}
\usepackage{ stmaryrd }
\usepackage{fancyvrb}
\usepackage{verbatim}
\usepackage{ulem}
\usepackage{proof}
\usepackage{xspace}

\beamertemplatenavigationsymbolsempty

\setbeamertemplate{itemize item}[circle]
\setbeamertemplate{enumerate item}[circle]
\newcommand{\derives}[1][*]{\xRightarrow[]{#1}}

\def\To{\derives[]}
\def\iff{\Leftrightarrow}

\tikzset{every state/.style={minimum size=0.2cm},
initial text={}
}

\tikzset{fnode/.style={shape=circle}}

\tikzset{eqedge/.style={thick, dashed}}
\tikzset{funedge/.style={thick}}
\tikzset{deredge/.style={thick, dashed, teal}}


\tikzstyle{processTree} = [
  ->,
  sibling distance=5em,
  % scale=0.6,
  every node/.style = {
    shape=rectangle,
    rounded corners=0.05cm,
    draw,
    align=center,
    minimum size=5mm,
    % scale=0.6,
    },
  ]

\newcommand{\iseq}{\stackrel{?}{=}}

\graphicspath{
  {pics/}
}

\newcommand{\incimage}[2][0.8]{
  \begin{center}
    \includegraphics[width=\textwidth, height=#1\textheight, keepaspectratio]{#2}
  \end{center}
  }

\title[]{Унификация посредством поиска путей с контекстно-свободными ограничениями в графе \\ Source-tracking unification}
\subtitle[]{}
\institute[]{
Лаборатория языковых инструментов JetBrains\\
}

\author[]{Екатерина Вербицкая}

\date{6 ноября 2020}

\definecolor{orange}{RGB}{179,36,31}

\begin{document}
{
  \begin{frame}
    \titlepage
  \end{frame}
}


\begin{frame}[fragile]
  \frametitle{TLDR}
\begin{center}
  Задачу унификации можно свести к \\ поиску путей с КС ограничениями в графе\footnote{Choppella, V., and Haynes, C. T. (2005). Source-tracking unification.}
\end{center}
\end{frame}

\begin{frame}[fragile]
  \frametitle{План докалада}
\begin{itemize}
  \item Что такое унификация
  \item Как задача унификации представима в виде графа
  \item Какой язык будем использовать в качестве ограничений
  \item Почему это работает
  \item Какую дополнительную информацию можно получить из пути
\end{itemize}

\end{frame}

\begin{frame}[fragile]
  \frametitle{Унификация}

  \begin{center}
    Даны два терма $t, s$
  \end{center}

  \begin{center}
    Задача: найти подстановку на свободных переменных термов (унификатор) $\theta$, такую что
  \end{center}
  \[
    t \theta = s \theta
  \]
\end{frame}


\begin{frame}[fragile]
  \frametitle{Подстановка}
  \[
    \text{Терм: } \mathcal{T} :: \mathcal{V} \mid \mathcal{F}^n \ \mathcal{T}_1 \dots \mathcal{T}_n
  \]

  \[
    \text{Подстановка: } \theta :: \mathcal{V} \to \mathcal{T}
  \]


  \begin{center}
    Применение подстановки $t\{x_1 \mapsto t_1, \dots, x_k \mapsto t_k\}$: \\ одновременно заменить свободные переменные $x_i$ терма $t$ на $t_i$
  \end{center}
  \[
     (f \ x \ a \ (g \ z) \ y)\{x \mapsto h \ a \ y, z \mapsto y\} = f \ (h \ a  \ y) \ a \ (g \ y) \ y
  \]
\end{frame}

\begin{frame}[fragile]
  \frametitle{Применение унификации}
\begin{verbatim}
apply :: (a -> b) -> a -> b
apply f x = f x

f :: Int -> Int
f x = x + 1

apply_f :: ?
apply_f = apply f
\end{verbatim}

\bigskip

Унифицируем \verb!a -> b! и \verb!Int -> Int!, получаем \verb!a == Int!, \verb!b == Int!

\bigskip

\verb!apply_f :: Int -> Int!
\end{frame}



\begin{frame}[fragile]
  \frametitle{Простой алгоритм унификации}

\begin{center}
  Будем искать подстановку как множество уравнений $\mathcal{E} = \{ t_i = s_i \}$
\end{center}

\begin{itemize}
  \item Упрощение термов: $(f \ t_1 \dots t_n = g \ s_1 \dots s_m) \in \mathcal{E}$
  \begin{itemize}
    \item Если $f, g$ --- различные константы, то $\mathcal{E} = \bot$
    \item Иначе заменяем уравнение в $\mathcal{E}$ на множество $t_1 = s_1, \dots, t_n = s_n$
  \end{itemize}
  \item Переориентация: $(t = x) \in \mathcal{E}$
  \begin{itemize}
    \item Если $t$ --- терм, $x$ --- переменная, заменяем в $\mathcal{E}$ уравнение на $x = t$
  \end{itemize}
  \item Элиминация переменных: $(x = t) \in \mathcal{E}$, $x$ входит в какое-то уравнение
  \begin{itemize}
    \item Если $x$ входит в $t$, $t \equiv x$, то удаляем уравнение из $\mathcal{E}$
    \item Иначе, если $x$ входит в $t$, то $\mathcal{E} = \bot$
    \item Иначе, подставляем $t$ вместо $x$ во всех уравнениях в $\mathcal{E}$
  \end{itemize}
\end{itemize}

\end{frame}

\begin{frame}[fragile]
  \frametitle{Унификация: пример}
\[
  \{ node \ El \ T \ T = node \ 1 \ (node \ 2 \ emp \ emp) \ (node \ 2 \ emp \ emp) \}
\]

\[
  \{ El = 1, T = node \ 2 \ emp \ emp, T = node \ 2 \ emp \ emp \}
\]

\[
  \{ El = 1, T = node \ 2 \ emp \ emp, node \ 2 \ emp \ emp = node \ 2 \ emp \ emp \}
\]

\[
  \{ El = 1, T = node \ 2 \ emp \ emp, 2 = 2, emp = emp, emp = emp \}
\]

\[
  \{ El = 1, T = node \ 2 \ emp \ emp \}
\]
\end{frame}


\begin{frame}[fragile]
  \frametitle{Унификация: пример}
\[
  \{ node \ El \ T \ T = node \ 1 \ (node \ 2 \ emp \ emp) \ (node \ 3 \ emp \ emp) \}
\]

\[
  \{ El = 1, T = node \ 2 \ emp \ emp, T = node \ 3 \ emp \ emp \}
\]

\[
  \{ El = 1, T = node \ 2 \ emp \ emp, node \ 2 \ emp \ emp = node \ 3 \ emp \ emp \}
\]

\[
  \{ El = 1, T = node \ 2 \ emp \ emp, 2 = 3, emp = emp, emp = emp \}
\]

\[
  \bot
\]
\end{frame}


\begin{frame}[fragile]
  \frametitle{Чем плох простой алгоритм}
  \begin{itemize}
    \item Не очень эффективный
    \item Не говорит, почему унификация не завершилась успехом
  \end{itemize}
\end{frame}


\begin{frame}[fragile]
  \frametitle{Граф унификации}
\begin{center}
  $a \to b \iseq Int \to Int$
\end{center}

\bigskip

\begin{center}
\begin{tikzpicture}[
  processTree,
  anchor=center]
  \node (0) [fnode] {$\to$}
    child { node (00) {a} }
    child { node (01) {b} };
  \node (1) [fnode] [right of=0, xshift=3cm] {$\to$}
    child { node [fnode] (10) {Int} }
    child { node [fnode] (11) {Int} };
  level 1/.style={sibling distance=5em},
  \draw [eqedge] (0.east) to node[above, draw=none, black] {$\varepsilon$} (1.west);
  \path
  (0) edge [funedge, left]  node [draw=none] {$1$}  (00)
      edge [funedge, right] node [draw=none] {$2$}  (01)
  (1) edge [funedge, left]  node [draw=none] {$1$}  (10)
      edge [funedge, right] node [draw=none] {$2$}  (11);
  \pause
  \draw [deredge] (00.south east) [out=315,in=225] to (10.south west);
  \draw [deredge] (01.south east) [out=315,in=225] to (11.south west);
\end{tikzpicture}
\end{center}
\end{frame}

\begin{frame}[fragile]
  \frametitle{Отношение эквивалентности на вершинах}
Отношение на вершинах $R$ \textit{замкнуто вниз}, если для любой метки на ребре $\delta$ и двух вершин в отношении $uRu'$ с ребрами $u \xrightarrow{\delta} v$ и $u' \xrightarrow{\delta} v'$ верно $vRv'$

\begin{center}
  \begin{tikzpicture}[
    processTree,
    anchor=center]
    \node (0) {$u$}
      child { node (00) {$v$} };
    \node (1) [right of=0, xshift=1.5cm] {$u'$}
      child { node (10) {$v'$} };
    level 1/.style={sibling distance=5em},
    \draw [deredge] (0.east) to (1.west);
    \path
    (0) edge [funedge, left]  node [draw=none] {$\delta$}  (00)
    (1) edge [funedge, right]  node [draw=none] {$\delta$}  (10);
    \pause
    \draw [deredge] (00.east) to (10.west);
  \end{tikzpicture}
\end{center}

\bigskip

\textit{Замыкание унификации отношения} $R$ это наименьшее замкнутое вниз отношение на вершинах, содержащее $R$
\end{frame}

\begin{frame}[fragile]
  \frametitle{Факторграф унификации}
Вершины \textit{равны}, если связаны $\varepsilon$-ребром

\bigskip

\textit{Факторизуем} граф унификации по отношению эквивалентности на вершинах, которое построено как замыкание унификации отношения равенства вершин

\pause

\begin{center}
  \begin{tikzpicture}[
  processTree,
  anchor=center,
  decoration = {
    snake,
    pre length=2pt,
    post length=2pt
  },]
  \node (0) [fnode] {$\to$}
    child { node (00) {a} }
    child { node (01) {b} };
  \node (1) [fnode] [right of=0, xshift=3cm] {$\to$}
    child { node [fnode] (10) {Int} }
    child { node [fnode] (11) {Int} };
  level 1/.style={sibling distance=5em},
  \draw [eqedge] (0.east) to node[above, draw=none, black] {$\varepsilon$} (1.west);
  \path
  (0) edge [funedge, left]  node [draw=none] {$1$}  (00)
      edge [funedge, right] node [draw=none] {$2$}  (01)
  (1) edge [funedge, left]  node [draw=none] {$1$}  (10)
      edge [funedge, right] node [draw=none] {$2$}  (11);
  \draw [deredge] (00.south east) [out=315,in=225] to (10.south west);
  \draw [deredge] (01.south east) [out=315,in=225] to (11.south west);

  \node (12) [fnode, right of=11, xshift=3cm, yshift=1cm] {$a \ b \ int$};
  \draw [decorate] ($(1.east)+(1,-0.5)$) to node [draw=none,above] {факторизация} ($(12.west)+(-0.5,0)$);
\end{tikzpicture}
\end{center}


\end{frame}

\end{document}