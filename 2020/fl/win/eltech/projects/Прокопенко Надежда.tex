\documentclass[12pt]{article}
\usepackage[left=2cm,right=2cm,top=1cm,bottom=1cm,bindingoffset=0cm]{geometry}
\usepackage[utf8x]{inputenc}
\usepackage[english,russian]{babel}
\usepackage{cmap}
\usepackage{amssymb}
\usepackage{amsmath}
\usepackage{pifont}
\usepackage{tikz}
\usepackage{verbatim}

\pagenumbering{gobble}

\begin{document}

\begin{center}
{\LARGE Формальные языки}

{\Large Проект для Прокопенко Надежды}

\end{center}

\bigskip

\begin{enumerate}
  \item {Реализовать синтаксический анализатор для языка с базовым синтаксисом и синтаксическим сахаром для циклов. }
  \begin{itemize}
    \item Помимо цикла c предусловием \verb!while! вам предстоит поддержать цикл с постусловием \verb!repeat!. Отличие цикла с постусловием от цикла с предусловием в том, что его тело будет обязательно исполнено хотя бы один раз, вне зависимости от значения выражения в условии.
    \begin{itemize}
      \item В абстрактном синтаксическом дереве не должно быть отдельного узла для цикла \verb!repeat!.
      \item Корректная проекция цикла выглядит следующим образом:

      \verb!repeat! $(condition) \ body \to $ $body;$ \verb!while! $(condition) \ body$

      Тут $condition$ --- условие в цикле, $body$ --- тело цикла.
    \end{itemize}
    \item Можно использовать любой способ писать парсер.
    \item Для языка предложить свой конкретный синтаксис. Описание конкретного синтаксиса прислать не позднее 20 декабря.
    \item Реализацию парсера прислать не позднее 23:59 9 января.
    \item Не позднее дня перед экзаменом надо будет созвониться, чтобы вы мне рассказали, что как работает. Созваниваться будем после того, как будет зачтен код.
    \item Скорее всего потребуется несколько итераций проверки, не затягивайте.
    \item В конце файла требования к оформлению, их надо соблюдать.
  \end{itemize}

\end{enumerate}

\begin{center}
  {\Large Описание базового синтаксиса}
\end{center}
\begin{itemize}
  \item Лексический синтаксис
  \begin{itemize}
    \item Идентификатор --- непустая последовательность букв латинского алфавита в любом регистре, цифр и символа нижнего подчеркивания (\verb!_!), начинающаяся на букву латинского алфавита в нижнем регистре, не являющаяся ключевым словом.
    \begin{itemize}
      \item Корректные идентификаторы: \verb!x!, \verb!list!, \verb!listNat_123!.
      \item Некорректные идентификаторы: \verb!Abc!, \verb!123!, \verb!_List!.
    \end{itemize}
    \item Число: натуральное или ноль в десятичной системе счисления, не может содержать лидирующие нули.
    \begin{itemize}
      \item Корректные числа: \verb!123!, \verb!0!.
      \item Некорректные числа: \verb!-1!, \verb!007!, \verb!89A!.
    \end{itemize}
    \item Ключевые слова не могут быть идентификаторами. Конкретные ключевые слова вы выбираете сами.
    \item Операторы языка:
      \begin{itemize}
        \item сложение \verb!+!,
        \item умножение \verb!*!,
        \item деление \verb!/!,
        \item вычитание \verb!-!,
        \item возведение в степень \verb!**!,
        \item конъюнкция \verb!&&!,
        \item дизъюнкция \verb!||!,
        \item логическое отрицание \verb!--!,
        \item операторы сравнения: \verb!<!, \verb!<=!, \verb!==!, \verb!/=!, \verb!>!, \verb!>=!,
      \end{itemize}
    \item Пробелы не являются значимыми, но не могут встречаться внутри одной лексемы.
  \end{itemize}
  \item Базовый абстрактный синтаксис
  \begin{itemize}
    \item Программа --- непустая последовательность определений функций.
    \item Определение функции содержит ее сигнатуру и тело. Сигнатура функции содержит ее название (идентификатор) и список аргументов (может быть пустым). Тело~--- последовательность инструкций (может быть пустой).
    \item Инструкции:
    \begin{itemize}
      \item Присвоение значения арифметического выражения переменной. Переменная может быть произвольным идентификатором.
      \item Возвращение значения из функции.
      \item Условное выражение с обязательной веткой \verb!else!. Условием является арифметическое выражение. В ветках --- произвольные последовательности инструкций (могут быть пустыми).
      \item Цикл с предусловием. Условием является арифметическое выражение. Тело цикла --- произвольная последовательность инструкций (может быть пустой).
    \end{itemize}
    \item Арифметические выражения заданы над числами и идентификаторами, операторы перечислены в таблице ниже с указанием их приоритетов, арности и ассоциативности.

    \begin{center}
      \begin{tabular}{ c | c | c }
        Наибольший приоритет & Арность & Ассоциативность  \\ \hline \hline
        \verb!-! & Унарная & \\
        \verb!**! & Бинарная &Правоассоциативная \\
       \verb!*!,\verb!/! & Бинарная & Левоассоциативная \\
       \verb!+!,\verb!-! & Бинарная & Левоассоциативная \\
       \verb!==!,\verb!/=!, \verb!<!,\verb!<=!, \verb!>!,\verb!>=! & Бинарная & Неассоциативная \\
       \verb!--! & Унарная &  \\

       \verb!&&! & Бинарная & Правоассоциативная \\
       \verb!||! & Бинарная & Правоассоциативная \\
       \hline \hline
       Наименьший приоритет & Арность & Ассоциативность
      \end{tabular}
      \end{center}
  \end{itemize}
\end{itemize}

\bigskip

\begin{center}
  {\Large Требования к оформлению}
\end{center}

      \begin{itemize}
        \item Результатом должно быть консольное приложение, которое принимает на вход программу и печатает результат синтаксического анализа в файл с таким же названием и дополнительным расширением \verb!.out!. (\verb!input.txt! $\to$ \verb!input.txt.out!)
        \item Результатом синтаксического анализа является абстрактное синтаксическое дерево в случае успешного разбора и сообщение об ошибке иначе.
        \begin{itemize}
          \item Если произошла лексическая ошибка, то сообщить о ней и завершиться, не пытаясь парсить дальше.
          \item Если произошла любая ошибка --- сообщить о ней и завершиться, не пытаясь восстанавливаться после ошибки.
        \end{itemize}
        \item Код может быть написан на любом языке программирования с использованием любых инструментов, но должен быть сопровожден инструкцией по сборке и запуску. Желательно выложить его на гитхаб и сопроводить тестами.
        \item Ссылку на репозиторий присылать в письме с темой \verb![FL_ElTech] project!.
      \end{itemize}


\end{document}
