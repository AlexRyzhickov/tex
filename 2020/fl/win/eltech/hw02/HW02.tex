\documentclass[12pt]{article}
\usepackage[left=2cm,right=2cm,top=2cm,bottom=2cm,bindingoffset=0cm]{geometry}
\usepackage[utf8x]{inputenc}
\usepackage[english,russian]{babel}
\usepackage{cmap}
\usepackage{amssymb}
\usepackage{amsmath}
\usepackage{url}
\usepackage{pifont}
\usepackage{tikz}
\usepackage{verbatim}

\usetikzlibrary{shapes,arrows}
\usetikzlibrary{positioning,automata}
\tikzset{every state/.style={minimum size=0.2cm},
initial text={}
}

\begin{document}
\begin{center} {\LARGE Формальные языки} \end{center}

\begin{center} \Large домашнее задание до 23:59 29.10 \end{center}
\bigskip

\begin{enumerate}
  \item
  {
    Привести однозначную контекстно-свободную грамматику для языка арифметических выражений над положительными целыми \emph{числами} с операциями \verb!+!, \verb!-!, \verb!*!, \verb!/!, \verb!^!, \verb!==!,\verb!<>!, \verb!<!, \verb!<=!, \verb!>!, \verb!>=! со следующими приоритетами и ассоциативностью:

  \begin{center}
    \begin{tabular}{ c | c }
      Наибольший приоритет & Ассоциативность  \\ \hline \hline
      \verb!^! & Правоассоциативна \\
     \verb!*!,\verb!/! & Левоассоциативна \\
     \verb!+!,\verb!-! & Левоассоциативна \\
     \verb!==!,\verb!<>!, \verb!<!,\verb!<=!, \verb!>!,\verb!>=! & Неассоциативна \\ \hline \hline
     Наименьший приоритет & Ассоциативность
    \end{tabular}
    \end{center}


    Неассоциативные операции встречаются только один раз: \verb!1 == 2! -- корректная строка, \verb!1 == 2 == 3!, \verb!(1 == 2) == 3!, \verb!1 < 2 > 3! --- некорректные строки
  }
  \item Привести грамматику из 1 задания в нормальную форму Хомского.
  \item Промоделировать работу алгоритма CYK на грамматике из 2 задания на трех корректных строках не короче 7 символов и на трех некорректных строках. (Привести таблицы и деревья вывода)
\end{enumerate}

\bigskip

\begin{center}
  \Large{Примеры оформления грамматик и таблиц}
\end{center}

\begin{center}
  \begin{align*}
    S' &\to A B \mid \varepsilon \\
    S  &\to A B \\
    A  &\to L S \mid ( \\
    B  &\to R S \mid \ ) \\
    L  &\to ( \\
    R  &\to \ )
  \end{align*}
\end{center}


\begin{center}
  \begin{tabular}{ c || c | c | c | c | c | c | }
      &  1  &  2  &  3  &  4  &  5  &  6   \\ \hline \hline
    1 & A L &     &  A  &     &  A  & S S' \\ \hline
    2 &     & A L &  S  &     &  S  &      \\ \hline
    3 &     &     & B R &     &  B  &      \\ \hline
    4 &     &     &     & A L &  S  &      \\ \hline
    5 &     &     &     &     & B R &      \\ \hline
    6 &     &     &     &     &     & B R  \\ \hline
  \end{tabular}
  \end{center}

\end{document}
