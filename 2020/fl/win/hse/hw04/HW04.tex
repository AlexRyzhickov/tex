\documentclass[12pt]{article}
\usepackage[left=2cm,right=2cm,top=2cm,bottom=2cm,bindingoffset=0cm]{geometry}
\usepackage[utf8x]{inputenc}
\usepackage[english,russian]{babel}
\usepackage{cmap}
\usepackage{amssymb}
\usepackage{amsmath}
\usepackage{url}
\usepackage{pifont}
\usepackage{tikz}
\usepackage{verbatim}
\usepackage{hyperref}

\usetikzlibrary{shapes,arrows}
\usetikzlibrary{positioning,automata}
\tikzset{every state/.style={minimum size=0.2cm},
initial text={}
}

\begin{document}
\begin{center} {\LARGE Формальные языки} \end{center}

\begin{center} \Large домашнее задание до 23:59 01.10 \end{center}
\bigskip

\begin{enumerate}
  \item
  {
    Существует ли такой регулярный язык $L$, что язык всех его подстрок не является регулярным? Обосновать.
  }
  \item
  {
    Является ли язык, задаваемый следующей грамматикой регулярным? Если является, привести регулярное выражение или конечный автомат. Если нет --- обосновать.
    \begin{align*}
      S &\to T U \\
      U &\to T U \mid b \\
      T &\to U U \mid a
    \end{align*}
  }
  \item
  {
    Равны ли регулярные выражения $b^* a ((a \mid b) \ b^* a)^*$ и $((a \mid b)^* \ b a \mid a)(aa)^*$ над алфавитом $\{a, b\}$? Обосновать.
  }
  \item
  {
    Изучить спецификации синтаксиса бинарных операторов трех языков программирования, на которых вы никогда не писали. В отчете привести ссылки на спецификации, а также указать, что было в них для вас неожиданным.
  }
  \item
  {
    Релизовать синтаксический анализатор методом рекурсивного спуска для упрощенного донельзя подмножества пролога (оставим только определения отношений, лишим отношения аргументов).
    \begin{itemize}
      \item Описание синтаксиса
      \begin{itemize}
        \item Программа на прологе --- последовательность \emph{определений отношений}.
        \item Определение отношений состоит из \emph{головы} и \emph{тела}, разделенных штопором (\verb!:-!), в конце стоит точка (\verb!.!).
        \item Голова --- идентификатор.
        \item Тело --- выражение с правоассоциативными бинарными операторами конъюнкции (\verb!,!) и дизъюнкции (\verb!;!) над идентификаторами. Конъюнкция имеет более высокий приоритет, чем дизъюнкция. Возможно использование скобок (\verb!(!, \verb!)!) для управления порядком вычислений.
        \item Тело и штопор могут отсутствовать.
        \item Где угодно могут встречаться пробельные символы: советую использовать лексер.
      \end{itemize}
      \item Примеры корректных определений отношений.
      \begin{itemize}
        \item \verb!f.!
        \item \verb!f :- g.!
        \item \verb!f :- g, h; t.!
        \item \verb!f :- g, (h; t).!
      \end{itemize}
      \item Примеры некорректных определений отношений.
      \begin{itemize}
        \item \verb!f! --- нет точки.
        \item \verb!:- f.! --- нет головы.
        \item \verb!f :- .! --- нет тела.
        \item \verb!f :- g; h, .! --- нет правого подвыражения у конъюнкции.
        \item \verb!f :- (g; (f).! --- несбалансированные скобки.
      \end{itemize}
      \item Код должен быть сопровожден инструкцией по сборке и запуску. Желательно выложить его на гитхаб и сопроводить тестами.
      \item Примеры простых реализаций парсеров методом рекурсивного спуска:
      \begin{itemize}
        \item \url{https://vey.ie/2018/10/04/RecursiveDescent.html}
        \item \url{http://www.cs.utsa.edu/~wagner/CS3723/rdparse/rdparser6.html}
        \item \url{https://prgwonders.blogspot.com/2017/10/recursive-descent-parser-in-c-for.html}
        \item \url{https://craftinginterpreters.com/parsing-expressions.html}
      \end{itemize}
    \end{itemize}
  }
\end{enumerate}


\end{document}
