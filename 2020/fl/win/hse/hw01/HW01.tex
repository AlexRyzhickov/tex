\documentclass[12pt]{article}
\usepackage[left=2cm,right=2cm,top=2cm,bottom=2cm,bindingoffset=0cm]{geometry}
\usepackage[utf8x]{inputenc}
\usepackage[english,russian]{babel}
\usepackage{cmap}
\usepackage{amssymb}
\usepackage{amsmath}
\usepackage{url}
\usepackage{pifont}
\usepackage{tikz}
\usepackage{verbatim}

\usetikzlibrary{shapes,arrows}
\usetikzlibrary{positioning,automata}
\tikzset{every state/.style={minimum size=0.2cm},
initial text={}
}

\begin{document}
\begin{center} {\LARGE Формальные языки} \end{center}

\begin{center} \Large домашнее задание до 23:59 10.09 \end{center}
\bigskip

\begin{enumerate}
  \item
  {
    Перечислить слова языка $L_1 \cap L_2,$ где $L_1 = \{ (cat)^n \mid n \geq 0 \}$ и $L_2 = \{ c^m a^m t^m \mid m \geq 0 \}$. Доказать, что других цепочек в пересечении нет.
  }
  \item
  {
    Почитать спецификацию синтаксиса вашего третьего самого любимого языка. Найти особенности лексического синтаксиса, о которых вы раньше не знали. В отчете описать особенности и привести ссылку на спецификацию.
  }
  \item
  {
    Построить детерминированный конечный автомат, который распознает язык неотрицательных чисел \emph{без лидирующих нулей}, делящихся на 3.
  }
  \item
  {
    Существуют ли такие языки $L_1$ и $L_2$, что $L_1^R \cdot (L_2^R)$ и $L_1^R \cdot L_2$ неравномощны? Обосновать ответ.
  }
\end{enumerate}

\newpage

\begin{center}
  \Large{Пример рисования автоматов при помощи \texttt{tikz}}
\end{center}

\bigskip

Пример автомата:

\begin{center}
  \begin{tikzpicture}[> = stealth,node distance=3cm, on grid]
    \node[state]           (q_2)                      {C};
    \node[state,initial]   (q_0) [above left=of q_2]  {A};
    \node[state]           (q_1) [below left=of q_2]  {B};
    \node[state]           (q_3) [right=of q_2]       {D};
    \node[state]           (q_4) [above right=of q_3] {E};
    \node[state,accepting] (q_5) [below right=of q_3] {F};
    \node[state,accepting] (q_6) [above right=of q_5] {G};

    \path[->] (q_0) edge [bend left=15]  node [right] {$1$} (q_1)
                    edge                 node [above] {$0$} (q_2)
              (q_1) edge [bend left=15]  node [left]  {$1$} (q_0)
                    edge                 node [below] {$0$} (q_2)
              (q_2) edge [bend right=15] node [below] {$1$} (q_3)
                    edge [bend left=15]  node [above] {$0$} (q_3)
              (q_3) edge                 node [below] {$1$} (q_5)
                    edge                 node [above] {$0$} (q_4)
              (q_4) edge                 node [above] {$1$} (q_6)
                    edge                 node [right] {$0$} (q_5)
              (q_5) edge [loop below]    node         {$1$} ()
                    edge [loop left]     node         {$0$} ()
              (q_6) edge                 node [below] {$1$} (q_5)
                    edge [loop right]    node         {$0$} ();
  \end{tikzpicture}
\end{center}

Расположение узлов относительно друг друга:

\begin{center}
  \begin{tikzpicture}[> = stealth,node distance=3cm, on grid]
    \node (center)                  {center};
    \node [above=of center]         {above};
    \node [above left=of center]    {above left};
    \node [above right=of center]   {above right};
    \node [below=of center]         {below};
    \node [below left=of center]    {below left};
    \node [below right=of center]   {below right};
    \node [left=of center]          {left};
    \node (right) [right=of center] {right};
    \node [right=of right]          {right of right};
  \end{tikzpicture}
\end{center}

Пример автомата с лекции:

\begin{center}
  \begin{tikzpicture}[> = stealth,node distance=4cm, on grid]
    \node[state,initial,accepting] (q_0) {0};
    \node[state] (q_1) [above right=of q_0, xshift=1.5cm] {1};
    \node[state] (q_2) [below right=of q_0, xshift=1.5cm] {2};

    \path[->] (q_0) edge [loop above]   node []             {$0,3,6,9$} ()
                    edge [bend left=15] node [sloped,above] {$1,4,7$} (q_1)
                    edge [bend left=15] node [sloped,above] {$2,5,8$} (q_2)
              (q_1) edge [loop right]   node []             {$0,3,6,9$} ()
                    edge [bend left=15] node [sloped,above] {$1,4,7$} (q_2)
                    edge [bend left=15] node [sloped,below] {$2,5,8$} (q_0)
              (q_2) edge [loop right]   node []             {$0,3,6,9$} ()
                    edge [bend left=15] node [sloped,below] {$1,4,7$} (q_0)
                    edge [bend left=15] node [sloped,above] {$2,5,8$} (q_1)

                    ;
  \end{tikzpicture}
\end{center}

\end{document}
