\documentclass[12pt]{article}
\usepackage[left=2cm,right=2cm,top=2cm,bottom=2cm,bindingoffset=0cm]{geometry}
\usepackage[utf8x]{inputenc}
\usepackage[english,russian]{babel}
\usepackage{cmap}
\usepackage{amssymb}
\usepackage{amsmath}
\usepackage{url}
\usepackage{pifont}
\usepackage{tikz}
\usepackage{verbatim}
\usepackage{hyperref}

\usetikzlibrary{shapes,arrows}
\usetikzlibrary{positioning,automata}
\tikzset{every state/.style={minimum size=0.2cm},
initial text={}
}

\begin{document}
\begin{center} {\LARGE Формальные языки} \end{center}

\begin{center} \Large домашнее задание до 23:59 24.09 \end{center}
\bigskip

\begin{enumerate}
  \item
  {
    Определить, какие из языков являются регулярными. Если язык регулярен, то привести для него регулярное выражение, иначе --- доказать нерегулярность.
    \begin{itemize}
      \item $\{ u a v \mid u, v \in \{ a, b\}^*, |u| = |v| \}$.
      \item Язык арифметических выражений над целыми числами (операции $+,*$) в префиксной записи. (\verb!"+ 1 * 2 3"!, \verb!"123"!, \verb!"* 1 * 2 3"! --- предложения языка).
      \item $\{u u^R v \mid u \in \{a, b\}^+, v \in \{a, b\}^*\}$
    \end{itemize}
  }
  \item {
    Существует ли над двухбуквенным алфавитом такой язык $L$, что $L^*$ не является регулярным? Обосновать.
  }
  \item
  {
    Реализовать на любимом языке программирования подход parsing with derivatives для регулярных языков.
    \begin{itemize}
      \item Реализовать тесты.
      \item Найти входные данные (регулярное выражение и строку), на которых распознавание занимает больше 2 секунд.
      \item Изучить, влияет ли на время распознавания сложность регулярного выражения (например, выражение \verb!(a|a)* a (a|a)*! сложнее, чем \verb!a+!, при этом распознает тот же язык).
      \item В отчете описать проведенные эксперименты.
      \item Реализация с пары тут: \url{https://github.com/kajigor/fl-2020-hse-win/tree/L03}
    \end{itemize}

  }
\end{enumerate}


\end{document}
