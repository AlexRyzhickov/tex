\documentclass[dvipsnames]{beamer}
\usepackage{beamerthemesplit}
\usepackage{wrapfig}
\usetheme{SPbGU}
\usepackage{pdfpages}
\usepackage{amsmath}
\usepackage{mathtools}
\usepackage{cmap}
\usepackage[T2A]{fontenc}
\usepackage[utf8]{inputenc}
\usepackage[english,russian]{babel}
\usepackage{indentfirst}
\usepackage{amsmath}
\usepackage{tikz}
\usepackage{multirow}
\usepackage[noend]{algpseudocode}
\usepackage{algorithm}
\usepackage{algorithmicx}
\usepackage{stmaryrd}
\usepackage{qtree}
\usetikzlibrary{shapes,arrows}
\usepackage{fancyvrb}
\usepackage{xcolor}
\usepackage{stmaryrd}
\usepackage{array}

\usetikzlibrary{positioning}
\usetikzlibrary{decorations.pathmorphing}
\usetikzlibrary{arrows.meta}

\usetikzlibrary{shapes,arrows}
\usetikzlibrary{positioning,automata}
\tikzset{
  snake it/.style={decorate, decoration=snake},
  every state/.style={minimum size=0.2cm},
  initial text={}
}

\newtheorem{rutheorem}{Теорема}
\newtheorem{ruproof}{Доказательство}
\newtheorem{rudefinition}{Определение}
\newtheorem{rulemma}{Лемма}

\beamertemplatenavigationsymbolsempty

\setbeamertemplate{itemize item}[circle]
\setbeamertemplate{enumerate item}[circle]

\newcommand{\derive}[0]{\xRightarrow[]{*}}
\newcommand{\derives}[0]{\xRightarrow[]{}}
\newcommand{\derivek}[1]{\xRightarrow[]{#1}}
\newcommand{\deriveg}[1]{\xRightarrow[#1]{*}}
\newcommand{\derivegone}[1]{\xRightarrow[#1]{}}

\newcommand{\state}[1]{{\color{red}{#1}}}
\newcommand{\stack}[1]{{\color{blue}{#1}}}
\newcommand{\symbl}[1]{{\color{PineGreen}{#1}}}
\newcommand{\conf}[3]{( \state{#1},\symbl{#2},\stack{#3} )}
\newcommand{\trans}[3]{\delta ( \state{#1},\symbl{#2},\stack{#3} )}

\newenvironment{myauto}[1][3]
{
  \begin{center}
    \begin{tikzpicture}[> = stealth,node distance=#1cm, on grid, very thick]
}
{
    \end{tikzpicture}
  \end{center}
}


\title[]{Теория автоматов и формальных языков}
\subtitle[]{Магазинные автоматы}
\institute[]{
Санкт-Петербургский государственный электротехнический университет <<ЛЭТИ>>\\
}

\author[]{Екатерина Вербицкая}

\date{21 ноября 2019}

\definecolor{orange}{RGB}{179,36,31}

\begin{document}
{
  \begin{frame}
    \titlepage
  \end{frame}
}

\begin{frame}[fragile]
  \frametitle{В предыдущей серии}
  \begin{itemize}
    \item Регулярные языки распознаются с помощью конечных автоматов
    \item Разные алгоритмы синтаксического анализа для контекстно-свободных языков
    \begin{itemize}
    	\item CYK
    	\item Рекурсивный спуск
    	\item LL(1)
    	\item LR(0), SLR(1), CLR(1), LALR(1)
    \end{itemize}
    \item Есть ли универсальный распознаватель для КС-языков?
  \end{itemize}
\end{frame}

\begin{frame}[fragile]
  \frametitle{TLDR}
  \begin{itemize}
  	\item Произвольный КС язык можно распознать при помощи магазинного автомата (он же автомат с магазинной памятью, он же pushdown automata, он же pda)
  	\item Магазинный автомат по сути --- автомат со стеком
  	\item Детерминированные магазинные автоматы могут распознавать только детерминированные КС языки
  	\item Недетерминированные магазинные автоматы могут распознавать произольные КС языки
  \end{itemize}
\end{frame}

\begin{frame}[fragile]
  \frametitle{Что такое магазинный автомат}

  \begin{center}
    \begin{tikzpicture}[> = stealth,very thick]

      \draw[draw=black] (0,0) rectangle ++(4,3);

      \draw[draw=black] (-5,-3) rectangle ++(7,1);
      \draw[draw=black] (-4,-3) rectangle ++(1.5,1);
      \draw[draw=black] (-2.5,-3) rectangle ++(1,1);
      \draw[draw=black] (-1.5,-3) rectangle ++(1,1);
      \draw[draw=black] (1,-3) rectangle ++(1,1);


      \draw[draw=black] (5.5,-3.5) rectangle ++(1,5);
      \draw[draw=black] (5.5,-2.5) rectangle ++(1,2);
      \draw[draw=black] (5.5,-0.5) rectangle ++(1,1);
      \draw (5.3,-3.5) -- (6.7, -3.5);



      \node[draw=none, fill=none] at (-1, 3) {\huge{finite}};
      \node[draw=none, fill=none] at (-1.2, 2.3) {\huge{control}};

      \node[draw=none, fill=none] at (0.5, 2.5) {$\mbox{\Huge $\delta$ }$};
      \node[draw=none, fill=none] at (3.7, 0.5) {$\mbox{\Huge $\state{p}$ }$};
      \node[draw=none, fill=none] at (6.1, 1) {$\mbox{\Huge $\stack{A}$ }$};
      \node[draw=none, fill=none] at (6.1, -3) {$\mbox{\Huge $\stack{\bot}$ }$};
      \node[draw=none, fill=none] at (6,-1.4) {$\mbox{\Huge $\vdots$}$};

      \node[draw=none, fill=none] at (-1, -2.5) {$\mbox{\Huge $\symbl{a}$}$};
      \node[draw=none, fill=none] at (-3.2, -2.5) {$\mbox{\LARGE $\dots$}$};
      \node[draw=none, fill=none] at (0.3, -2.5) {$\mbox{\LARGE $\dots$}$};

      \node[draw=none, fill=none] at (3.5, -0.5) {\huge{state}};
      \node[draw=none, fill=none] at (6, -4) {\huge{stack}};
      \node[draw=none, fill=none] at (6, 2) {\huge{top}};
      \node[draw=none, fill=none] at (-1.5, -3.5) {\huge{input tape}};

      \draw (3.55,0.5) circle (0.4);

      \draw [->, rounded corners=8mm] (0,0.5) -- (-1, 0.9) -- (-2.5, 0.8) -- (-1.4,-1) -- (-1.5, -2);
      \draw [->, rounded corners=4mm] (4,0.5) -- (5, 0.4) -- (4.2,1.2) -- (5.5, 1);
      \end{tikzpicture}
  \end{center}

\end{frame}

\begin{frame}[fragile]
  \frametitle{Что такое магазинный автомат: неформально}
\begin{itemize}
	\item Автомат, переходы которого осуществляются по входному символу, текущему состоянию и символу на вершине стека
	\begin{itemize}
		\item У конечного автомата не было стека
	\end{itemize}
	\item Никакие состояния стека, кроме вершины, не доступны
	\item Во время перехода может изменяться стек
	\begin{itemize}
		\item Положить что-то на стек (push)
		\item Снять верхушку со стека (pop)
	\end{itemize}
	\item А может и не изменяться
	\begin{itemize}
		\item Магазинный автомат может вообще игнорировать стек
		\item Или стек может не изменяться, хоть значение оттуда и читается
	\end{itemize}
	\item Итого: по тройке (входной символ, состояние, символ на вершине стека) получается новое состояние, и модифицируется (или нет) стек
\end{itemize}
\end{frame}

\begin{frame}[fragile]
  \frametitle{Формальное определение}
  \emph{Недетерминированный} магазинный автомат это $(Q, \Sigma, \Gamma, \delta, q_0, \bot, F)$
  \begin{itemize}
    \item $Q$ --- конечное множество состояний
    \item $\Sigma$ --- конечное множество символов, входной алфавит
    \item $\Gamma$ --- конечное множество символов, стековый алфавит
    \item $\delta: Q \times (\Sigma \cup \varepsilon) \times \Gamma \to 2^{Q \times \Gamma^*}$ --- функция переходов
    \item $q_0 \in Q$ --- стартовое состояние
    \item $\bot \in \Gamma$ --- начальный элемент стека
    \item $F \subseteq Q$ --- множество принимающих (конечных) состояний
  \end{itemize}
\end{frame}


\begin{frame}[fragile]
  \frametitle{Отношение переходов}

\begin{center}
    $(\state{q}, \stack{\alpha}) \in \trans{p}{a}{A}$  означает
\end{center}

  \begin{itemize}
    \item Если магазинный автомат находится в состоянии $p \in Q$,
    \item на вершине стека находится $A \in \Gamma$,
    \item а со входа читается символ $a \in \Sigma \cup \varepsilon$,
    \vspace{0.5cm}
    \item то изменяем состояние на $q \in Q$,
    \item снимаем со стека символ $A$, записываем на стек строку $\alpha \in \Gamma^*$
    \vspace{0.5cm}
    \item $\Sigma \cup \varepsilon$ сигнализирует о том, что вход можно и не читать
  \end{itemize}

  \begin{myauto}
    \node[state]   (q_1)                 {$\state{p}$};
    \node[state]   (q_2)  [right=of q_1] {$\state{q}$};

    \path[->] (q_1) edge node [above] {$\symbl{a}; \stack{A / \alpha}$} (q_2);
  \end{myauto}
\end{frame}

\begin{frame}[fragile]
  \frametitle{Семантика магазинного автомата}
\begin{itemize}
  \item Мгновенное описание МА: $\conf{p}{\omega}{\beta} \in Q \times \Sigma^* \times \Gamma^*$
  \begin{itemize}
  	\item $p$ --- текущее состояние автомата
  	\item $\omega$ --- непрочитанный фрагмент входного потока
  	\item $\beta$ --- содержимое стека (верхушка записана первой)
  \end{itemize}
  \item Отношение $\vdash$ на мгновенных описаниях (шаг)
  \begin{itemize}
  	\item Для каждого $(\state{q}, \stack{a}) \in \trans{p}{a}{A}$, верно $\conf{p}{a x}{A \gamma} \vdash \conf{q}{x}{\alpha \gamma}$ для произвольных $x \in \Sigma^*, \gamma \in \Gamma^*$
  \end{itemize}
  % \item В недетерминированных магазинных автоматах может существовать несколько шагов
  % \begin{itemize}
  %   \item Можно выбрать любой
  %   \item Если какой-нибудь выбор приведет к успеху, значит, строка распознается
  % \end{itemize}
  \item Шаг не определен, если стек пуст
\end{itemize}

\end{frame}


\begin{frame}[fragile]
  \frametitle{Семантика магазинного автомата: вычисление}
  \begin{itemize}
    \item Вычисление --- последовательность шагов
    \item Начальное мгновенное описание $\conf{q_0}{\omega}{\bot}$
    \item Выбирается любой из подходящих шагов
    \item Если какой-нибудь выбор приведет к успеху, значит, строка распознается
    \vspace{0.5cm}
    \item Два варианта окончания работы
    \begin{itemize}
      \item По достижении конечного состояния
      \begin{itemize}
        \item $L(M) = \{ \omega \in \Sigma^* \mid \conf{q_0}{\omega}{\bot} \vdash^* \conf{f}{\varepsilon}{\gamma}, f \in F, \gamma \in \Gamma^* \}$
      \end{itemize}
      \item По опустошении стека
      \begin{itemize}
      	\item $N(M) = \{ \omega \in \Sigma^* \mid \conf{q_0}{\omega}{\bot}) \vdash^* \conf{q}{\varepsilon}{\varepsilon}, q \in Q\}$
      \end{itemize}
      \item Эти варианты эквивалентны: по автомату, завершающемуся по первой схеме, можно посмотроить автомат, завершающийся по второй схеме, и наоборот
    \end{itemize}
    \item $\vdash^*$ --- транзитивно рефлексивное замыкание отношения $\vdash$
  \end{itemize}
\end{frame}

\begin{frame}[fragile]
  \frametitle{Пример: язык $\{ 0^n 1^n \mid n \geq 0 \}$ (конечное состояние)}
  \begin{myauto}
    \node[state,initial]   (p)              {$\state{p}$};
    \node[state]           (q) [right=of p] {$\state{q}$};
    \node[state,accepting] (r) [right=of q] {$\state{r}$};

    \path[->] (p) edge [loop above]    node [left] {$\symbl{0}; \stack{\bot / A\bot}$} ()
                  edge [loop below]    node [left] {$\symbl{0}; \stack{A / AA}$} ()
                  edge node [below] {$\symbl{\varepsilon}$} (q)
              (q) edge [loop above]    node [above] {$\symbl{1}; \stack{A / \varepsilon}$} ()
                  edge node [below] {$\symbl{\varepsilon}; \stack{\bot / \bot}$} (r)
              ;
  \end{myauto}

Вычисление на строке $0011$:
\begin{itemize}
  \item $\conf{p}{0011}{\bot} \vdash \conf{q}{0011}{\bot} \vdash \conf{r}{0011}{\bot}$ --- провал
  \item $\conf{p}{0011}{\bot} \vdash \conf{p}{011}{A\bot} \vdash \conf{q}{011}{A\bot}$ --- провал
  \item $\conf{p}{0011}{\bot} \vdash \conf{p}{011}{A\bot} \vdash \conf{p}{11}{AA\bot} \vdash \conf{q}{11}{AA\bot} \vdash \conf{q}{1}{A\bot} \vdash \conf{q}{\varepsilon}{\bot} \vdash \conf{r}{\varepsilon}{\bot}$ --- успех (по принимающему состоянию)
\end{itemize}

\end{frame}

\begin{frame}[fragile]
  \frametitle{Пример: язык $\{ 0^n 1^n \mid n \geq 0 \}$ (конечное состояние)}
  \begin{myauto}
    \node[state,initial]   (p)              {$\state{p}$};
    \node[state]           (q) [right=of p] {$\state{q}$};
    \node[state,accepting] (r) [right=of q] {$\state{r}$};

    \path[->] (p) edge [loop above]    node [left] {$\symbl{0}; \stack{\bot / A\bot}$} ()
                  edge [loop below]    node [left] {$\symbl{0}; \stack{A / AA}$} ()
                  edge node [below] {$\symbl{\varepsilon}$} (q)
              (q) edge [loop above]    node [above] {$\symbl{1}; \stack{A / \varepsilon}$} ()
                  edge node [below] {$\symbl{\varepsilon}; \stack{\bot / \bot}$} (r)
              ;
  \end{myauto}

Вычисление на строке $00111$:
\begin{itemize}
  \item $\conf{p}{00111}{\bot} \vdash \conf{q}{00111}{\bot} \vdash \conf{r}{00111}{\bot}$ --- провал
  \item $\conf{p}{00111}{\bot} \vdash \conf{p}{0111}{A\bot} \vdash \conf{q}{0111}{A\bot}$ --- провал
  \item $\conf{p}{00111}{\bot} \vdash \conf{p}{0111}{A\bot} \vdash \conf{p}{111}{AA\bot} \vdash \conf{q}{111}{AA\bot} \vdash \conf{q}{11}{A\bot} \vdash \conf{q}{1}{\bot} \vdash \conf{r}{1}{\bot}$ --- провал
\end{itemize}

\end{frame}

\begin{frame}[fragile]
  \frametitle{Пример: язык $\{ 0^n 1^m \mid 0 \leq m \leq n \}$ (опустошение стека)}
  \begin{myauto}
    \node[state,initial]   (p)              {$\state{p}$};
    \node[state]           (q) [right=of p] {$\state{q}$};

    \path[->] (p) edge [loop above]    node [left] {$\symbl{0}; \stack{\bot / A\bot}$} ()
                  edge [loop below]    node [left] {$\symbl{0}; \stack{A / AA}$} ()
                  edge node [below] {$\symbl{1}; \stack{A / \varepsilon}$} (q)
              (q) edge [loop above]    node [above] {$\symbl{1}; \stack{A / \varepsilon}$} ()
                  edge [loop below]    node [below] {$\symbl{\varepsilon}; \stack{\bot / \varepsilon}$} ()
                  edge [loop right]    node [right] {$\symbl{\varepsilon}; \stack{A / \varepsilon}$} ()
              ;
  \end{myauto}

Вычисление на строке $00111$:
\begin{itemize}
  \item $\conf{p}{00111}{\bot} \vdash \conf{q}{00111}{\bot}$ --- провал
  \item $\conf{p}{00111}{\bot} \vdash \conf{p}{0111}{A\bot} \vdash \conf{p}{111}{AA\bot} \vdash \conf{q}{11}{A\bot} \vdash \conf{q}{1}{\bot} $ --- провал
\end{itemize}

Вычисление на строке $001$:
\begin{itemize}
  \item $\conf{p}{001}{\bot} \vdash \conf{p}{01}{A\bot} \vdash \conf{p}{1}{AA\bot} \vdash \conf{q}{\varepsilon}{A\bot} \vdash \conf{q}{\varepsilon}{\bot} \vdash \conf{q}{\varepsilon}{\varepsilon}  $ --- успех
\end{itemize}
\end{frame}

\begin{frame}[fragile]
  \frametitle{Пример: язык $\{ \omega c \omega^R \mid \omega \in \{0, 1\}^*  \}$ (опустошение стека)}
  \[\stack{X} \in \{\stack{G}, \stack{B}, \stack{\bot}\}\]
  \begin{myauto}
    \node[state,initial]   (p)              {$\state{p}$};
    % \node[draw=none]       (n) [left=of p, xshift=-1cm] {$X \in \{G, B, \bot\}$};
    \node[state]           (q) [right=of p] {$\state{q}$};

    \path[->] (p) edge [loop above]    node [left] {$\symbl{1}; \stack{X / G X}$} ()
                  edge [loop below]    node [left] {$\symbl{0}; \stack{X / B X}$} ()
                  edge node [below] {$\symbl{c}; \stack{X / X}$} (q)
              (q) edge [loop above]    node [above] {$\symbl{0}; \stack{B / \varepsilon}$} ()
                  edge [loop below]    node [below] {$\symbl{1}; \stack{G / \varepsilon}$} ()
                  edge [loop right]    node [right] {$\symbl{\varepsilon}; \stack{\bot / \varepsilon}$} ()
              ;
  \end{myauto}

  Вычисление на строке $110c011$:
  \begin{itemize}
    \item $\conf{p}{110c011}{\bot} \vdash \conf{p}{10c011}{G\bot} \vdash \conf{p}{0c011}{GG\bot} \vdash \conf{p}{c011}{BGG\bot} \vdash \conf{q}{011}{BGG\bot} \vdash \conf{q}{11}{GG\bot} \vdash \conf{q}{1}{G\bot} \vdash \conf{q}{\varepsilon}{\bot} \vdash  \conf{q}{\varepsilon}{\varepsilon}$ --- успех
  \end{itemize}

\end{frame}



\begin{frame}[fragile]
  \frametitle{Формальное определение ДМА}
  \emph{Детерминированный} магазинный автомат это $(Q, \Sigma, \Gamma, \delta, q_0, \bot, F)$
  \begin{itemize}
    \item $Q$ --- конечное множество состояний
    \item $\Sigma$ --- конечное множество символов, входной алфавит
    \item $\Gamma$ --- конечное множество символов, стековый алфавит
    \item $\delta: Q \times (\Sigma \cup \varepsilon) \times \Gamma \to Q \times \Gamma^*$ --- функция переходов
    \begin{itemize}
      \item $\forall q \in Q, \bot \in \Gamma:$ если есть эпсилон-переход $\delta(q, \varepsilon, \bot)$ то нет переходов по терминалам $\delta(q, a, \bot)$
    \end{itemize}
    \item $q_0 \in Q$ --- стартовое состояние
    \item $\bot \in \Gamma$ --- начальный элемент стека
    \item $F \subseteq Q$ --- множество принимающих (конечных) состояний
  \end{itemize}
\end{frame}

\begin{frame}[fragile]
  \frametitle{Детерминированные магазинные автоматы vs недетерминированные}
\begin{itemize}
	\item В общем случае одной входной строке может соответствовать несколько вычислений
	\begin{itemize}
		\item Некоторые из них могут завершаться в принимающих состояниях
	\end{itemize}
	\item Если существует хотя бы одно вычисление, завершающееся в принимающем состоянии, строка принадлежит языку
	\item Если для каждой строки существует ровно одно вычисление в магазинном автомате, то он является \emph{детерминированным}
	\begin{itemize}
		\item Соответствующий язык является \emph{детерминированным КС языком}
	\end{itemize}
	\item Детерминированный магазинный автомат является частным случаем недетерминированного, поэтому детерминированные КС языки --- строгое подмножество контекстно-свободных
\end{itemize}
\end{frame}

\begin{frame}[fragile]
  \frametitle{Неэквивалентность двух видов приема для детерминированных магазинных автоматов}

  \emph{Беспрефиксный язык} --- язык, в котором никакое слово не является префиксом другого

  \begin{itemize}
    \item Прием языка детерминированным магазинным автоматом по пустому стеку и по допускающему состоянию эквивалентно только для беспрефиксных языков
    \item Рассмотрим слово $\omega = \alpha \beta: \alpha, \beta \in \Sigma^*, \omega, \alpha \in L \text{, где } L \subseteq \Sigma^*$
    \item При попытке распознать слово $\omega$ ДМП завершит свою работу, как только прочитает $\alpha$
    \item $\omega$ никогда не будет принята
    \item Можно построить ДМП, принимающий по допускающему состоянию, который допускает префиксный язык
  \end{itemize}
\end{frame}

\begin{frame}[fragile]
  \frametitle{Построение магазинного автомата по КС-грамматике}
  \begin{itemize}
    \item Интуиция:
    \begin{itemize}
      \item Для каждого нетерминала: заменяем его на стеке на правую часть правила
      \item Для каждого терминала: считываем со входа этот терминал и снимаем со стека
    \end{itemize}
    \item Построение:
    \begin{itemize}
      \item Для каждого правила $A \rightarrow \alpha$ добавляем $(\state{q}, \stack{\alpha})$ в $\trans{q}{\varepsilon}{A}$
      \item Для каждого терминала $a$ добавляем $(\state{q},\stack{\varepsilon})$ в $\trans{q}{a}{a}$
    \end{itemize}
    \item Относительно бесполезный автомат: как найти правильное вычисление?
  \end{itemize}
  \begin{myauto}
    \node[state,initial] (q) {$\state{q}$};
    \path[->] (q) edge [loop above] node [above] {$\symbl{\varepsilon}; \stack{A / \alpha}$} ()
                  edge [loop right] node [right] {$\symbl{a}; \stack{a / \varepsilon}$} ();
  \end{myauto}
\end{frame}

\begin{frame}[fragile]
  \frametitle{Магазинный автомат по грамматике $S \to (S)S \mid \varepsilon$ }
  \begin{myauto}
    \node[state] (q) {$\state{q}$};
    \path[->] (q)
    edge [loop above] node [above] {$\symbl{\varepsilon}; \stack{S / (S)S}$} ()
    edge [loop left]  node [left ] {$\symbl{\varepsilon}; \stack{S / \varepsilon}$} ()
    edge [loop right] node [right] {$\symbl{(}; \stack{( / \varepsilon}$} ()
    edge [loop below] node [below] {$\symbl{)}; \stack{) / \varepsilon}$} ()
    ;
  \end{myauto}

  Вычисление на строке $(())()$:
  \begin{itemize}
    \item $\conf{q}{(())()}{S} \vdash \conf{q}{(())()}{(S)S} \vdash \conf{q}{())()}{S)S} \vdash \conf{q}{())()}{(S)S)S} \vdash \conf{q}{))()}{S)S)S} \vdash \conf{q}{))()}{)S)S} \vdash \conf{q}{)()}{S)S} \vdash \conf{q}{)()}{)S} \vdash \conf{q}{()}{S} \vdash \conf{q}{()}{(S)S} \vdash \conf{q}{)}{S)S} \vdash \conf{q}{)}{)S} \vdash \conf{q}{\varepsilon}{S} \vdash \conf{q}{\varepsilon}{\varepsilon} \vdash $--- успех
  \end{itemize}
\end{frame}

\begin{frame}[fragile]
  \frametitle{Построение магазинного автомата по КС-грамматике в нормальной форме Грейбах}
  \begin{itemize}
    \item Нормальная форма Грейбах: все правила имеют вид $A \to x \alpha$
    \begin{itemize}
      \item $A \in V_N$
      \item $x \in V_T$
      \item $\alpha \in V_N^*$
    \end{itemize}
    \item Построение магазинного автомата:
    \begin{itemize}
      \item Для каждого правила $A \rightarrow x \alpha$ добавляем $(\state{q}, \stack{\alpha})$ в $\trans{q}{x}{A}$
    \end{itemize}
    \item Автоматы получаются проще
    \item Не сильно более полезный автомат: нужны алгоритмы синтаксического анализа
  \end{itemize}
  \begin{myauto}
    \node[state,initial] (q) {$\state{q}$};
    \path[->] (q) edge [loop above] node [above] {$\symbl{x}; \stack{A / \alpha}$} ();
  \end{myauto}
\end{frame}

\begin{frame}[fragile]
  \frametitle{Магазинный автомат по грамматике в НФГ \\ $S \to (SRS \mid (R$ \\$R \to \ )$ }
  \begin{myauto}
    \node[state] (q) {$\state{q}$};
    \path[->] (q)
    edge [loop above] node [above] {$\symbl{(}; \stack{S / SRS}$} ()
    edge [loop left]  node [left ] {$\symbl{(}; \stack{S / R}$} ()
    edge [loop right] node [right] {$\symbl{)}; \stack{R / \varepsilon}$} ()
    ;
  \end{myauto}

  Вычисление на строке $(())()$:
  \begin{itemize}
    \item $\conf{q}{(())()}{S} \vdash \conf{q}{())()}{SRS} \vdash \conf{q}{))()}{RRS} \vdash \conf{q}{)()}{RS} \vdash \conf{q}{()}{S} \vdash \conf{q}{)}{R} \vdash \conf{q}{\varepsilon}{\varepsilon} \vdash  $--- успех
  \end{itemize}
\end{frame}

\begin{frame}[fragile]
  \frametitle{Построение КС грамматики по магазинному автомату }
  \begin{itemize}
    \item По магазинному автомату $(Q, \Sigma, \Gamma, \delta, q_0, \bot, F)$ строим грамматику $(V_N, V_T, P, S)$
    \item $V_N = \{S\} \cup \{[\state{q}\stack{A}\state{p}] \mid p, q \in Q, A \in \Gamma\}$
    \item $V_T = \Sigma$
    \item $P = \{S \to [\state{q_0} \stack{\bot} \state{q}] \mid \state{q} \in Q\} \cup $ \\ $\ \ \ \ \ \ \{[\state{q}\stack{A}\state{q_{m+1}}] \to \symbl{a} [\state{q_1} \stack{B_1} \state{q_2}] [\state{q_2} \stack{B_2} \state{q_3}] \dots [\state{q_m} \stack{B_m} \state{q_{m+1}}] \mid $ \\ $\ \ \ \ \ \ \ \ \ \ \ \ \ \ \state{q_i} \in Q; \symbl{a} \in \Sigma \cup \{\varepsilon\}; \stack{A}, \stack{B_j} \in \Gamma; (\state{q_1}, \stack{B_1} \stack{B_2} \dots \stack{B_m}) \in \delta(\state{q}, \symbl{a}, \stack{A})\} \cup $ \\ $\ \ \ \ \ \ \{[\state{q} \stack{A} \state{p}] \to a \mid (\state{p}, \stack{\varepsilon}) \in \delta(\state{q}, \symbl{a}, \stack{A})\}$
  \end{itemize}
\end{frame}

\begin{frame}[fragile]
  \frametitle{Пример построения КС грамматики по магазинному автомату}
  \begin{myauto}
    \node[state] (q) {$\state{q}$};
    \path[->] (q)
    edge [loop above] node [above] {$\symbl{(}; \stack{S / SRS}$} ()
    edge [loop left]  node [left ] {$\symbl{(}; \stack{S / R}$} ()
    edge [loop right] node [right] {$\symbl{)}; \stack{R / \varepsilon}$} ()
    ;
  \end{myauto}

  $$
  \begin{array}{lcl}
    [qSq] & \to  & ( \ [qRq] \\
          & \mid & ( \ [qSq] \ [qRq] \ [qSq] \\ \relax
    [qRq] & \to  & )
  \end{array}
  $$

\end{frame}


\begin{frame}[fragile]
  \frametitle{Лемма о накачке для КС языков}
  \begin{rutheorem}
    Если язык $L$ является контекстно-свободным, то

    $\exists p \geq 1: \forall s \in L : |s| \geq p $ можно разбить на подстроки $s = uvwxy: |vwx| \leq p, |vx| \geq 1$ и

     $\forall n \geq 0. \, u v^n w x^n y \in L$
  \end{rutheorem}
  \vspace{-0.5cm}
  \begin{columns}[T]
    \begin{column}{0.38\textwidth}
      \begin{center}
        \begin{tikzpicture}
        \node[draw=none, fill=none] at (0, 4) {S};
        \node[draw=none, fill=none] at (-0.1, 2.8) {A};
        \node[draw=none, fill=none] at (0.1, 1.6) {A};
        \node[draw=none, fill=none] at (-0, 0.8) {A};

        \draw (-2,0) -- (0,3.8);
        \draw (0,3.8) -- (2,0);
        \draw (-1.5,0) -- (-0.1,2.6);
        \draw (-0.1,2.6) -- (1.5,0);
        \draw (-1,0) -- (0.1,1.4);
        \draw (0.1,1.4) -- (1,0);
        \draw (-0.5,0) -- (0,0.6);
        \draw (0,0.6) -- (0.5,0);

        \fill[Orchid] (-2,-0.5) rectangle ++(0.5,0.5);
        \fill[Dandelion] (-1.5,-0.5) rectangle ++(1,0.5);
        \fill[LimeGreen] (-0.5,-0.5) rectangle ++(1,0.5);
        \fill[Salmon] (0.5,-0.5) rectangle ++(1,0.5);
        \fill[Turquoise] (1.5,-0.5) rectangle ++(0.5,0.5);

        \draw[draw=black] (-2,-0.5) rectangle ++(4,0.5);
        \draw[draw=black] (-1.5,-0.5) rectangle ++(3,0.5);
        \draw[draw=black] (-1,-0.5) rectangle ++(2,0.5);
        \draw[draw=black] (-0.5,-0.5) rectangle ++(1,0.5);

        \draw [thick,densely dotted, rounded corners=1mm](0,3.8) --  (0.1, 3.4) -- (-0.15, 3.2) -- (-0.1, 2.9);

        \draw [thick,densely dotted, rounded corners=1mm](-0.1, 2.6) --  (-0.1, 2.3) -- (0.15, 1.9) -- (0.1, 1.75);

        \draw [thick,densely dotted, rounded corners=1mm](0.1,1.4) -- (-0.05, 1.1) -- (0, 1);

        \node[draw=none, fill=none] at (-1.75,-0.25) {$u$};
        \node[draw=none, fill=none] at (-1.25,-0.25) {$v$};
        \node[draw=none, fill=none] at (-0.75,-0.25) {$v$};
        \node[draw=none, fill=none] at (0,-0.25) {$w$};
        \node[draw=none, fill=none] at (0.75,-0.25) {$x$};
        \node[draw=none, fill=none] at (1.25,-0.25) {$x$};
        \node[draw=none, fill=none] at (1.75,-0.25) {$y$};
        \end{tikzpicture}
      \end{center}
    \end{column}
    \begin{column}{0.05\textwidth}
      \vspace{2cm}
      \[\color{teal} \Mapsfrom\]
    \end{column}
    \begin{column}{0.3\textwidth}
      \begin{center}
        \begin{tikzpicture}
        \node[draw=none, fill=none] at (0, 4) {S};
        \node[draw=none, fill=none] at (-0.1, 2.3) {A};
        \node[draw=none, fill=none] at (0.1, 1) {A};

        \draw (-1.5,0) -- (0,3.8);
        \draw (0,3.8) -- (1.5,0);
        \draw (-1,0) -- (-0.1,2.1);
        \draw (-0.1,2.1) -- (1,0);
        \draw (-0.5,0) -- (0.1,0.8);
        \draw (0.1,0.8) -- (0.5,0);

        \fill[Orchid] (-1.5,-0.5) rectangle ++(0.5,0.5);
        \fill[Dandelion] (-1,-0.5) rectangle ++(0.5,0.5);
        \fill[LimeGreen] (-0.5,-0.5) rectangle ++(1,0.5);
        \fill[Salmon] (0.5,-0.5) rectangle ++(0.5,0.5);
        \fill[Turquoise] (1,-0.5) rectangle ++(0.5,0.5);

        \draw[draw=black] (-1.5,-0.5) rectangle ++(3,0.5);
        \draw[draw=black] (-1,-0.5) rectangle ++(2,0.5);
        \draw[draw=black] (-0.5,-0.5) rectangle ++(1,0.5);

        \draw [thick,densely dotted, rounded corners=1mm](0,3.8) --  (0.1, 2.9) -- (-0.15, 2.7) -- (-0.1, 2.4);

        \draw [thick,densely dotted, rounded corners=1mm](-0.1, 2.1) --  (-0.1, 1.7) -- (0.15, 1.4) -- (0.1, 1.1);


        \node[draw=none, fill=none] at (-1.25,-0.25) {$u$};
        \node[draw=none, fill=none] at (-0.75,-0.25) {$v$};
        \node[draw=none, fill=none] at (0,-0.25) {$w$};
        \node[draw=none, fill=none] at (0.75,-0.25) {$x$};
        \node[draw=none, fill=none] at (1.25,-0.25) {$y$};
        \end{tikzpicture}
      \end{center}
    \end{column}
    \begin{column}{0.05\textwidth}
      \vspace{2cm}
      \[\color{teal} \Mapsto\]
    \end{column}
    \begin{column}{0.23\textwidth}
      \begin{center}
        \begin{tikzpicture}
        \node[draw=none, fill=none] at (0, 4) {S};
        \node[draw=none, fill=none] at (0.1, 1) {A};

        \draw (-1,0) -- (0,3.8);
        \draw (0,3.8) -- (1,0);

        \draw (-0.5,0) -- (0.1,0.8);
        \draw (0.1,0.8) -- (0.5,0);

        \fill[Orchid] (-1,-0.5) rectangle ++(0.5,0.5);
        \fill[LimeGreen] (-0.5,-0.5) rectangle ++(1,0.5);
        \fill[Turquoise] (0.5,-0.5) rectangle ++(0.5,0.5);

        \draw[draw=black] (-1,-0.5) rectangle ++(2,0.5);
        \draw[draw=black] (-0.5,-0.5) rectangle ++(1,0.5);

        \draw [thick,densely dotted, rounded corners=1mm] (0,3.8) --  (-0.1, 2.5) -- (0.15, 1.4) -- (0.1, 1.1);


        \node[draw=none, fill=none] at (-0.75,-0.25) {$u$};
        \node[draw=none, fill=none] at (0,-0.25) {$w$};
        \node[draw=none, fill=none] at (0.75,-0.25) {$y$};
        \end{tikzpicture}
      \end{center}
    \end{column}
  \end{columns}
\end{frame}

\begin{frame}[fragile]
  \frametitle{Лемма о накачке для КС языков: пример}
  Язык $L = \{ a^n b^n c^n \}$

  Предполагаем, что он КС, тогда по Лемме существует $p$...

  Рассмотрим слово $a^p b^p c^p = uvwxy, |vwx| \leq p, |vx| \geq 1 $

  \begin{itemize}
    \item $vwx = a^j, j \leq p$
    \item $vwx = a^j b^k, j + k \leq p$
    \item $vwx = b^j, j \leq p$
    \item $vwx = b^j c^k, j + k \leq p$
    \item $vwx = c^j, j \leq p$
  \end{itemize}

  Строка $u v^i w x^i y $ не содержит одинаковое количество букв для всех $i$. Например, рассмотреть $i = 2$. Получили противоречие --- успех
\end{frame}

\end{document}
