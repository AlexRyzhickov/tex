\documentclass[12pt]{article}
\usepackage[left=1cm,right=1.5cm,top=2cm,bottom=2cm,bindingoffset=0cm]{geometry}
\usepackage{hyperref}
\usepackage{fontspec}
\usepackage{polyglossia}
\usepackage{amssymb}
\setdefaultlanguage{russian}
\pagestyle{empty}
\setmainfont[Mapping=tex-text]{CMU Serif}

\begin{document}
\begin{center}
{\LARGE Формальные языки}

\bigskip

{\Large Павел Громов}
\end{center} 

\bigskip

\begin{enumerate}
  \item Построить грамматику для языка $\{ a^n b c^n d \mid n \geq 1 \} \cup \{ a^n f c^n g \mid n \geq 1 \}$.
  \item Проверить, относится ли построенная грамматика к классу SLR(1). Если является, привести таблицу подходящего анализатора и продемонстрировать успешный и неуспешный вывод на 2 цепочках длины не меньше 7. Если нет, обосновать.
  \item Для языка арифметических выражений над целыми числами и переменными (обычными идентификаторами) с операциями сложения (\verb!+!) и умножения (\verb!*!) и скобками реализовать оптимизатор. Оптимизатор вычисляет все фрагменты выражений, которые можно вычислить статически. 
  Например \verb!2*5+x*0+y! упрощается до \verb!10+y!; \verb!2*(5+x)*0+y! упрощается до \verb!y!. 
  
\end{enumerate}
\end{document}