\documentclass[12pt]{article}
\usepackage[left=1cm,right=1.5cm,top=2cm,bottom=2cm,bindingoffset=0cm]{geometry}
\usepackage{hyperref}
\usepackage{fontspec}
\usepackage{polyglossia}
\usepackage{amssymb}
\setdefaultlanguage{russian}
\pagestyle{empty}
\setmainfont[Mapping=tex-text]{CMU Serif}

\begin{document}
\begin{center}
{\LARGE Формальные языки}

\bigskip

{\Large Елизавета Сидорова}
\end{center} 

\bigskip

\begin{enumerate}
  \item Упростить регулярное выражение $(a|b)^∗ a b (a|b)^∗ \mid (a|b)^∗ a \mid b^∗$
  \item С использованием любого генератора парсеров реализовать парсер для языка P. Намеренно плохое описание конкретного синтаксиса ниже.
  \item Можно ли распознать язык $\{ a^n b c^n \mid n \geq 1 \} \cup \{ a^n d c^{2n} \mid n \geq 1 \}$ при помощи алгоритма SLR(1)? Если можно --- привести автомат и таблицу анализатора и продемонстрировать успешный и неуспешный вывод на строках длины не меньше 7. Если нельзя --- обосновать.
\end{enumerate}

\begin{center}
    \Large{Конкретный синтаксис языка \verb!P!}
\end{center}

Программа на языке \verb!P! состоит из возможно пустого множества определений отношений, за которыми следует цель, отделенная символом \verb!?-!.

Отношение состоит из нескольких строк, его определяющих. 
Каждая строка состоит из головы и тела, разделенных символом \verb!:-!. 

Голова состоит из идентификатора (название отношения) и списка аргументов в круглых скобках (\verb!(!, \verb!)!), разделенных запятыми (\verb!,!). 

Аргументом может быть либо переменная, либо атом. 

Атом --- идентификатор и список аргументов в круглых скобках через запятую. 

Переменная всегда начинается с заглавной латинской буквы. Идентификатор --- со строчной латинской буквы.
Имена переменных и идентификаторов содержат только латинские буквы либо цифры \verb!0! -- \verb!9!.

Тело состоит из возможно пустой последовательности атомов, разделенных запятой, в конце тела стоит точка (\verb!.!).

Цель --- возможно пустая последовательность атомов, разделенных запятой, в конце стоит точка. 

Если список аргументов атома пуст, скобки опускаются. 

Если тело пусто, символ \verb!:-! опускается. 

Пробелы и переносы строк между лексемами не являются значащими. Их удаление не должно повлиять на результат синтаксического анализа корректной программы

\begin{center}
    \Large{Пример программы на языке \verb!P!}
\end{center}

\begin{verbatim}
eval(St, var(X), U) :- elem(X, St, U).
eval(St, conj(X,Y), U)  :- eval(St, X, V), eval(St, Y, W), and(V, W, U).
eval(St, disj(X,Y), U)  :- eval(St, X, V), eval(St, Y, W), or(V, W, U).
eval(St, not(X), U) :- eval(St, X, V), neg(U, V).

elem(zero, cons(H,T), H).
elem(succ(N), cons(H,T), V) :- elem(N, T, V).

nand(false, false, true).
nand(false, true,  true).
nand(true,  false, true).
nand(true,  true,  false).

neg(X, R) :- nand(X, X, R).

or(X, Y, R) :- nand(X, X, Xx), nand(Y, Y, Yy), nand(Xx, Yy, R).

and(X, Y, R) :- nand(X, Y, Xy), nand(Xy, Xy, R).

?- eval(St, conj(disj(X,Y),not(var(Z))), true).
\end{verbatim}
\end{document}
  
  

\end{document}