\documentclass[12pt]{article}
\usepackage[left=2cm,right=2cm,top=2cm,bottom=2cm,bindingoffset=0cm]{geometry}
\usepackage{fontspec}
\usepackage{polyglossia}
\usepackage{amssymb}
\setdefaultlanguage{russian}
\setmainfont[Mapping=tex-text]{CMU Serif}

\begin{document}
\begin{center} {\LARGE Формальные языки} \end{center}

\begin{center} {\Large домашнее задание до 23:59 23.10} \end{center}
\bigskip

\begin{enumerate}
  \item 
  {
    Реализовать алгоритм проверки пустоты порождаемого грамматикой языка. Программа на вход принимает файл с грамматикой, и печатает ``empty'', если она порождает пустой язык, и ``nope'' иначе. Формат описания грамматики ниже. 
  }
  \item {
    Реализовать преобразование грамматики в нормальную форму Хомского. Программа принимает на вход файл с грамматикой, преобразовывает грамматику в нормальную форму Хомского и печатает результат в файл. 
 }
  \item 
  {
    Реализовать алгоритм синтаксического анализа CYK. Программа принимает на вход файл с грамматикой (не обязательно в нормальной форме Хомского) и файл с входной строкой. Результатом анализа должна быть заполненная таблица (в человекочитаемом формате, можно csv), и дерево вывода (в человекочитаемом формате, можно dot), если строка выводима в данной грамматике. 

     
  }
\end{enumerate}

\begin{center} Требования к программе \end{center}

\begin{itemize}
    \item Сделать консольное  приложение, принимающее на вход путь к файлу, содержащему описание конечного автомата, производящее минимизацию и печатающее его результат в файл.
    \item Если формат входных данных не соблюден, приложение должно об этом сообщать. 
    \item Составить набор тестов, демонстрирующий правильность работы реализации (качество тестового покрытия важно!).
    \item Код должен быть размещен на гитхабе, собираться одним скриптом, содержать инструкцию по сборке и запуску собранного приложения, собираться на чистой Ubuntu 18.04 или Windows 10. Все зависимости, в случае их отсутствия в системе, должны доставляться скриптом.
    \begin{itemize}
        \item Инструкция по запуску должна содержать информацию о том, где находится бинарник, как именно его полагается запускать, какой формат описания автомата, куда пишется результат.
        \item Можно писать на любом языке программирования. 
    \end{itemize} 
\end{itemize}

\begin{center} Формат грамматики \end{center}

\begin{itemize}
    \item Стартовый нетерминал записан в первой строке.
    \item Во второй строке через пробел записаны нетерминалы грамматики.
    \item В третьей строке через пробел записаны терминалы грамматики.
    \item В остальных строках записаны правила.
    \item { Правила для нетерминала перечислены через вертикальную черту, для каждого нетерминала есть только одна строка. Формат:

 \verb!<nonterminal> -> <production_1> | ... | <production_k>! }
    \item Каждое правило на новой строке.
    \item Нетерминалы и терминалы в правилах разделяются пробелами.
    \item Каждое правило состоит из терминалов и нетерминалов. Для $\varepsilon$ используйте ключевое слово \verb!_eps!
\end{itemize}

\begin{center} Пример \end{center}

\begin{verbatim}
Expr
+ * 0 1 ( )
Expr Term Factor
Expr -> Expr + Term | Term 
Term -> Term * Factor | Factor 
Factor -> 0 | 1 | ( Expr )
\end{verbatim}




\end{document}