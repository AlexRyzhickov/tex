\documentclass[12pt]{article}
\usepackage[left=1cm,right=1cm,top=1cm,bottom=1cm,bindingoffset=0cm]{geometry}
\usepackage{fontspec}
\usepackage{polyglossia}
\setdefaultlanguage{russian}
\setmainfont[Mapping=tex-text]{CMU Serif}

\pagenumbering{gobble}

\begin{document}
\begin{center} 
{\LARGE Формальные языки}

{\Large домашнее задание до 23:59 17.12}
\end{center}

\begin{enumerate}
{
   \item { Привести полную L-атрибутную грамматику для следующего языка }
   \begin{itemize}
       \item Сначала идет список определенных переменных с указанием их типов. 
       \item Дальше идет список присваиваний переменным значений.
       \item Последней строчкой указано выражение, значение которого должна считать L-атрибутная грамматика.
       \item В конце каждой строки стоит символ \textbf{;}.
       \item Переменные бывают двух типов: \verb!bool! и \verb!int!.
       \item Выражения бывают двух видов: арифметические и логические. 
       \item Предложите свою семантику, которая предусматривает конкретную стратегию поведения в различных случаях: 
       \begin{itemize}
           \item Тип переменной был не указан, а присваивание в нее происходит.
           \item Присваиваем значение, тип которого отличается от типа переменной.
           \item Во время вычисления выражения встречаем переменную, которая не определена или тип которой не известен.
           \item В одном выражении присутствуют как логические, так и арифметические операторы
       \end{itemize}
   \end{itemize} 
}
\end{enumerate}

Пример программы: 

\begin{verbatim}
int x, y, z;
bool u, v;
x = 13;
y = 42; 
v = true;
z = x + 3 * y - v;
¬ v & u;
\end{verbatim}


\end{document}