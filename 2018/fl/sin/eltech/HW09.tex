\documentclass[12pt]{article}
\usepackage[left=1cm,right=1cm,top=1cm,bottom=1cm,bindingoffset=0cm]{geometry}
\usepackage{fontspec}
\usepackage{polyglossia}
\setdefaultlanguage{russian}
\setmainfont[Mapping=tex-text]{CMU Serif}

\pagenumbering{gobble}

\begin{document}
\begin{center} 
{\LARGE Формальные языки}

{\Large домашнее задание до 23:59 03.12}
\end{center}

\begin{enumerate}
{
   \item { Формализовать правила для поддержки левой рекурсии в Parser Expression Grammar. }
   
   \item { Написать программу, осуществляющую матчинг Parser Expression Grammar. Проверить, что она работает на грамматиках с пары, корректных и некорректных цепочках. }
}
\end{enumerate}


\begin{center} Требования к программе \end{center}

\begin{itemize}
    \item Сделать консольное  приложение, принимающее на вход путь к файлу, содержащему описание PEG, и путь к файлу, содержащему входную строку.
    \item Если формат входных данных не соблюден, приложение должно об этом сообщать. 
    \item Составить набор тестов, демонстрирующий правильность работы реализации (качество тестового покрытия важно!).
    \item Код должен быть размещен на гитхабе, собираться одним скриптом, содержать инструкцию по сборке и запуску собранного приложения, собираться на чистой Ubuntu 18.04 или Windows 10. Все зависимости, в случае их отсутствия в системе, должны доставляться скриптом.
    \begin{itemize}
        \item Инструкция по запуску должна содержать информацию о том, где находится бинарник, как именно его полагается запускать, какой формат описания автомата, куда пишется результат.
        \item Можно писать на любом языке программирования. 
    \end{itemize} 
\end{itemize}

\begin{center} Формат грамматики \end{center}

\begin{itemize}
    \item Стартовое выражение записано в первой строке.
    \item Во второй строке через пробел записаны нетерминалы грамматики.
    \item В третьей строке через пробел записаны терминалы грамматики.
    \item В остальных строках записаны правила.
    \item { Правила для нетерминала перечислены через косую черту, для каждого нетерминала есть только одна строка. Формат:

 \verb!<nonterminal> -> <expression_1> / ... / <expression_k>! }
    \item Каждое правило на новой строке.
    \item Нетерминалы и терминалы в правилах разделяются пробелами.
    \item Каждое правило состоит из терминалов и нетерминалов. Для $\varepsilon$ используйте ключевое слово \verb!_eps!
    \item Обозначение для предикатов, как на паре.
\end{itemize}

\newpage

\begin{center} Пример 1 \end{center}

\begin{verbatim}
S
a b c
S A B
S -> &(A c) a+ B !.
A -> a A? b
B -> b B? c
\end{verbatim}

\begin{center} Пример 2 \end{center}

\begin{verbatim}
C
(* *) a b c 
Begin End C N Z
Begin -> (* 
End -> *)
C -> Begin N* End
N -> C / (!Begin !End Z)
Z -> (a / b / c)* 
\end{verbatim}


\end{document}