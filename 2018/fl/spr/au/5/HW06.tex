\documentclass[12pt]{article}
\usepackage[left=2cm,right=2cm,top=2cm,bottom=2cm,bindingoffset=0cm]{geometry}
\usepackage{fontspec}
\usepackage{polyglossia}
\setdefaultlanguage{russian}
\setmainfont[Mapping=tex-text]{CMU Serif}

\begin{document}
\centering {\LARGE Формальные языки}

{\Large домашнее задание до 23:59 20.05}
\bigskip

\begin{enumerate}
{
  \item Докажите или опровергните, что данный язык не является контекстно-свободным $\{ a^{n+m} b^m c^n d^m \, | \, n, m \geq 0 \}$ (2 балла)
  \item {Напишите L-атрибутную грамматику для языка арифметических выражений в инфиксной записи над натуральными числами с операциями $+, -$, скобками и перемеными с $let$-присваиваниями (4 балла)
  \begin{itemize}
      \item Результат --- целое число
      \item Законы арифметики должны быть соблюдены
      \item Только последнее присваивание в переменную учитывается
      \item На всякий случай примеры
      \begin{itemize}
          \item ``$1-(2+3)$''$ \rightarrow -4$
          \item ``$let \ x = 13 \ in \ x + x - 24$''$ \rightarrow 2$
          \item ``$let \ x = 1 \ in \ let \ y = 3 \ in \ let \ x = 7 \ in \ x - y$''$ \rightarrow 4$
          \item ``$let \ x = 1 \ in \ y $''$ \rightarrow Error $
      \end{itemize}
      \item Промоделировать вычисления на примерах из задания
  \end{itemize}
  }
}
\end{enumerate}

 
\end{document}