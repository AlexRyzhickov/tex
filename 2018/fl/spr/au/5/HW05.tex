\documentclass{article}

\usepackage[left=1cm,right=1cm,top=1cm,bottom=1cm,bindingoffset=0cm]{geometry}
\usepackage{listings}
\usepackage{indentfirst}
\usepackage{verbatim}
\usepackage{amsmath, amsthm, amssymb}
\usepackage{stmaryrd}
\usepackage{graphicx}
\usepackage{euscript}
\usepackage{hyperref}

\usepackage[utf8]{inputenc}
\usepackage[english,russian]{babel}
\usepackage[T2A]{fontenc}

\lstdefinelanguage{llang}{
keywords={skip, do, while, read, write, if, then, else, begin, end},
sensitive=true,
%%basicstyle=\small,
commentstyle=\scriptsize\rmfamily,
keywordstyle=\ttfamily\underbar,
identifierstyle=\ttfamily,
basewidth={0.5em,0.5em},
columns=fixed,
fontadjust=true,
literate={->}{{$\to$}}1
}

\lstset{
language=llang
}

\newcommand{\aset}[1]{\left\{{#1}\right\}}
\newcommand{\term}[1]{\mbox{\texttt{\bf{#1}}}}
\newcommand{\cd}[1]{\mbox{\texttt{#1}}}
\newcommand{\sembr}[1]{\llbracket{#1}\rrbracket}
\newcommand{\conf}[1]{\left<{#1}\right>}
\newcommand{\fancy}[1]{{\cal{#1}}}

\newcommand{\trule}[2]{\frac{#1}{#2}}
\newcommand{\crule}[3]{\frac{#1}{#2},\;{#3}}
\newcommand{\withenv}[2]{{#1}\vdash{#2}}
\newcommand{\trans}[3]{{#1}\xrightarrow{#2}{#3}}
\newcommand{\ctrans}[4]{{#1}\xrightarrow{#2}{#3},\;{#4}}
\newcommand{\llang}[1]{\mbox{\lstinline[mathescape]|#1|}}
\newcommand{\pair}[2]{\inbr{{#1}\mid{#2}}}
\newcommand{\inbr}[1]{\left<{#1}\right>}
\newcommand{\highlight}[1]{\color{red}{#1}}
\newcommand{\ruleno}[1]{\eqno[\textsc{#1}]}
\newcommand{\inmath}[1]{\mbox{$#1$}}
\newcommand{\lfp}[1]{fix_{#1}}
\newcommand{\gfp}[1]{Fix_{#1}}

\newcommand{\NN}{\mathbb N}
\newcommand{\ZZ}{\mathbb Z}

\begin{document}

%% Весь этот текст можно удалить
%% ====== от сих =====
\begin{center} {\LARGE Формальные языки} \end{center}

\begin{center} {\Large домашнее задание до 23:59 14.05} \end{center}
\bigskip

\begin{enumerate}
  \item Придумать три-четыре примера синтаксического сахара для вашего конкретного синтаксиса. Дополнить описанием спецификацию конкретного синтаксиса (2 балла).
  \item Модифицируйте синтаксический анализатор вашего языка, поддержав предложенные конструкции. Абстрактное синтаксическое дерево должно соответствовать старому абстрактному синтаксису (дублирую внизу) (2 балла).
\end{enumerate}

\centering {\Large Абстрактный синтаксис языка L }
$$
X \mbox{ --- счетно-бесконечное множество идентификаторов}
$$
$$
\otimes=\{\llang{+}, \llang{-}, \llang{*}, \llang{/}, \llang{\%}, \llang{==}, \llang{!=}, 
\llang{>}, \llang{>=}, \llang{<}, \llang{<=}, \llang{\&\&}, \llang{||}\}
$$

\begin{itemize}
\item Определения (функций): $\fancy{D}=\fancy{X}_{name} \; \fancy{X}_0 \dots \fancy{X}_k \leftarrow \fancy{S}$. $\fancy{X}_{name}$ --- имя функции;  $\fancy{X}_0 \dots \fancy{X}_k$ --- ее аргументы; $\fancy{S}$~--- тело.
\item Вызовы функций: $\fancy{C} = \fancy{X}_{name} \; \fancy{E}_0 \dots \fancy{E}_k$. \emph{Аргументами могут быть произвольные выражения.}
\item Выражения: $\fancy{E}=\fancy{C} \cup X\cup\NN\cup(\fancy{E}\otimes\fancy{E})$. \emph{Вызовы функций могут быть использованы в выражениях.} В выражениях могут использоваться круглые скобки.


\item Операторы: 

$$
\begin{array}{rll}
  \fancy{S}=
            &\fancy{X}\;\llang{:=}\;\fancy{E}&\cup\\
            &\fancy{C}&\cup\\
            &\llang{write}\;\fancy{E}&\cup\\
            &\llang{read}\;\fancy{X}&\cup\\
            &\llang{while}\;\fancy{E}\;\llang{do}\;\fancy{S}&\cup\\
            &\llang{if}\;\fancy{E}\;\llang{then}\;\fancy{S}\;\llang{else}\;\fancy{S}&\cup\\
            &\fancy{S}^* \, \emph{(Последовательность операторов)}
            
\end{array}
$$
\item Программы: $\fancy{P}=(\fancy{D}^*, \fancy{S})$ --- несколько определений, за которыми следует текст самой программы
\end{itemize}
\end{document}

