\documentclass{article}

\usepackage[left=1cm,right=1cm,top=1cm,bottom=1cm,bindingoffset=0cm]{geometry}
\usepackage{listings}
\usepackage{indentfirst}
\usepackage{verbatim}
\usepackage{amsmath, amsthm, amssymb}
\usepackage{stmaryrd}
\usepackage{graphicx}
\usepackage{euscript}
\usepackage{hyperref}

\usepackage[utf8]{inputenc}
\usepackage[english,russian]{babel}
\usepackage[T2A]{fontenc}

\lstdefinelanguage{llang}{
keywords={skip, do, while, read, write, if, then, else, begin, end},
sensitive=true,
%%basicstyle=\small,
commentstyle=\scriptsize\rmfamily,
keywordstyle=\ttfamily\underbar,
identifierstyle=\ttfamily,
basewidth={0.5em,0.5em},
columns=fixed,
fontadjust=true,
literate={->}{{$\to$}}1
}

\lstset{
language=llang
}

\newcommand{\aset}[1]{\left\{{#1}\right\}}
\newcommand{\term}[1]{\mbox{\texttt{\bf{#1}}}}
\newcommand{\cd}[1]{\mbox{\texttt{#1}}}
\newcommand{\sembr}[1]{\llbracket{#1}\rrbracket}
\newcommand{\conf}[1]{\left<{#1}\right>}
\newcommand{\fancy}[1]{{\cal{#1}}}

\newcommand{\trule}[2]{\frac{#1}{#2}}
\newcommand{\crule}[3]{\frac{#1}{#2},\;{#3}}
\newcommand{\withenv}[2]{{#1}\vdash{#2}}
\newcommand{\trans}[3]{{#1}\xrightarrow{#2}{#3}}
\newcommand{\ctrans}[4]{{#1}\xrightarrow{#2}{#3},\;{#4}}
\newcommand{\llang}[1]{\mbox{\lstinline[mathescape]|#1|}}
\newcommand{\pair}[2]{\inbr{{#1}\mid{#2}}}
\newcommand{\inbr}[1]{\left<{#1}\right>}
\newcommand{\highlight}[1]{\color{red}{#1}}
\newcommand{\ruleno}[1]{\eqno[\textsc{#1}]}
\newcommand{\inmath}[1]{\mbox{$#1$}}
\newcommand{\lfp}[1]{fix_{#1}}
\newcommand{\gfp}[1]{Fix_{#1}}

\newcommand{\NN}{\mathbb N}
\newcommand{\ZZ}{\mathbb Z}

\begin{document}

%% Весь этот текст можно удалить
%% ====== от сих =====
\begin{center} {\LARGE Формальные языки} \end{center}

\begin{center} {\Large домашнее задание до 23:59 29.04} \end{center}
\bigskip

\begin{enumerate}
  \item Изучить описание синтаксиса вашего второго самого любимого языка программирования. Найти пару особенностей синтаксиса, о которых вы раньше не знали. В отчете указать, где и какую спецификацию читали и привести примеры программ с найденными особенностями. (2 балла)
  \item Провести синтаксический анализ при помощи алгоритма CYK: вручную или реализовать и запустить код. Язык --- корректные арифметические выражения над цифрами с операциями $-$ и $*$ с естественными приоритетом и ассоциативностью. Проанализировать хотя бы одну корректную цепочку длины не меньше 7 и хотя бы одну некорректную цепочку длины не меньше 7. Результатом должна быть заполненная таблица и дерево вывода (в случае ошибки дерева не должно быть). (3 балла)
  \item Реализовать парсер для предложенного вами конкретного синтаксиса языка L (описание языка ниже), используя любимый способ писать парсеры. Не забыть про тесты (8 баллов)
    \begin{itemize}
        \item Можно реализовать любой понравившийся вам алгоритм синтаксического анализа.
        \item Можно пользоваться парсер-комбинаторами, можно даже реализовать свою библиотеку.
        \item Можно использовать генераторы синтаксических анализаторов (yacc, bison, antlr, любой другой).
        \item Желательно, чтобы ваш синтаксический анализатор принимал на вход то, что выдает ваш лексер --- те самые токены с позициями во входной строке; если для этого нужно править лексер --- правьте. Если технологии не сочетаются, реализовать функциональность лексического анализа, как требуется в вашем случае. Комментарии перед синтаксическим анализом имеет смысл отфильтровать
        \item Создайте консольное приложение для запуска синтаксического анализа.
        \begin{itemize}
            \item Консольное приложение обязательно должно принимать адрес файла со входной программой
            \item Программа может быть многострочной
            \item Результатом синтаксического анализа должно быть \emph{абстрактное синтаксическое дерево}, напечатанное в человекочитаемом формате; можно в файл, но название файла должно быть связано с названием входного файла (можно в dot).
            \item Лексемы в дереве вывода должны отображаться со всей необходимой информацией (тип лексемы, значение и привязка к коду).
            \item Код должен быть размещен на гитхабе, собираться одним скриптом, содержать инструкцию по сборке и запуску собранного приложения, собираться на чистой Ubuntu 16.04 или Windows 10. Все зависимости, в случае их отсутствия в системе, должны доставляться скриптом.
        \end{itemize}
     \end{itemize}
\end{enumerate}

\centering {\Large Абстрактный синтаксис языка L }
$$
X \mbox{ --- счетно-бесконечное множество идентификаторов}
$$
$$
\otimes=\{\llang{+}, \llang{-}, \llang{*}, \llang{/}, \llang{\%}, \llang{==}, \llang{!=}, 
\llang{>}, \llang{>=}, \llang{<}, \llang{<=}, \llang{\&\&}, \llang{||}\}
$$

\begin{itemize}
\item Определения (функций): $\fancy{D}=\fancy{X}_{name} \; \fancy{X}_0 \dots \fancy{X}_k \leftarrow \fancy{S}$. $\fancy{X}_{name}$ --- имя функции;  $\fancy{X}_0 \dots \fancy{X}_k$ --- ее аргументы; $\fancy{S}$~--- тело.
\item Вызовы функций: $\fancy{C} = \fancy{X}_{name} \; \fancy{E}_0 \dots \fancy{E}_k$. \emph{Аргументами могут быть произвольные выражения.}
\item Выражения: $\fancy{E}=\fancy{C} \cup X\cup\NN\cup(\fancy{E}\otimes\fancy{E})$. \emph{Вызовы функций могут быть использованы в выражениях.} В выражениях могут использоваться круглые скобки.


\item Операторы: 

$$
\begin{array}{rll}
  \fancy{S}=
            &\fancy{X}\;\llang{:=}\;\fancy{E}&\cup\\
            &\fancy{C}&\cup\\
            &\llang{write}\;\fancy{E}&\cup\\
            &\llang{read}\;\fancy{X}&\cup\\
            &\llang{while}\;\fancy{E}\;\llang{do}\;\fancy{S}&\cup\\
            &\llang{if}\;\fancy{E}\;\llang{then}\;\fancy{S}\;\llang{else}\;\fancy{S}&\cup\\
            &\fancy{S}^* \, \emph{(Последовательность операторов)}
            
\end{array}
$$
\item Программы: $\fancy{P}=(\fancy{D}^*, \fancy{S})$ --- несколько определений, за которыми следует текст самой программы
\end{itemize}
\end{document}

