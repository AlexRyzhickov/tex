\documentclass[12pt]{article}
\usepackage[left=1cm,right=2cm,top=2cm,bottom=2cm,bindingoffset=0cm]{geometry}
\usepackage{hyperref}
\usepackage{fontspec}
\usepackage{polyglossia}
\setdefaultlanguage{russian}
\setmainfont[Mapping=tex-text]{CMU Serif}

\begin{document}
%% Весь этот текст можно удалить
%% ====== от сих =====
\centering {\LARGE Формальные языки}

{\Large домашнее задание до 23:59 25.03}
\bigskip

\begin{enumerate}
  \item 
  { Упростите следующие регулярные выражения ($a, b$ --- символы алфавита), приведите аргументацию. Можно воспользоваться свойствами или построить автоматы по исходным выражениям, детерминизировать и минимизировать их, а потом по ним построить обратно регулярные выражения --- можно вручную, можно с помощью любимого программного средства.  
    \begin{enumerate} 
      \item { $ (a | b)^{∗} ab (a | b)^{∗} | (a | b)^{∗} a | b^{∗} $ }
      \item { $ (a | b)^{*} (ab | ba) (a | b)^{∗} | a^{∗} | b^{∗} $ }
    \end{enumerate}
  }
  
  \item
  { Составьте регулярные выражения для следующих языков (подвыражения можно называть и использовать названия): 
    \begin{enumerate}
      \item
      { Списки целых чисел, разделенных точкой с запятой. Между скобками и числами, числами и разделителями может быть произвольное число пробелов (символ пробела обозначайте как $\backslash s$). 
        \begin{itemize} 
          \item \verb![], [1], [1;1;2;3;5;8], [  ], [ 4; 8;   15; 16   ; 23; 42]! --- корректные списки
          \item \verb!][, [[]], [1;2;3, [a], [1,2,3], [1;2;], [1; 23  4; 5]! --- некорректные списки
        \end{itemize}
      }
      \item
      { Кортежи, где элементами могут быть целые числа, идентификаторы или списки целых чисел. Пробелы не являются значимыми символами; используйте круглые скобки и запятые как разделители. 
        \begin{itemize} 
          \item \verb!(); (1); (1,1,2,3,5,8); (   ); ( 4, eight,   15, [13; 16   ; 23; 42])! --- корректные кортежи
          \item \verb!)(; (()); (1,2,3; (1;2;3); (1,2,); (1, 23  4, 5)! --- некорректные кортежи
        \end{itemize}
      }
      \item
      { Кортежи, где элементами могут быть целые числа или кортежи. 
      \begin{itemize} 
          \item \verb!(1, (), 23); ((1, 3), (4, (5,7), 6)); (13, 42)! --- корректные кортежи
      \end{itemize}
      }
      \item Язык чисел с плавающей точкой и экспоненциальной частью.
        \begin{itemize}
            \item Алфавит: $\{e, +, -, ., 0,1,\dots,9\}$.
            \item Строки состоят из мантиссы и экспоненциальной части, разделенных буквой $e$.
            \item Мантисса состоит из целой и дробной части, разделенных точкой.
            \item Целая и экспоненциальная части не содержат лидирующих нулей.
            \item Перед целой частью может стоять один минус. 
            \item Перед экспоненциальной частью может стоять один плюс или один минус. 
            \item Целая, дробная и экспоненциальная часть могут быть опущены, но не все одновременно. 
            \item Число, состоящее только из целочисленной части, является корректным. 
            \item Число, состоящее только из дробной части, является корректным.
            \item Число, состоящее только из экспоненциальной части, является корректным.
            \item Если дробная часть опущена, а экспоненциальная присутствует, перед $e$ не должно стоять точки
            \item \verb;0, 10, 0.0, .1, 2., 3.14, 5.00005, 123.321e123, -42e+23, e5; --- числа
            \item \verb;01, ., .e, 1.3e023, +1, 1.e12,; $\varepsilon$ --- не числа
        \end{itemize}
    \end{enumerate}
  }

  \item 
  { \begin{itemize}
      \item Для всех встретившихся в задании регулярных выражений составьте минимальный ДКА. Можно вручную, можно с помощью библиотек или своего кода. 
    \end{itemize}
  }
\end{enumerate}

%% ===== и до сих =====
\end{document}
