\documentclass{article}

\usepackage[left=1cm,right=1cm,top=1cm,bottom=1cm,bindingoffset=0cm]{geometry}
\usepackage{listings}
\usepackage{indentfirst}
\usepackage{verbatim}
\usepackage{amsmath, amsthm, amssymb}
\usepackage{stmaryrd}
\usepackage{graphicx}
\usepackage{euscript}
\usepackage{hyperref}

\usepackage[utf8]{inputenc}
\usepackage[english,russian]{babel}
\usepackage[T2A]{fontenc}

\lstdefinelanguage{llang}{
keywords={skip, do, while, read, write, if, then, else, begin, end},
sensitive=true,
%%basicstyle=\small,
commentstyle=\scriptsize\rmfamily,
keywordstyle=\ttfamily\underbar,
identifierstyle=\ttfamily,
basewidth={0.5em,0.5em},
columns=fixed,
fontadjust=true,
literate={->}{{$\to$}}1
}

\lstset{
language=llang
}

\newcommand{\aset}[1]{\left\{{#1}\right\}}
\newcommand{\term}[1]{\mbox{\texttt{\bf{#1}}}}
\newcommand{\cd}[1]{\mbox{\texttt{#1}}}
\newcommand{\sembr}[1]{\llbracket{#1}\rrbracket}
\newcommand{\conf}[1]{\left<{#1}\right>}
\newcommand{\fancy}[1]{{\cal{#1}}}

\newcommand{\trule}[2]{\frac{#1}{#2}}
\newcommand{\crule}[3]{\frac{#1}{#2},\;{#3}}
\newcommand{\withenv}[2]{{#1}\vdash{#2}}
\newcommand{\trans}[3]{{#1}\xrightarrow{#2}{#3}}
\newcommand{\ctrans}[4]{{#1}\xrightarrow{#2}{#3},\;{#4}}
\newcommand{\llang}[1]{\mbox{\lstinline[mathescape]|#1|}}
\newcommand{\pair}[2]{\inbr{{#1}\mid{#2}}}
\newcommand{\inbr}[1]{\left<{#1}\right>}
\newcommand{\highlight}[1]{\color{red}{#1}}
\newcommand{\ruleno}[1]{\eqno[\textsc{#1}]}
\newcommand{\inmath}[1]{\mbox{$#1$}}
\newcommand{\lfp}[1]{fix_{#1}}
\newcommand{\gfp}[1]{Fix_{#1}}

\newcommand{\NN}{\mathbb N}
\newcommand{\ZZ}{\mathbb Z}

\begin{document}

%% Весь этот текст можно удалить
%% ====== от сих =====
\begin{center} {\LARGE Формальные языки} \end{center}

\begin{center} {\Large домашнее задание до 23:59 08.04} \end{center}
\bigskip

\begin{enumerate}
  \item Реализовать методом рекурсивного спуска синтаксический анализатор арифметических выражений. (8 баллов за полностью выполненное задание)
    \begin{itemize}
        \item Сделать консольное  приложение, принимающее на вход путь к файлу, содержащему арифметическое выражение, производящее синтаксический анализ и печатающее его результат в консоль.
        \begin{itemize}
            \item Результатом работы синтаксического анализа является, в случае успешного разбора, дерево вывода, в случае ошибки --- сообщение об ошибке. Сообщение об ошибке должно идентифицировать позицию во входном потоке, на котором она произошла.
        \end{itemize}
        \item Составить набор тестов, демонстрирующий правильность работы полученного лексера (качество тестового покрытия важно!).
        \item Код должен быть размещен на гитхабе, собираться одним скриптом, содержать инструкцию по сборке и запуску собранного приложения, собираться на чистой Ubuntu 16.04 или Windows 10. Все зависимости, в случае их отсутствия в системе, должны доставляться скриптом.
        \begin{itemize}
            \item Инструкция по запуску должна содержать информацию о том, где находится бинарник, как именно его полагается запускать, какой формат аргументов командной строки, куда пишется результат.
            \item Можно писать на любом языке программирования, но желательно не пользоваться готовыми решениями и библиотеками. 
        \end{itemize} 
     \end{itemize}
\end{enumerate}

\medskip

\begin{center} {\Large Спецификация языка арифметических выражений } \end{center}

\medskip

Язык описывает корректные арифметические выражения над натуральными числами с естественным приоритетом операций и скобками. Допустимые операции: сложение, вычитание, умножение, деление, возведение в степень. 

Примеры корректных цепочек: 

\begin{itemize}
  \item \verb!0 + 13 * 42 - 7 / 0!
  \item \verb!(0 + 13) * ((42 - 7) / 0)!
  \item \verb!1 - 2 - 3 - (5 - 6)!
  \item \verb!13!
  \item \verb!(((((13)))))!
  \item \verb!42 ^ 24 - 156 * 123!
  \item \verb!(42 ^ (24 - 156) * 123)!
\end{itemize}


\end{document}

