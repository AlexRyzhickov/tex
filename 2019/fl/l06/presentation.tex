\documentclass{beamer}
\usepackage{beamerthemesplit}
\usepackage{wrapfig}
\usetheme{SPbGU}
\usepackage{pdfpages}
\usepackage{amsmath}
\usepackage{mathtools}
\usepackage{cmap} 
\usepackage[T2A]{fontenc} 
\usepackage[utf8]{inputenc}
\usepackage[english,russian]{babel}
\usepackage{indentfirst}
\usepackage{amsmath}
\usepackage{tikz}
\usepackage{multirow}
\usepackage[noend]{algpseudocode}
\usepackage{algorithm}
\usepackage{algorithmicx}
\usepackage{ stmaryrd }
\usepackage{qtree}
\usetikzlibrary{shapes,arrows}
\usepackage{fancyvrb}

\usetikzlibrary{positioning}
\usetikzlibrary{decorations.pathmorphing}

\usetikzlibrary{shapes,arrows}
\usetikzlibrary{positioning,automata}
\tikzset{
  snake it/.style={decorate, decoration=snake},
  every state/.style={rectangle, minimum size=0.2cm},
  initial text={}
}

\newtheorem{rutheorem}{Теорема}
\newtheorem{ruproof}{Доказательство}
\newtheorem{rudefinition}{Определение}
\newtheorem{rulemma}{Лемма}
\beamertemplatenavigationsymbolsempty

\setbeamertemplate{itemize item}[circle]
\setbeamertemplate{enumerate item}[circle]

\newcommand{\derives}[1][*]{\xRightarrow[]{#1}}
\newcommand{\deriveg}[1]{\xRightarrow[#1]{*}}
\newcommand{\derivegone}[1]{\xRightarrow[#1]{}}
\newcommand{\lritem}[3]{#1 \to #2 \cdot #3}
\newcommand{\clritem}[4]{#1 \to #2 \cdot #3, \{#4\}}

\title[]{Теория автоматов и формальных языков}
\subtitle[]{Контекстно-свободные языки: LR-анализ}
\institute[]{
Санкт-Петербургский государственный электротехнический университет <<ЛЭТИ>>\\
}

\author[]{Екатерина Вербицкая}

\date{07 ноября 2019}

\definecolor{orange}{RGB}{179,36,31}

\begin{document}
{
  \begin{frame}
    \titlepage
  \end{frame}
}


\begin{frame}[fragile]
  \transwipe[direction=90]
  \frametitle{В предыдущей серии}
  \begin{itemize}
    \item Нисходящий анализ
    \item Алгоритм синтаксического анализа LL(1)
    \item LL(k) грамматики и языки 
  \end{itemize}
\end{frame}

\begin{frame}[fragile]
  \transwipe[direction=90]
  \frametitle{Восходящий синтаксический анализ}
  \begin{itemize}
    \item Начинаем с символов входной строки, строим дерево вывода до стартового нетерминала 
    \item CYK~--- один из примеров восходящего синтаксического анализа
  \end{itemize}
  
\end{frame}

\begin{frame}[fragile]
  \transwipe[direction=90]
  \frametitle{Восходящий анализ: LR}
  \begin{itemize}
    \item \textbf{L}eft-to-right, \textbf{R}ight-most вывод
    \item Разрешен предпросмотр
    \item Предиктивен
    \item Обрабатывает леворекурсивные грамматики
    \item Достаточно хорош для используемых на практике языков
  \end{itemize}
\end{frame}

\begin{frame}[fragile]
  \transwipe[direction=90]
  \frametitle{LR-анализ}
  \begin{itemize}
    \item Используют:
    \begin{itemize}
      \item Входной буфер (откуда читается входная строка)
      \item Стек (для промежуточных данных)
      \item \textbf{Таблицы} анализатора (управляет процессом разбора)
      \begin{itemize}
        \item Разные модификации используют разные таблицы
        \item Таблица определяет ``мощность'' анализатора
      \end{itemize}
    \end{itemize}
    \item Оперирует состояниями
    \item Работает за $O(n)$, где $n$~--- длина входной строки
  \end{itemize}
\end{frame}

\begin{frame}[fragile]
  \transwipe[direction=90]
  \frametitle{Таблица LR-анализатора}
\begin{center}
  Управляет процессом разбора 
\end{center}

\begin{center}
  \begin{tabular}{c||c|c|c|c||c|c|c}
      & $\dots$ & $t$ & $\dots$ & \$ & $\dots$ & $B$ & $\dots$  \\ \hline 
    $\dots$ & $\dots$ & $\dots$ & $\dots$ & $\dots$ & $\dots$ & $\dots$ & $\dots$ \\ \hline 
    $13$ & $\dots$ & $s_i$ & $\dots$ & $r_k$ & $\dots$ & $j$ & $\dots$ \\ \hline 
    $\dots$ & $\dots$ & $\dots$ & $\dots$ & $\dots$ & $\dots$ & $\dots$ & $\dots$ 
  \end{tabular}
\end{center}

\begin{itemize}
  \item $s_i$~--- shift 
  \item $r_k$~--- reduce 
  \item $j$~--- goto 
  \item $acc$~--- accept 
\end{itemize}
\end{frame}

\begin{frame}[fragile]
  \transwipe[direction=90]
  \frametitle{$LR(0)$ анализ}

  \begin{itemize}
    \item Разбирает наименьший класс языков 
    \item Использует $LR(0)$ пункты: $\lritem{A}{\alpha}{\beta}$, где $A \to \alpha \beta$~--- правило грамматики 
    \item Множества $LR(0)$ пунктов суть состояния анализатора
  \end{itemize}

\end{frame}

\begin{frame}[fragile]
  \transwipe[direction=90]
  \frametitle{$Closure$}
  Используется при вычислении множеств $LR(0)$ пунктов, которые могут быть применены на данном этапе во время синтаксического анализа

  \begin{itemize}
    \item Все пункты из $I$ в $closure(I)$
    \item Если $\lritem{A}{\alpha}{B \beta} \in closure(I) \text{ и } B \to \gamma$~--- правило грамматики, то
    $\lritem{B}{}{\gamma} \in closure(I)$
  \end{itemize}
\end{frame}

\begin{frame}[fragile]
  \transwipe[direction=90]
  \frametitle{Пример $closure$}
  
  \begin{align*}
    S' &\to S \\ 
    S  &\to A A \\ 
    A  &\to a A \mid b 
  \end{align*}

  \[
  \begin{array}{rcclr}
    closure(\{\lritem{S'}{}{S}\}) & = & \{ &\lritem{S'}{}{S} & \\
                                  &   &    &\lritem{S}{}{AA} & \\ 
                                  &   &    &\lritem{A}{}{aA} & \\ 
                                  &   &    &\lritem{A}{}{b}  &\}
  \end{array}  
  \]

  \[
  \begin{array}{rcclr}
    closure(\{\lritem{A}{a}{A}\}) & = & \{ & \lritem{A}{a}{A} & \\ 
                                  &   &    & \lritem{A}{}{aA} & \\ 
                                  &   &    & \lritem{A}{}{b}  & \}
  \end{array}
  \]

\end{frame}


\begin{frame}[fragile]
  \transwipe[direction=90]
  \frametitle{$goto$}
  
  
  \begin{center}
    $goto(I, X)$~--- передвигаем точку за символ $X$ во всех пунктах в $I$
  \end{center}

  \begin{center}
    Если $\lritem{A}{\alpha}{X \beta} \in I$ , добавляем $closure(\{ \lritem{A}{aX}{\beta} \})$ в $goto(I, X)$
  \end{center}
\end{frame}


\begin{frame}[fragile]
  \transwipe[direction=90]
  \frametitle{Пример $goto$}
  
  \begin{align*}
    S' &\to S \\ 
    S  &\to A A \\ 
    A  &\to a A \mid b 
  \end{align*}

  \[
  \begin{array}{rcclr}
    I & = & \{ &\lritem{S'}{}{S} & \\
      &   &    &\lritem{S}{}{AA} & \\ 
      &   &    &\lritem{A}{}{aA} & \\ 
      &   &    &\lritem{A}{}{b}  &\}
  \end{array}  
  \]

  \[
  \begin{array}{rcclr}
    goto(I, A) & = & \{ & \lritem{S}{A}{A} & \\ 
               &   &    & \lritem{A}{}{aA} & \\ 
               &   &    & \lritem{A}{}{b}  & \}
  \end{array}
  \]

\end{frame}

\begin{frame}[fragile]
  \transwipe[direction=90]
  \frametitle{Автомат LR(0)-анализатора}
  \begin{itemize}
    \item Состояния~--- множества пунктов
    \item Переходы по символам грамматики 
    \item Начальное состояние~--- $closure(\{\lritem{S'}{}{S}\})$ 
    \item Следующие состояния считаются при помощи $goto(I, X)$
  \end{itemize}

\end{frame}

\begin{frame}[fragile]
  \transwipe[direction=90]
  \frametitle{Пример LR(0)-автомата}
\begin{center}
  \begin{tikzpicture}[> = stealth,node distance=3cm, on grid]
    \node[draw=none, fill=none] at (-4, 3)      {1};
    \node[draw=none, fill=none] at (-4.1, 1.1)  {0};
    \node[draw=none, fill=none] at (-1.2, 3.7)  {2};
    \node[draw=none, fill=none] at (-0.7, 0.5)  {4};
    \node[draw=none, fill=none] at (-0.8, -1.8) {3};
    \node[draw=none, fill=none] at (2.3, 3.5)   {5};
    \node[draw=none, fill=none] at (2.3, -2.5)  {6};

    \node [state] (s_4) 
      {
        $\lritem{A}{b}{}$
      };
    \node [state] (s_0) [left=of  s_4] 
      {
        $
        \begin{aligned}
          S' &\to \cdot S  \\ 
          S  &\to \cdot AA \\ 
          A  &\to \cdot aA \\ 
          A  &\to \cdot b  
        \end{aligned}
        $
      };
    \node [state] (s_3) [below=of s_4] 
      {
        $
        \begin{aligned}
          A &\to a \cdot A \\ 
          A &\to \cdot a A \\ 
          A &\to \cdot b 
        \end{aligned}
        $
      };
    \node [state] (s_2) [above=of s_4] 
      {
        $
        \begin{aligned}
          S &\to A \cdot A \\ 
          A &\to \cdot a A \\ 
          A &\to \cdot b 
        \end{aligned}
        $
      };
    \node [state] (s_1) [above=of s_0] 
      {
        $\lritem{S'}{S}{}$
      };
    \node [state] (s_5) [right=of s_2] 
      {
        $\lritem{S}{AA}{}$
      };
    \node [state] (s_6) [right=of s_3] 
      {
        $\lritem{A}{aA}{}$
      };

    \path[->] (s_0) edge [left]                node {$S$} (s_1)
                    edge [above]               node {$A$} (s_2)
                    edge [above]               node {$b$} (s_4) 
                    edge [right]               node {$a$} (s_3)
              (s_2) edge [above]               node {$A$} (s_5)
                    edge [right]               node {$b$} (s_4)
                    edge [right, bend left=30] node {$a$} (s_3)
              (s_3) edge [right]               node {$b$} (s_4)
                    edge [above]               node {$A$} (s_6)
                    edge [loop left]           node {$a$} ()
    ;
  \end{tikzpicture}
\end{center}
\end{frame}

\begin{frame}[fragile]
  \transwipe[direction=90]
  \frametitle{Таблица LR(0)-анализатора}
  \begin{itemize}
    \item По горизонтали: состояния
    \item По вертикали: терминалы + \$ + нетерминалы
  \end{itemize}

  \begin{itemize}
    \item $acc$ в ячейку, соответствующую стартовому состоянию и \$ 
    \item $s_i$ в ячейку $(j,t)$, если в автомате есть переход из состояния $j$ по терминалу $t$ в состояние $i$
    \item $i$ в ячейку $(j, N)$, если в автомате есть переход из состояния $j$ по нетерминалу $N$ в состояние $i$
    \item $r_k$ в ячейку $(j,t)$, если в состоянии $j$ есть пункт $\lritem{A}{\alpha}{}$, где $A \to \alpha$~--- $k$-ое правило грамматики, $t$~--- терминал грамматики
  \end{itemize}
\end{frame}

\begin{frame}[fragile]
  \transwipe[direction=90]
  \frametitle{Таблица LR(0)-анализатора}
  \begin{center}
    \begin{tabular}{c||c|c|c||c|c}
        & $a$   & $b$   & \$    & $A$ & $S$ \\ \hline \hline 
      0 & $s_3$ & $s_4$ &       & $2$ & $1$ \\ \hline 
      1 &       &       & $acc$ &     &     \\ \hline 
      2 & $s_3$ & $s_4$ &       & $5$ &     \\ \hline 
      3 & $s_3$ & $s_4$ &       & $6$ &     \\ \hline 
      4 & $r_3$ & $r_3$ & $r_3$ &     &     \\ \hline 
      5 & $r_1$ & $r_1$ & $r_1$ &     &     \\ \hline 
      6 & $r_2$ & $r_2$ & $r_2$ &     &    
    \end{tabular}
  \end{center}
\end{frame}

\begin{frame}[fragile]
  \transwipe[direction=90]
  \frametitle{Таблица SLR(1)-анализатора}
  \begin{center}
    $r_k$ в ячейку $(j,t)$, если в состоянии $j$ есть пункт $\lritem{A}{\alpha}{}$, 
    
    где $A \to \alpha$~--- $k$-ое правило грамматики, $t \in FOLLOW(A)$
  \end{center}

  \begin{center}
    \begin{tabular}{c||c|c|c||c|c}
        & $a$   & $b$   & \$    & $A$ & $S$ \\ \hline \hline 
      0 & $s_3$ & $s_4$ &       & $2$ & $1$ \\ \hline 
      1 &       &       & $acc$ &     &     \\ \hline 
      2 & $s_3$ & $s_4$ &       & $5$ &     \\ \hline 
      3 & $s_3$ & $s_4$ &       & $6$ &     \\ \hline 
      4 & $r_3$ & $r_3$ & $r_3$ &     &     \\ \hline 
      5 &       &       & $r_1$ &     &     \\ \hline 
      6 & $r_2$ & $r_2$ & $r_2$ &     &    
    \end{tabular}
  \end{center}
\end{frame}


\begin{frame}[fragile]
  \transwipe[direction=90]
  \frametitle{Пример синтаксического анализа SLR(1)}

  \begin{center}
    \begin{tabular}{c||c|c|c||c|c}
        & $a$   & $b$   & \$    & $A$ & $S$ \\ \hline \hline 
      0 & $s_3$ & $s_4$ &       & $2$ & $1$ \\ \hline 
      1 &       &       & $acc$ &     &     \\ \hline 
      2 & $s_3$ & $s_4$ &       & $5$ &     \\ \hline 
      3 & $s_3$ & $s_4$ &       & $6$ &     \\ \hline 
      4 & $r_3$ & $r_3$ & $r_3$ &     &     \\ \hline 
      5 &       &       & $r_1$ &     &     \\ \hline 
      6 & $r_2$ & $r_2$ & $r_2$ &     &    
    \end{tabular}
  \end{center}

  Строка: $aabb\$$

  Стек:  $0, a, 3, a, 3, b, 4, A, 6, A, 6, A, 2, b, 4, A, 5, S, 1$

\end{frame}

\begin{frame}[fragile]
  \transwipe[direction=90]
  \frametitle{Canonical LR}

\begin{center}
  Пункты дополняются множеством предпросмотра (lookahead): $\clritem{A}{\alpha}{\beta}{\gamma_0, \dots, \gamma_n}$
\end{center}

\begin{itemize}
  \item $A \to \alpha \beta$ --- правило грамматики
  \item $\gamma_0, \dots, \gamma_n$ --- терминалы
  \item Множество предпросмотра --- терминалы, которые должны встретиться в выведенной строке сразу после строки, выводимой из данного правила 
\end{itemize}
\end{frame}

\begin{frame}[fragile]
  \transwipe[direction=90]
  \frametitle{Closure in CLR}
  \begin{itemize}
    \item Все пункты из $I$ в $closure(I)$
    \item Если $\clritem{A}{\alpha}{B \beta}{\gamma_0, \dots, \gamma_n} \in closure(I)$ и 
    
    $B \to \delta$~--- правило грамматики, то
    
    $\clritem{B}{}{\delta}{FIRST(\beta \gamma_0), \dots, FIRST(\beta \gamma_n)} \in closure(I)$
  \end{itemize}
\end{frame}

\begin{frame}[fragile]
  \transwipe[direction=90]
  \frametitle{Goto in CLR}
  \begin{center}
    Если $\clritem{A}{\alpha}{X \beta}{\overline{\gamma_i}} \in I$ , 
    
    добавляем $closure(\{ \clritem{A}{aX}{\beta}{\overline{\gamma_i}} \})$ в $goto(I, X)$
  \end{center}
\end{frame}

\begin{frame}[fragile]
  \transwipe[direction=90]
  \frametitle{Пример CLR автомата }
  \begin{center}
    \begin{tikzpicture}[> = stealth,node distance=3.25cm, on grid, scale=0.8, every node/.style={scale=0.8}]
      \node[draw=none, fill=none] at (-1.4, 0.9)  {0};
      \node[draw=none, fill=none] at (3.1, 0.55)  {3};
      \node[draw=none, fill=none] at (-1.4, -1.6) {1};
      \node[draw=none, fill=none] at (-1.4, -4.1) {2};
      \node[draw=none, fill=none] at (2.9, -1.95) {4};
      \node[draw=none, fill=none] at (2.9, -4.45) {5};
      \node[draw=none, fill=none] at (6.7, -1.6)  {6};
      \node[draw=none, fill=none] at (6.7, -4.1)  {7};
      \node[draw=none, fill=none] at (11, -1.95)  {8};
      \node[draw=none, fill=none] at (11, -4.45)  {9};
      \node[state] (s_0)
      {
        $
        \begin{aligned}
          S' &\to \cdot S, \{\$\}  \\ 
          S  &\to \cdot \mid \cdot ( S ) S, \{\$\}
        \end{aligned}
        $
      };
      \node[state] (s_1) [below=2cm of s_0] 
      {
        $
        \begin{aligned}
          S  &\to ( \cdot S ) S, \{\$\} \\ 
          S  &\to \cdot \mid \cdot ( S ) S, \{)\}
        \end{aligned}
        $
      };
      \node[state] (s_2) [below=2cm of s_1] 
      {
        $
        \begin{aligned}
          S  &\to ( \cdot S ) S, \{)\} \\ 
          S  &\to \cdot \mid \cdot ( S ) S, \{)\}
        \end{aligned}
        $
      };
      \node[state] (s_3) [right=of s_0] 
      {
        $ S' \to S \cdot, \{\$\} $
      };
      \node[state] (s_4) [right=of s_1]
      {
        $ S \to ( S \cdot ) S, \{\$\} $
      };
      \node[state] (s_5) [right=of s_2] 
      {
        $ S \to ( S \cdot ) S, \{)\} $
      };
      \node[state] (s_6) [right=of s_4] 
      {
        $
        \begin{aligned}
          S  &\to ( S ) \cdot S, \{\$\} \\ 
          S  &\to \cdot \mid \cdot ( S ) S, \{\$\}
        \end{aligned}
        $
      };
      \node[state] (s_7) [right=of s_5]
      {
        $
        \begin{aligned}
          S  &\to ( S ) \cdot S, \{)\} \\ 
          S  &\to \cdot \mid \cdot ( S ) S, \{)\}
        \end{aligned}
        $
      };
      \node[state] (s_8) [right=of s_6] 
      {
        $ S \to ( S ) S \cdot, \{\$\} $
      };
      \node[state] (s_9) [right=of s_7] 
      {
        $ S \to ( S ) S \cdot, \{)\} $
      };
      
      \path[->] 
        (s_0) edge [left]                 node {$($} (s_1)
              edge [above]                node {$S$} (s_3)
        (s_1) edge [left]                 node {$($} (s_2)
              edge [above]                node {$S$} (s_4)
        (s_2) edge [loop below]           node {$($} ()
              edge [above]                node {$S$} (s_5)
        (s_4) edge [above]                node {$)$} (s_6)
        (s_5) edge [above]                node {$)$} (s_7)
        (s_6) edge [above]                node {$S$} (s_8)
              edge [above, bend right=20] node {$($} (s_1)
        (s_7) edge [above]                node {$S$} (s_9)
              edge [below, bend left=20]  node {$($} (s_2)
        ;
    \end{tikzpicture}
  \end{center}
\end{frame}

\begin{frame}[fragile]
  \transwipe[direction=90]
  \frametitle{Таблица CLR-анализатора}
  \begin{center}
    $r_k$ в ячейку $(j,\gamma_i)$, если в состоянии $j$ есть пункт $\clritem{A}{\alpha}{}{\gamma_0, \dots, \gamma_n}$, 
    
    где $A \to \alpha$~--- $k$-ое правило грамматики
  \end{center}

  \begin{center}
    \begin{tabular}{c||c|c|c||c}
        & $($   & $)$   & \$    & $S$ \\ \hline \hline 
      0 & $s_1$ &       & $r_2$ & $3$ \\ \hline 
      1 & $s_2$ & $r_2$ &       & $4$ \\ \hline 
      2 & $s_2$ & $r_2$ &       & $5$ \\ \hline 
      3 &       &       & $acc$ &     \\ \hline 
      4 &       & $s_6$ &       &     \\ \hline 
      5 &       & $s_7$ &       &     \\ \hline 
      6 & $s_1$ &       & $r_2$ & $8$ \\ \hline
      7 & $s_2$ & $r_2$ &       & $9$ \\ \hline 
      8 &       &       & $r_1$ &     \\ \hline 
      9 &       & $r_1$ &       &     
    \end{tabular}
  \end{center}
\end{frame}

\end{document}
 

  
