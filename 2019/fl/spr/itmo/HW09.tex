\documentclass[12pt]{article}
\usepackage[left=2cm,right=2cm,top=2cm,bottom=2cm,bindingoffset=0cm]{geometry}
\usepackage{fontspec}
\usepackage{polyglossia}
\setdefaultlanguage{russian}
\setmainfont[Mapping=tex-text]{CMU Serif}

\begin{document}
\begin{center}
\LARGE {Формальные языки}

\Large {домашнее задание до 23:59 17.04}
\end{center} 

\bigskip

Ветку завела: 

\url{https://github.com/kajigor/fl\_ifmo\_2019\_spr/tree/HW09}

\begin{enumerate}
  \item Добавить в парсер для выражений переменные, унарный минус и логическое отрицание. Синтаксически переменные выглядят как идентификаторы, начинающиеся со строчной буквы или символа подчеркивания.
  \item Реализовать оптимизации для выражений. Технически это функция, определенная на абстрактных синтаксических деревьях выражений, в деревья же. Какие оптимизации применять --- ваше решение. Примером могут быть тождества в духе \verb!<expression> * 0 = 0!. В отчете обязательно укажите, какие оптимизации вы реализовывали. Чем больше оптимизаций, тем лучше. Не забывайте про тесты. 
  \item Предложить конкретный (и абстрактный) синтаксис для языка, который вам предстоит парсить в будущем. Наряду с выражениями с переменными, в нем должны быть связывания (по типу \verb!let!-выражений), условный оператор \verb!if!, возможность объявлять и вызывать функции.
\end{enumerate}

\begin{table}[h]
\centering
\begin{tabular}{l|l|l}
Приоритет & Оператор & Ассоциативность \\ \hline
Высший & \verb!^! & Правоассоциативна \\
       & \verb!*!, \verb!/! & Левоассоциативна  \\
       & \verb!+!, \verb!-! & Левоассоциативна  \\
       & \verb!==!, \verb!/=!, \verb!<=!, \verb!<!, \verb!>=!, \verb!>! & Неассоциативна \\
       & \verb!&&! & Правоассоциативна \\
Низший & \verb!||! & Правоассоциативна 
\end{tabular}
\end{table}
\end{document}
