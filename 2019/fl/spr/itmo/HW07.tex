\documentclass[12pt]{article}
\usepackage[left=2cm,right=2cm,top=2cm,bottom=2cm,bindingoffset=0cm]{geometry}
\usepackage{fontspec}
\usepackage{polyglossia}
\setdefaultlanguage{russian}
\setmainfont[Mapping=tex-text]{CMU Serif}

\begin{document}
\begin{center}
\LARGE {Формальные языки}

\Large {домашнее задание до 23:59 03.04}
\end{center} 

\bigskip

Шаблон кода для задания живет в соответсвующей ветке: 

\url{https://github.com/kajigor/fl\_ifmo\_2019\_spr/tree/HW07}

\begin{enumerate}
  \item
  Привести грамматику арифметических выражений из прошлого домашнего задания в Нормальную Форму Хомского: можно вручную или написать свой преобразователь. В отчете привести грамматику по итогу каждого шага.  
  \item 
  Реализовать парсер арифметических выражений из прошлой домашки, используя ваши парсер-комбинаторы. Результатом является абстрактное синтаксическое дерево (тип данных вам предоставлен). Если тип результата парсера не совпадает с тем, что указан в шаблоне, можете его изменить. Не забывайте про тесты.
\end{enumerate}

\begin{table}[h]
\centering
\begin{tabular}{l|l|l}
Приоритет & Оператор & Ассоциативность \\ \hline
Высший & \verb!^! & Правоассоциативна \\
       & \verb!*!, \verb!/! & Левоассоциативна  \\
       & \verb!+!, \verb!-! & Левоассоциативна  \\
       & \verb!==!, \verb!/=!, \verb!<=!, \verb!<!, \verb!>=!, \verb!>! & Неассоциативна \\
       & \verb!&&! & Правоассоциативна \\
Низший & \verb!||! & Правоассоциативна 
\end{tabular}
\end{table}
\end{document}
