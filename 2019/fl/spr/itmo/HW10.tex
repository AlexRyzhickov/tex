\documentclass[12pt]{article}
\usepackage[left=2cm,right=2cm,top=2cm,bottom=2cm,bindingoffset=0cm]{geometry}
\usepackage{fontspec}
\usepackage{polyglossia}
\usepackage{proof}

\setdefaultlanguage{russian}
\setmainfont[Mapping=tex-text]{CMU Serif}

\begin{document}
\begin{center}
\LARGE {Формальные языки}

\Large {домашнее задание до 23:59 24.04}
\end{center} 

\bigskip

Ветку завела: 

\url{https://github.com/kajigor/fl\_ifmo\_2019\_spr/tree/HW10}

\begin{enumerate}
  \item Добавить в спецификацию языка шаблоны и сопоставления с ними. Должна быть возможность специфицировать алгебраические типы данных по типу хаскельной data, а также возможность осуществлять сопоставление с шаблоном как в аргументе функции, так и при помощи специальной конструкции. В отчете привести абстрактный и конкретный синтаксис получившегося языка. 
  \item Реализовать парсер для языка с выражениями, условным оператором, определениями функций и шаблонами. Реализация должна быть в соответствии с вашей спецификацией конкретного и абстрактного синтаксиса. 
  \item Реализовать проверку типов для нашего языка: вплоть до шаблонов и унарных операций. Правила типизации шаблонов прислать в отчете. Проверку типов не обязательно совмещать с синтаксическим анализом, это может быть отдельная функция над вашими абстрактными синтаксическими деревьями. 
\end{enumerate}

\infer[, v \in Int]{v : Int}{}

\bigskip

\infer[, v \in Bool]{v : Bool}{}

\bigskip

\infer[, \oplus - \textrm{ arithmetic binop}]{e_1 \oplus e_2 : Int}{
  e_1 : Int &
  e_2 : Int
}

\bigskip

\infer[, = - \textrm{ comparison binop}]{e_1 = e_2 : Bool}{
  e_1 : A &
  e_2 : A
}

\bigskip

\infer[, \otimes - \textrm{ logical binop}]{e_1 \otimes e_2 : Bool}{
  e_1 : Bool &
  e_2 : Bool
}

\bigskip

\infer{\textrm{if } c \textrm{ then } t \textrm{ else } r : A}{
  c : Bool &
  t : A &
  r : A
}

\bigskip 

\infer{\Gamma \vdash x : A}{
  x : A \in \Gamma
}

\bigskip 

\infer{\Gamma \vdash \textrm{let } x = e_1 \textrm{ in } e_2 : B }{
  \Gamma \vdash e_1 : A & 
  \Gamma, x : A \vdash e_2 : B
}

\bigskip

\infer{\Delta; \Gamma \vdash f(e) : B}{
  \langle f(x : A) = b : B \rangle \in \Delta & 
  \Delta; \Gamma \vdash e : A & 
  \Delta; \Gamma, x : A \vdash b : B
}

\bigskip 

\infer[, p = e]{\Delta; \Gamma \vdash p : A}{
  \Delta; \Gamma \vdash e : A
}

\bigskip

\infer[, p = \langle f(x : t) = b : t_1 \rangle ; p_1]{\Delta; \Gamma \vdash p_1 : A}{
  \Delta, \langle f(x : t) = b : t_1 \rangle; \Gamma \vdash p_1 : A
} 
\end{document}
