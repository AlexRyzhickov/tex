\documentclass{article}

\usepackage[left=2cm,right=2cm,top=2cm,bottom=2cm,bindingoffset=0cm]{geometry}
\usepackage{listings}
\usepackage{indentfirst}
\usepackage{verbatim}
\usepackage{amsmath, amsthm, amssymb}
\usepackage{stmaryrd}
\usepackage{graphicx}
\usepackage{euscript}
\usepackage{hyperref}

\usepackage[utf8]{inputenc}
\usepackage[english,russian]{babel}
\usepackage[T2A]{fontenc}

\begin{document}

\begin{center} {\LARGE Формальные языки} \end{center}

\begin{center} {\Large домашнее задание до 23:59 06.03} \end{center}
\bigskip

Шаблон кода для задания живет в соответсвующей ветке: 

\href{https://github.com/kajigor/fl_ifmo_2019_spr/tree/HW04}{https://github.com/kajigor/fl\_ifmo\_2019\_spr/tree/HW04}

\begin{enumerate}
  \item 
  {
    Модифицировать комбинаторы, чтобы они работали с новым типом парсера: 
    
    \verb!newtype Parser str ok err = Parser { runParser :: str -> Either err (str, ok) }!
  }
  
  \item 
  {
  Придумать абстракцию входного потока, чтобы во время сообщения об ошибках можно было указывать координаты во входе. Поддержать в комбинаторах и парсерах.
  }
  
  \item
  {
    Реализовать проверки для автоматов: 
    
    \begin{itemize} 
      \item На детерминированность
      \item На недетерминированность
      \item На полноту
      \item На минимальность
    \end{itemize}
    
    Преобразовывать автоматы пока не нужно
  }
\end{enumerate}


    Не забывайте писать свои тесты, пожалуйста. 
\end{document}

