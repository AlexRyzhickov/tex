\documentclass[12pt]{article}
\usepackage[left=1cm,right=1.5cm,top=2cm,bottom=2cm,bindingoffset=0cm]{geometry}
\usepackage{hyperref}
\usepackage{fontspec}
\usepackage{polyglossia}
\usepackage{amssymb}
\usepackage{cprotect}
\setdefaultlanguage{russian}
\pagestyle{empty}
\newfontfamily{\cyrillicfonttt}{Liberation Mono}
\setmainfont[Mapping=tex-text]{CMU Serif}

\begin{document}
\begin{center}
{\LARGE Формальные языки}

\bigskip

{\Large Дмитрий Иванов}
\end{center}

\bigskip

Предложить конкретный синтаксис для языка \verb!M!. При помощи любого инструмента реализовать парсер для этого языка.

\begin{itemize}
  \item Конкретный синтаксис должен быть описан человекочитаемо: грамматика для antlr, конечно, специфицирует конкретный синтаксис, но лучше все-таки добавить подробностей и объяснений, что именно вы хотели ею выразить.
  \item Парсер должен быть встроен в консольное приложение, принимающее на вход путь к входному файлу.
  \item В результате работы парсера на некорректном входе пользователь должен получить цивилизованное сообщение об ошибке (пробросить исключение наружу --- плохая идея).
  \item В результате работы парсера на корректном входе должно получаться абстрактное синтаксическое дерево, которое соответствует описанию абстрактного синтаксиса ниже. Его можно либо вывести в консоль, либо сохранить в файл --- как удобнее.
  \item Конкретный синтаксис может быть вообще любым, учитывайте, что парсить этот язык вам же.
  \item Необходимо написать тесты.
\end{itemize}


\begin{center}
    \Large{Абстрактный синтаксис языка \verb!М!}
\end{center}

Программа на языке \verb!M! состоит из возможно пустого множества определений отношений, за которыми следует цель.

Определение отношения состоит из названия отношения, имен аргументов и тела.

Тело отношения --- цель.

Цель --- унификация двух термов; конъюнкция или дизъюнкция двух целей; связывание свободных переменных или вызов отношений.

$X$ --- имена переменных, $C$ --- имена конструкторов, $R$ --- имена отношений.


\begin{figure*}[h!]
  \[
  \begin{array}{cccll}
    &\mathcal{T} & = & X \cup \{C (t_1, \dots, t_i) \mid t_j\in\mathcal{T}\} & \mbox{термы над переменными} \\
    &\mathcal{G} & = & \mathcal{T}\equiv\mathcal{T}   &  \mbox{унификация термов} \\
    &            &   & \mathcal{G}\wedge\mathcal{G}     & \mbox{конъюнкция целей} \\
    &            &   & \mathcal{G}\vee\mathcal{G}       &\mbox{дизъюнкция целей} \\
    &            &   & \mbox{fresh}\;\{X\}\;.\;\mathcal{G} & \mbox{свободные переменные} \\
    &            &   & R (t_1,\dots,t_i),\;t_j\in\mathcal{T} & \mbox{вызов отношения} \\
    &\mathcal{P} & = & \{R = \lambda\;x_1\dots x_{i}\,.\, g;\}\; g & \mbox{программа}
  \end{array}
  \]
  \cprotect\caption{Абстрактный синтаксис языка \verb!M!}
  \label{syntax}
\end{figure*}


\end{document}