\documentclass[12pt]{article}
\usepackage[left=1cm,right=1.5cm,top=2cm,bottom=2cm,bindingoffset=0cm]{geometry}
\usepackage{hyperref}
\usepackage{fontspec}
\usepackage{polyglossia}
\usepackage{amssymb}
\usepackage{cprotect}
\setdefaultlanguage{russian}
\pagestyle{empty}
\newfontfamily{\cyrillicfonttt}{Liberation Mono}
\setmainfont[Mapping=tex-text]{CMU Serif}

\begin{document}
\begin{center}
{\LARGE Формальные языки}

\bigskip

{\Large Екатерина Леденева}
\end{center}

\bigskip

Сравнить возможности инструментов для синтаксического анализа.

\begin{itemize}
  \item Выбрать три инструмента для синтаксического анализа:
  \begin{itemize}
    \item Генератор парсеров, реализующий алгоритм семейства LL.
    \item Генератор парсеров, реализующий алгоритм семейства LR.
    \item Библиотека парсер-комбинаторов.
  \end{itemize}
  \item Изучить, какие классы языков они поддерживают: привести грамматики, которые обрабатывает один инструмент, но не обрабатывает другой.
  \item Изучить, как эти инструменты поддерживают нетривиальные особенности синтаксиса, привести примеры парсеров с такими особенностями.
  \begin{itemize}
    \item Контекстно-чувствительная лексика.
    \item Многострочные комментарии с контролем правильной вложенности скобок.
    \item Неоднозначность грамматик.
    \item Поддержка часто повторяющихся паттернов типа арифметики, списков с разделителями, комментариев...
  \end{itemize}
  \item Подготовить отчет
\end{itemize}

Альтернативно: сравнить возможности алгоритмов синтаксического анализа.

\begin{itemize}
  \item Выбрать несколько алгоритмов синтаксического анализа:
  \begin{itemize}
    \item CYK, Earley
    \item LL, GLL
    \item LR, Tomita GLR, RNGLR, BRNGLR
    \item Parsing by derivatives
    \item Parser expression grammars, parser combinators
    \item Что-нибудь про multiple context-free grammar
  \end{itemize}
  \item Сравнить временную и пространственную сложность, классы поддерживаемых языков и грамматик.
  \item Подготовить отчет с примерами и пояснениями.
\end{itemize}

\end{document}