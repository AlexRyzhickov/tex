\documentclass[12pt]{article}
\usepackage[left=1cm,right=1.5cm,top=2cm,bottom=2cm,bindingoffset=0cm]{geometry}
\usepackage{hyperref}
\usepackage{fontspec}
\usepackage{polyglossia}
\usepackage{amssymb}
\setdefaultlanguage{russian}
\pagestyle{empty}
\newfontfamily{\cyrillicfonttt}{Liberation Mono}
\setmainfont[Mapping=tex-text]{CMU Serif}

\begin{document}
\begin{center}
{\LARGE Формальные языки}

\bigskip

{\Large Тигран Акопян}
\end{center}

\bigskip

С использованием любого инструмента реализовать парсер для языка P. Намеренно плохое описание конкретного синтаксиса ниже.

\begin{itemize}
  \item Парсер должен быть встроен в консольное приложение, принимающее на вход путь к входному файлу.
  \item В результате работы парсера на некорректном входе пользователь должен получить цивилизованное сообщение об ошибке (пробросить исключение наружу --- плохая идея).
  \item В результате работы парсера на корректном входе должно получаться абстрактное синтаксическое дерево, которое соответствует описанию абстрактного синтаксиса ниже. Его можно либо вывести в консоль, либо сохранить в файл --- как удобнее.
  \item Конкретный синтаксис может быть вообще любым, учитывайте, что парсить этот язык вам же.
  \item Необходимо написать тесты.
\end{itemize}

\begin{center}
    \Large{Конкретный синтаксис языка \verb!P!}
\end{center}

Программа на языке \verb!P! состоит из возможно пустого множества определений отношений, за которыми следует цель, отделенная символом \verb!:?!.

Отношение состоит из нескольких строк, его задающих.

Каждая строка состоит из головы и тела, разделенных символом \verb!::!, в конце стоит точка (\verb!.!).

Голова является атомом.

Атом состоит из идентификатора (название отношения или конструктора) и списка аргументов в круглых скобках (\verb!(!, \verb!)!), разделенных запятыми (\verb!,!).

Аргументом может быть либо переменная, либо атом.

Если список аргументов атома пуст, скобки опускаются.

Переменная всегда начинается с заглавной латинской буквы. Идентификатор --- со строчной латинской буквы.
Имена переменных и идентификаторов содержат только латинские буквы либо цифры \verb!0! -- \verb!9!.

Тело является целью, также может быть пустым.

Цель --- арифметика над атомами с операциями конъюнкция (\verb!,!) и дизъюнкция (\verb!;!), может содержать скобки (\verb!(!, \verb!)!). Конъюнкция обладает большим приоритетом, чем дизъюнкция. Обе операции правоассоциативны.

Если тело пусто, символ \verb!::! опускается.

Пробелы и переносы строк между лексемами не являются значащими. Их удаление не должно повлиять на результат синтаксического анализа корректной программы

\newpage

\begin{center}
    \Large{Пример программы на языке \verb!P!}
\end{center}

\begin{verbatim}
eval(St, var(X), U) :: elem(X, St, U).
eval(St, conj(X,Y), true) :: eval(St, X, true), eval(St, Y, true).
eval(St, conj(X,Y), false) :: eval(St, X, false); eval(St, Y, false).
eval(St, disj(X,Y), true) :: eval(St, X, true); eval(St, Y, true).
eval(St, disj(X,Y), false) :: eval(St, X, false), eval(St, Y, false).
eval(St, not(X), true) :: eval(St, X, false).
eval(St, not(X), false) :: eval(St, X, true).

elem(zero, cons(H,T), H).
elem(succ(N), cons(H,T), V) :: elem(N, T, V).

:? eval(St, conj(disj(X,Y),not(var(Z))), true).
\end{verbatim}

\begin{center}
  \Large{Еще пример программы на языке \verb!P!}
\end{center}

\begin{verbatim}
eval(St, var(X), U) :: elem(X, St, U).
eval(St, conj(X,Y), U)  :: eval(St, X, V), eval(St, Y, W), and(V, W, U).
eval(St, disj(X,Y), U)  :: eval(St, X, V), eval(St, Y, W), or(V, W, U).
eval(St, not(X), U) :: eval(St, X, V), neg(U, V).

elem(zero, cons(H,T), H).
elem(succ(N), cons(H,T), V) :: elem(N, T, V).

nand(false, false, true).
nand(false, true,  true).
nand(true,  false, true).
nand(true,  true,  false).

neg(X, R) :: nand(X, X, R).

or(X, Y, R) :: nand(X, X, Xx), nand(Y, Y, Yy), nand(Xx, Yy, R).

and(X, Y, R) :: nand(X, Y, Xy), nand(Xy, Xy, R).

:? eval(St, conj(disj(X,Y),not(var(Z))), true).
\end{verbatim}
\end{document}