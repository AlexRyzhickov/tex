\section{Example}
\label{example}

In this section we demonstrate by example how conservative partial deduction works.
The example program is a relational interpreter of propositional formulas under given variable assignments.
The complete code of the example program is provided in Listing~\ref{eval:whole}.

The relation \lstinline{eval$^o$} has three arguments.
The first argument, \lstinline{s}, is a list of boolean values which plays the role of variable assignments.
The $i$-th value of the substitution is the value of the $i$-th variable.
The second argument, \lstinline{fm}, is a formula with the following abstract syntax.
A formula is either a \emph{variable} represented with a Peano number, a \emph{negation} of a formula, a \emph{conjunction} of two formulas or a \emph{disjunction} of two formulas.
The third argument, \lstinline{res}, is the value of the formula under the given assignment.

The relation \lstinline{eval$^o$} is in canonical normal form: it is a single disjunction which consists of 4 conjunctions of unifications and relation calls, and all its fresh variables are introduced at the top level.
The unification in each conjunction determines the shape of the input formula and binds variables to the corresponding subformulas.
For each of the subformulas, the relation \lstinline{eval$^o$} is called recursively.
Then the results of the evaluation of the subformulas are combined by the corresponding boolean connective to get the result for the input formula.
For example, when the formula is a conjunction of two subformulas \lstinline{x} and \lstinline{y}, the results of their execution \lstinline{v} and \lstinline{w} are combined by the call to the relation \lstinline{and$^o$} to compute the result: \lstinline{and$^o$ v w res}.
If the input formula is a variable, then its value is looked up in the substitution list by means of the relation \lstinline{elem$^o$}.

\begin{figure*}[!t]
  \centering
  \begin{minipage}{0.95\textwidth}
    \begin{lstlisting}[label={eval:whole}, caption={Evaluator of propositional formulas}, captionpos=b, frame=tb]
  let rec eval$^o$ s fm res = conde [fresh (x y z v w) (
      ( fm === conj x y /\ eval$^o$ s x v /\  eval$^o$ s y w /\  and$^o$ v w res );
      ( fm === disj x y /\ eval$^o$ s x v /\  eval$^o$ s y w /\  or$^o$   v w res );
      ( fm === neg x    /\ eval$^o$ s x v /\  not$^o$ v res );
      ( fm === var v    /\ elem$^o$ s v res ))]

  let not$^o$  x y = nand$^o$ x x y
  let or$^o$   x y z = nand$^o$ x x xx /\  nand$^o$ y y yy /\ nand$^o$ xx yy z
  let and$^o$ x y z = nand$^o$ x y xy /\   nand$^o$ xy xy z
  let nand$^o$ a b c = conde [
      ( a === false /\ b === false /\ c === true );
      ( a === false /\ b === true  /\ c === true );
      ( a === true  /\ b === false /\ c === true );
      ( a === true  /\ b === true  /\ c === false )]

  let elem$^o$ n s v = conde [ fresh (h t m) (
    ( n === zero /\ s === h : t /\ h === v );
    ( n === succ m /\ s === h : t /\ elem$^o$ m t v ))]


    \end{lstlisting}
  \end{minipage}
\end{figure*}

Consider the goal \lstinline{fresh (s fm) (eval$^o$ s fm true)}.
The partially constructed process graph for this goal is presented in Fig.~\ref{fig:evalTree}.
The following notation is used.
Rectangle nodes contain configurations, while diamond nodes correspond to splitting.
Each configuration contains a goal along with a substitution in angle brackets.
To visually differentiate constructors from variables within goals, we made them bold.
For brevity, we only put a fragment of the substitution computed at each step in the corresponding node.
The leaf node which contains only the substitution and no goal in its configuration is a success node.
The call selected to be unfolded is underlined in a conjunction.
Nodes corresponding to failures are not present within the process graph.
Dashed arrows mark renamings.

\begin{figure}[!t]
  \centering
  \begin{minipage}{0.95\textwidth}
    \begin{tikzpicture}[
  processTree]

  \node(0) {\underline{\rel{eval}{\texttt{fm s \textbf{true}}}}};

  \node(00)[below of=0, xshift=-7.5cm, yshift=0.6cm, anchor=north] {
    \rel{eval}{\texttt{x s a}} \conj \\
    \rel{eval}{\texttt{y s b}} \conj \\
    \underline{\rel{and}{\texttt{a b \textbf{true}}}} \\
    \subst{\texttt{fm} $\to$ \texttt{\textbf{conj} x y}}};

  \node(000)[below of=00, anchor=north] {
    \rel{eval}{\texttt{x s \textbf{true}}} \conj \\
    \rel{eval}{\texttt{y s \textbf{true}}} \\
    \subst{\texttt{a} $\to$ \texttt{\textbf{true},} \\ \texttt{b} $\to$ \texttt{\textbf{true}}}};

  \node(0000)[diamond,below of=000, yshift=0.5cm,anchor=north] { \conj };

  \node(00000)[below of=0000, xshift=-1.7cm, yshift=1cm, anchor=north] {
          \rel{eval}{\texttt{x s \textbf{true}}}};
  \node(00001)[below of=0000, xshift=1.7cm, yshift=1cm, anchor=north] {
          \rel{eval}{\texttt{y s \textbf{true}}}};

  \node(01)[below of=0, yshift=0.6cm, anchor=north] {
    \rel{eval}{\texttt{x s a}} \conj \\
    \rel{eval}{\texttt{y s b}} \conj \\
    \underline{\rel{or}{\texttt{a b \textbf{true}}}} \\
    \subst{\texttt{fm} $\to$ \texttt{\textbf{disj} x y}}};

  \node(010)[below of=01, xshift=-3.6cm, anchor=north] {
      \rel{eval}{\texttt{x s \textbf{true}}} \conj \\
      \rel{eval}{\texttt{y s \textbf{true}}} \\
      \subst{\texttt{a} $\to$ \texttt{\textbf{true},} \\ \texttt{b} $\to$ \texttt{\textbf{true}}}};

  \node(0100)[below of=010, anchor=north, draw=none, fill=none, yshift=0.5cm]{...};
  \node(011)[below of=01, anchor=north] {
      \rel{eval}{\texttt{x s \textbf{true}}} \conj \\
      \rel{eval}{\texttt{y s \textbf{false}}} \\
      \subst{\texttt{a} $\to$ \texttt{\textbf{true},} \\ \texttt{b} $\to$ \texttt{\textbf{false}}}};

  \node(0110)[diamond,below of=011, yshift=0.5cm,anchor=north] { \conj };

  \node(01100)[below of=0110, xshift=-1.7cm, yshift=1cm, anchor=north] {
          \rel{eval}{\texttt{x s \textbf{true}}}};
  \node(01101)[below of=0110, xshift=1.7cm, yshift=1cm, anchor=north] {
          \rel{eval}{\texttt{y s \textbf{false}}}};
  \node(011010)[below of=01101, xshift=-1.5cm, yshift=1.5cm, anchor=north, draw=none, fill=none] {...};
  \node(011011)[below of=01101, xshift=-0.5cm, yshift=1.5cm, anchor=north, draw=none, fill=none] {...};
  \node(011012)[below of=01101, xshift=0.5cm,  yshift=1.5cm, anchor=north, draw=none, fill=none] {...};
  \node(011013)[below of=01101, xshift=1.5cm,  yshift=1.5cm, anchor=north, draw=none, fill=none] {...};
  \node(012)[below of=01, xshift=3.7cm, anchor=north] {
      \rel{eval}{\texttt{x s \textbf{false}}} \conj \\
      \rel{eval}{\texttt{y s \textbf{true}}} \\
      \subst{\texttt{a} $\to$ \texttt{\textbf{false},} \\ \texttt{b} $\to$ \texttt{\textbf{true}}}};

  \node(0120)[below of=012, anchor=north, draw=none, fill=none, yshift=0.5cm]{...};


  \node(02)[below of=0, xshift=7.5cm, yshift=0.6cm, anchor=north] {
    \rel{eval}{\texttt{x s a} \conj \\
    \underline{\rel{not}{\texttt{a \textbf{true}}}} \\
    \subst{\texttt{fm} $\to$ \texttt{\textbf{neg} x}}}};

  \node(020)[below of=02,yshift=-0.23cm, anchor=north] {
    \rel{eval}{\texttt{x s \textbf{false}}} \\
    \subst{\texttt{a} $\to$ \texttt{\textbf{false}}}};

  \node(03)[below of=0, xshift=11.5cm, yshift=0.6cm, anchor=north] {
    \rel{elem}{\texttt{x s \textbf{true}}} \\
    \subst{\texttt{fm} $\to$ \texttt{\textbf{var} v}}
  };

  \node(030)[below of=03, xshift=-1cm, anchor=north, yshift=-0.5cm]{
    \subst{\texttt{s} $\to$ \texttt{h\%t}, \\ \texttt{x} $\to$ \texttt{\textbf{zero}} \\ \texttt{h} $\to$ \texttt{\textbf{true}}}
  };
  \node(031)[below of=03, xshift=2cm, anchor=north, yshift=-0.5cm]{
    \rel{elem}{\texttt{y t \textbf{true}}} \\
    \subst{\texttt{s} $\to$ \texttt{h\%t}, \\ \texttt{x} $\to$ \texttt{\textbf{succ} y}}};


  \draw [->] (0.south west) to (00.north east);
  \draw [->] (0) to (01);
  \draw [->] (0) to (02.north west);
  \draw [->] (0.south east) to (03.north west);
  \draw [->] (00) to (000);
  \draw [->] (000) to (0000);
  \draw [->] (0000) to (00000);
  \draw [->] (0000) to (00001);
  \draw [->] (01) to (010);
  \draw [->] (01) to (011);
  \draw [->] (01) to (012);
  \draw [->] (010) to (0100);
  \draw [->] (011) to (0110);
  \draw [->] (0110) to (01100);
  \draw [->] (0110) to (01101);
  \draw [->] (01101) to (011010);
  \draw [->] (01101) to (011011);
  \draw [->] (01101) to (011012);
  \draw [->] (01101) to (011013);
  \draw [->] (012) to (0120);
  \draw [->] (02) to (020);
  \draw [->] (03) to (030);
  \draw [->] (03) to (031);

  \draw [dashed,<-] ($(0.south west)+(0.1,0)$) .. controls +(-5,-2.5) and +(-4,0) .. (01100.west);
  \draw [dashed,->] (00000.north west) to [out=90,in=180] ($(0.west)$);
  \draw [dashed,->] (00001) to [out=90,in=-135] ($(0.west)+(0,-0.1)$);

  \draw[dashed,->] (020.south) to [out=-90,in=0] (01101.east);
  \draw[dashed,->] (031.north east) to [out=90,in=0] (03.south east);

\end{tikzpicture}
  \end{minipage}
  \caption{Partially constructed process graph for the relation eval$^o$.}
  \label{fig:evalTree}
\end{figure}


\begin{figure}[!t]
  \centering
  \begin{minipage}{0.95\textwidth}
    \begin{tikzpicture}[
  processTree,
  anchor=center,
  node distance=2.5cm]
  \node(0) {\rel{and}{\texttt{x y} \texttt{\textbf{true}}}};
  \node(00) [below of=0, yshift=1cm] {\rel{nand}{\texttt{x y xy} \conj \rel{nand}{\texttt{xy xy \texttt{\textbf{true}}}}}};
  \node(000) [below of=00, xshift=-10cm] {
    \rel{nand}{\texttt{\textbf{true true true}}} \\ \subst{\texttt{x} $\to$ \texttt{\textbf{false}}, \\ \texttt{y} $\to$ \texttt{\textbf{false}}, \\ xy $\to$ \texttt{\textbf{true}}}};
  \node(0000) [below of=000, draw=none] {fail};
  \node(001) [below of=00, xshift=-3cm] {
    \rel{nand}{\texttt{\textbf{true true true}}} \\ \subst{\texttt{x} $\to$ \texttt{\textbf{false}}, \\ \texttt{y} $\to$ \texttt{\textbf{true}}, \\ xy $\to$ \texttt{\textbf{true}}}};
  \node(0010) [below of=001, draw=none] {fail};

  \node(002) [below of=00, xshift=3cm] {
    \rel{nand}{\texttt{\textbf{true true true}}} \\ \subst{\texttt{x} $\to$ \texttt{\textbf{true}}, \\ \texttt{y} $\to$ \texttt{\textbf{false}}, \\ xy $\to$ \texttt{\textbf{true}}}};
  \node(0020) [below of=002, draw=none] {fail};

  \node(003) [below of=00, xshift=10cm] {
    \rel{nand}{\texttt{\textbf{false false true}}} \\ \subst{\texttt{x} $\to$ \texttt{\textbf{true}}, \\ \texttt{y} $\to$ \texttt{\textbf{true}}, \\ xy $\to$ \texttt{\textbf{false}}}};
  \node(0030) [below of=003] { \subst{\texttt{x} $\to$ \texttt{\textbf{true}}, \texttt{y} $\to$ \texttt{\textbf{true}}, xy $\to$ \texttt{\textbf{false}}}};

  \draw[->] (0) to (00);

  \draw[->] (00) to (000);
  \draw[->] (00) to (001);
  \draw[->] (00) to (002);
  \draw[->] (00) to (003);

  \draw[->] (000) to (0000);
  \draw[->] (001) to (0010);
  \draw[->] (002) to (0020);
  \draw[->] (003) to (0030);

\end{tikzpicture}
  \end{minipage}
  \caption{Unfolding of and$^o$.}
  \label{fig:and}
\end{figure}


Conservative partial deduction starts by unfolding the single relation call in this goal once.
Besides introducing fresh variables into the context, unifications in each conjunct are computed into substitutions.
This produces 4 branches, each of which is processed further.

Consider the first branch, in which the input formula \lstinline{fm} is a conjunction of subformulas \lstinline{x} and \lstinline{y}.
First, each relation call within a conjunction is examined separately to select one of the calls to unfold.
To do so, we unfold each of them in isolation and use the less branching heuristic.
Both recursive calls to \lstinline{eval$^o$} are done with three distinct fresh variables, and are not selected according to the less branching heuristic.
By unfolding the call to \lstinline{and$^o$} several times, we determine it has a single non-failing branch (see Fig.~\ref{fig:and}), which is less than the same relation would have if called on all free and distinct variables, thus this call is selected to be unfolded.
The result of unfolding the call is a single substitution which associates the variables \lstinline{a} and \lstinline{b} with \lstinline{true}.
By applying the computed substitution to the goal we get a conjunction of two calls to the \lstinline{eval$^o$} relation with the last argument being \lstinline{true}.
This conjunction embeds the root goal, thus we split the conjunction and both calls become leaves which rename the root.
This finishes the processing of the first branch.

Consider the second branch, in which the input formula \lstinline{fm} is a disjunction of subformulas \lstinline{x} and \lstinline{y}.
Similarly to the first branch, the heuristic selects the call to the boolean relation to be unfolded which produces three possible substitutions for variables \lstinline{a} and \lstinline{b}.
The substitutions are propagated into the goals and three branches, each of which embeds the root goal, are added into the process graph.
Each goal is then split, and the calls with the last argument being \lstinline{true} rename the root.
Finally, the call \lstinline{eval$^o$ y s false} is processed, which creates 4 branches similar to the branches of the root goal.


\begin{figure*}[!t]
  \centering
  \begin{minipage}{0.65\textwidth}
    \begin{lstlisting}[label={eval:conspd}, caption={Specialized evaluator of propositional formulas}, captionpos=b, frame=tb]
  let rec eval$^o_{true}$ s fm = conde [fresh (x y z v w) (
      ( fm === conj x y /\ eval$^o_{true}$ s x /\  eval$^o_{true}$ s y );
      ( fm === disj x y /\ (conde [
          ( eval$^o_{true}$ s x /\    eval$^o_{true}$ s y );
          ( eval$^o_{true}$ s x /\    eval$^o_{false}$ s y );
          ( eval$^o_{false}$ s x /\    eval$^o_{true}$ s y );
      ]);
      ( fm === neg x /\ eval$^o_{false}$ s x v );
      ( fm === var v /\ elem$^o_{true}$ s v ))]

  let rec eval$^o_{false}$ s fm = conde [fresh (x y z v w) (
      ( fm === conj x y /\ (conde [
          ( eval$^o_{false}$ s x /\    eval$^o_{false}$ s y );
          ( eval$^o_{true}$ s x /\    eval$^o_{false}$ s y );
          ( eval$^o_{false}$ s x /\    eval$^o_{true}$ s y );
      ]);
      ( fm === disj x y /\ eval$^o_{true}$ s x /\  eval$^o_{true}$ s y );
      ( fm === neg x /\ eval$^o_{true}$ s x v );
      ( fm === var v /\ elem$^o_{false}$ s v ))]

  let elem$^o_{true}$ n s = conde [ fresh (h t m) (
      ( n === zero /\ s === true : t );
      ( n === succ m /\ s === h : t /\ elem$^o_{true}$ m t ))]

  let elem$^o_{false}$ n s = conde [ fresh (h t m) (
      ( n === zero /\ s === false : t );
      ( n === succ m /\ s === h : t /\ elem$^o_{false}$ m t ))]


    \end{lstlisting}
  \end{minipage}
\end{figure*}

The third branch is driven until a call with the last argument being \lstinline{false} is encountered.
Since it renames one of the nodes which is already present in the process graph, we stop exploring this branch and add the back edge.

The last branch, in which the input formula \lstinline{fm} is a variable \lstinline{v}, contains the single call to the relation \lstinline{elem$^o$}.
The unfolding of this call produces two leaves: a success node and a renaming of the parent node.
This finishes the construction of the process graph.

The process graph is then residualized into a specialized version of the \lstinline{eval$^o$} relation (see Listing~\ref{eval:conspd}).
This program does not contain any calls to boolean connectives.
Neither does the program contain the original, not specialized, relation \lstinline{eval$^o$}.

It is worth noting that the result produced by the Conservative Partial Deduction is not ideal.
For example, in the definition of the \lstinline{eval$^o_{true}$}, when the input formula \lstinline{fm} is a disjunction of subformulas \lstinline{x} and \lstinline{y}, the recursive call \lstinline{eval$^o_{true}$ s x } is done twice in two disjuncts.
The ideal version of the relation \lstinline{eval$^o_{true}$} should contain this recursive call only once.
This can be done, for example, by common subexpression elimination~\cite{muchnick1997advanced}.
However, ideally, \lstinline{y} should not be evaluted at all, since the value of the formula \lstinline{fm} does not depend on it.
It is unclear if and how this kind of transformation can be done automatically.
Such transformation would require, first, realising that a disjunction of two relation calls \lstinline{eval$^o_{true}$ s y} and \lstinline{eval$^o_{false}$ s y} exhaust all possible values for \lstinline{y}.
Secondly, the transformation would have to examine if a relation restricts values of a given argument regardless of the other arguments' values.
