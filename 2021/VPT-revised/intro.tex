\section{Introduction}
\label{intro}

A family of embedded domain-specific languages \mk\footnote{\mk language web site: \url{http://minikanren.org}. Access date: 28.02.2021} implement relational programming~---~a paradigm closely related to the pure logic programming.
The minimal core of the language, also known as \muk, can be implemented in as little as 39 lines of \scheme~\cite{friedmanmukanren}.
An introduction to the language and some of its extensions in a series of examples can be found in the book The Reasoned Schemer~\cite{TheReasonedSchemer}.
The formal certified semantics for \mk is described in~\cite{rozplokhas2020certified}.

Relational programming is a paradigm based on the idea of describing programs as relations.
The core feature of relational programming is the ability to run a program in various directions by executing goals with free variables.
The distribution of free variable occurrences determines the direction of relational search.
For example, having specified a relation for adding two numbers, one can also compute the subtraction of two numbers or find all pairs of numbers which can be summed up to get the given one.
One of the most prominent applications of relational programming amounts to implementing interpreters as relations.
By running a relational interpreter for some language \emph{backwards} one can do program synthesis.
In general, it is possible to create a solver from a recognizer by translating it into \mk and running it in the appropriate direction~\cite{lozov2019relational}.

The search employed in \mk is complete which means that every answer will be found, although it may take a long time.
The promise of \mk falls short when speaking of performance.
The execution time of a program in \mk is highly unpredictable and varies greatly for various directions.
What is even worse, it depends on the order of the relation calls within a program.
One order can be good for one direction, but slow down the computation dramatically in the other direction.

Partial evaluation~\cite{jonesbook} is a technique for specialization, i.e. improving the performance of a program given some information about it beforehand.
It may either be a known value of some argument, its structure (e.g. the length of an input list) or, in case of a relational program, --- the direction in which the relation is intended to be run.
An earlier paper~\cite{lozov2019relational} has shown that \emph{conjunctive partial deduction}~\cite{de1999conjunctive} can sometimes improve the performance of \mk programs.
Depending on the particular \emph{control} decisions, it may also not affect the execution time of a program or even make it slower.

Control issues in partial deduction of logic programming language \pro have been studied before~\cite{leuschel2002logic}.
Unlike Prolog, where atoms in the right-hand side of a clause cannot be arbitrarily reordered without changing the semantics of a program, in \mk the subgoals of conjunction/disjunction can be freely switched.
This opens yet another possibility for optimization, not taken into account by approaches initially developed in the context of conventional logic programming.
% The ideas described there make use of the deterministic search strategy of \pro.
% However in \mk it is allowed to reorder some relation calls within the goal for better performance.
% While sometimes conjunctive partial deduction gives great performance boost, sometimes it does not behave as well as it could have.

In this paper, we study issues which conjunctive partial deduction faces being applied for \mk.
We also describe a novel approach to partial deduction for relational programming, \emph{conservative partial deduction}.
We implemented this approach and compared it with the existing specialization system (\ecce) for several programs.
We report here the results of the comparison and discuss why some \mk programs run slower after specialization.
