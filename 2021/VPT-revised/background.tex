\section{Background}

In this section we provide some background on relational programming and relational interpreters.

\subsection{\mk}

\begin{figure*}[t]
  \[
  \begin{array}{cccll}
    &\mathcal{T} & = & \mathcal{X} \cup \{C_i^{k_i} (t_1, \dots, t_{k_i}) \mid t_j\in\mathcal{T}\} & \mbox{terms over the set of variables $\mathcal{X}$} \\
    &\mathcal{G} & = & \mathcal{T}\equiv\mathcal{T}   &  \mbox{unification} \\
    &            &   & \mathcal{G}\wedge\mathcal{G}     & \mbox{conjunction} \\
    &            &   & \mathcal{G}\vee\mathcal{G}       &\mbox{disjunction} \\
    &            &   & \mbox{\lstinline|fresh|}\;\mathcal{X}\;.\;\mathcal{G} & \mbox{fresh variable introduction} \\
    &            &   & R_i^{k_i} (t_1,\dots,t_{k_i}),\;t_j\in\mathcal{T} & \mbox{relational symbol invocation} \\
    &\mathcal{S} & = & \{R_i^{k_i} = \lambda\;x_1^i\dots x_{k_i}^i\,.\, g_i;\}\; g & \mbox{specification}
  \end{array}
  \]
  \caption{The syntax of the source language}
  \label{syntax}
  \end{figure*}



This paper considers the minimal relational core of the \mk language.
The syntax of the language is presented in Fig.~\ref{syntax}.
A specification of the \mk program consists of a set of \emph{relation definitions} accompanied by a top-level \emph{goal} which plays a role of a query.
Goal, being the central syntactic category of the language, can take the form of either a term \emph{unification}, \emph{conjunction} or \emph{disjunction} of goals, a \emph{fresh} syntactic variable introduction, or a relation \emph{call}.
We consider the alphabet of constructors $\{C^{k_i}_i\}$ and relational symbols $\{R^{k_i}_i\}$ to be predefined and accompanied with their arities.

The formal semantics of the language is best described in~\cite{rozplokhas2020certified}.
Here we only briefly introduce the semantics.
A stream of substitutions for free variables within the query goal is computed during the execution of a \mk program.
Depending on the kind of the goal, one of the following situations is possible.

\begin{enumerate}
  \item Term unification $t_1 \equiv t_2$ computes the most-general unification in the context of the current substitution. If succeeded, the unifier is added into the current substitution and then it is returned as a singleton stream. Otherwise, an empty stream is returned.
  \item Introduction of the fresh variable $fresh \ x. g$ allocates a new \emph{semantic} variable, substitutes it for all fresh occurrences of $x$ within $g$, then evaluates the goal.
  \item An execution of a relational call $R^{k_i}_i(t_1, \dots, t_{k_i})$ is done by first substuting the terms $t_j$ for the respective formal parameters and then running the resulting goal.
  \item When executing a conjunction $g_1 \wedge g_2$, first the goal $g_1$ is run in the context of the current substitution which results in the stream of substitutions, in each of which $g_2$ is run. The resulting stream of streams is then concatenated.
  \item Disjunction $g_1 \vee g_2$ applies both goals to the current substitution and concatenates the results.
\end{enumerate}

Consider the relation \lstinline{add$^o$} in Listing~\ref{eval:arith}.
It defines the relation between three Peano numbers $x$, $y$ and $z$, such that $x + y = z$, using the \oc language\footnote{\oc: statically typed \mk embedding in \ocaml. The repository of the project: \url{https://github.com/JetBrains-Research/OCanren}. Access date: 28.02.2021}.
The keyword \lstinline{conde} provides syntactic sugar for a disjunction, while \lstinline{zero} and \lstinline{succ} are constructors.
The query \lstinline{fresh (z) (add$^o$ (succ zero) (succ zero) z)} results in the only substitution \lstinline{[z $\mapsto$ succ (succ zero)]}, while the query \lstinline{fresh (x y) (add$^o$ x y (succ (succ zero)))} executes to three valid substitutions: \lstinline{[x $\mapsto$ zero, y $\mapsto$ succ (succ zero)]}, \lstinline{[x $\mapsto$ succ zero,} \lstinline{y $\mapsto$ succ zero]}, \lstinline{[x $\mapsto$ succ (succ zero), y $\mapsto$ zero]}.

The \emph{interleaving} search~\cite{10.1145/1090189.1086390} is at the core of \mk.
It evaluates disjuncts incrementally, passing control from one to the other.
This search strategy is what makes the search in \mk complete.
It also allows for reordering of both disjuncts and conjuncts within a goal which may improve the efficiency of a program.
This reordering generally leads to the reordering of the answers computed by a \mk program.
The denotational semantics of \mk ignores the order of the answers because the search is complete and thus all possible answers will be found eventually.

\begin{figure*}[!t]
  \centering
  \begin{minipage}{0.68\textwidth}
    \begin{lstlisting}[label={eval:arith}, caption={Evaluator of arithmetic expressions}, captionpos=b, frame=tb]
  let rec add$^o$ x y z = conde [
      (x === zero /\ y === z);
      (fresh (p) (x === succ p /\ add$^o$ p (succ y) z) ) ]

  let rec eval$^o$ fm res = conde [fresh (x y xr yr) (
      (fm === num res);
      (eval$^o$ x xr /\  eval$^o$ y yr /\
        conde [
          (fm === sum x y /\ add$^o$ xr yr res);
          (fm === prod x y /\ $\dots$);
          $\dots$ ] )
    \end{lstlisting}
  \end{minipage}
\end{figure*}


\subsection{Relational Interpreters}

The kind of relational programs most interesting to us is relational interpreters.
They may be used to solve complex problems such as generating quines~\cite{byrd2012minikanren} or to solve search problems by only implementing programs which check that a solution is correct~\cite{lozov2019relational}.
The latter application is the focus of our research project thus we provide a brief description of it.

Search problems are notoriously complicated.
In fact, they are much more complex than verification~---~checking that some candidate solution is indeed a solution.
The ability of \mk programs to be evaluated in different directions along with the complete semantics of the language allows for automatic generation of a solver from a verifier using relational conversion~\cite{lozov2017typed}.
Unfortunately, generated relational interpreters are often inefficient, since the conversion introduces a lot of extra unifications and boilerplate.
This kind of inefficiency is a prime candidate for specialization.

Consider the relational interpreter \lstinline{eval$^o$ fm res} in Listing~\ref{eval:arith}.
It evaluates an arithmetic expression \lstinline{fm} which can take the form of a number (\lstinline{num res}) or a binary expression such as the \lstinline{sum x y} or \lstinline{prod x y}.
Running the interpreter backwards synthesizes expressions which evaluate to the given number. For example one possible answer to the query \lstinline{eval$^o$ fm (succ (succ zero))} is \lstinline{sum (num (succ zero)) (sum (num zero) (num (succ zero)))}.


