
\subsection*{Review 2}


This revised version of the paper has been improved a lot from the previous version, with a lot more
detailed explanations and background, and has taken a lot of the previous referees comments on board.
The paper is a lot easier to read, although many typos remain (see below).

The paper is still very weak, and a lot of fundamental flaws remain. The order of evaluation that is
chosen for the transformation is rather strange and does not seem suited to obtaining the best possible
results. A relation is selected for unfolding if it can lead to the discovery of contradictions between conjuncts;
no details of the algorithm for this are given. It is not clear why this would be a good idea, and what benefits
it provides. The goal of the unfolding should be to control the size of the term, not the size of the answer set.

When a relation is selected for unfolding, it is fully processed in isolation. This also seems strange. This is
in fact following a more depth-first evaluation strategy than the original CPD, while miniKanren itself follows
a breadth-first evaluation strategy, so it seems even more inappropriate for this language.

\emph{Due to the use of interleaving, the search in \mk is complete. Thus the order in which subgoals are evaluated is irrelevant: different evaluation strategies influence the order of the answers, not which answers can be potentially computed. \mk does not follow a breadth-first evaluation strategy, it interleaves partially computed lazy streams of answers and thus makes progress exploring the search space both top-to-bottom and left-to-right. }

This therefore means
that the conservative approach is not able to perform the deforestation or tupling transformations, which is a
major problem.

\emph{This is a major problem indeed. We are currently working on mitigating it.}

It is claimed that the original CPD will not be able to apply deforestation for doubleAppend if
one extra disjunct is added into the appendo relation, or to apply tupling for maxLength if the calls to maxo and
lengtho are reordered; I do not believe that either of these is the case.

\emph{
  We put the prolog programs for maxlength and double appendo in gists. Each file contains an input program in prolog, a goal to specialize, and the result of transformation done by ECCE with the default control settings. We used the instance of ECCE available online here: \url{http://wyvern.cs.uni-duesseldorf.de/ecce/index.php}.
}
\emph{
\begin{itemize}
  \item Original maxLength: \url{https://gist.github.com/kajigor/58f79b0e5847ad860b0af12d98e2c9e2}.
  \item maxLength with the reordered max and len atoms: \url{https://gist.github.com/kajigor/dc348c05ba0a973c6b3e8cdb671ce73b}.
  \item Original doubleAppend: \url{https://gist.github.com/kajigor/0f878c6843e4954ab628f94edfe574cd}.
  \item Modified doubleAppend for which ECCE performs deforestation: \url{https://gist.github.com/kajigor/a30f21604d8ebc91923135fe5026a83c}.
  \item Modified doubleAppend for which ECCE fails to perform deforestation: \url{https://gist.github.com/kajigor/2d9dfbaf8ff6134d97892dcf3bd9bc1e}.
\end{itemize}
}

The evaluation of the proposed transformation scheme is still inadequate. Only two examples are considered (with variants)
and the results are compared with CPD for Prolog using ECCE, so it does not seem fair. If the examples included opportunites
for deforestation or tupling, then presumably ECCE would have performed considerable better in these cases.

\emph{Section~\ref{discussion} states that our method cannot perform deforestation and tupling for some programs, and ConsPD will not improve them.}

I will argue for the weak acceptance of this paper since it makes the following contributions: it is the first partial
evaluator for miniKanren and it makes it clear that the described approach should not be used for this purpose.

Detailed Comments:

\begin{itemize}
  \item{Abstract: at the early stages $\rightarrow$ at an early stage}
  \item{Section 1, para 1: related to the pure logic programming $\rightarrow$ related to pure logic programming}
  \item{Section 1, para 4:  — the direction in which $\rightarrow$ the direction in which}
  \item{Section 1, para 5: of logic programming language $\rightarrow$ of the logic programming language}
  \item{Section 2.1, para 1: plays a role of a query $\rightarrow$ plays the role of a query}
  \item{Section 2.1, para 1: Goal, being the central $\rightarrow$ Goals, being the central}
  \item{Section 2.1, item 1: If succeeded $\rightarrow$ If it succeeds}
  \item{Section 3, para 2: configurations which represent computation state $\rightarrow$ configurations which represent computation states}
  \item{Section 3, final para: once for one local control tree $\rightarrow$ once for each local control tree}
  \item{Figure 2: drive_conj has an inconsistent number of arguments}
  \item{Section 4, para 1: relational programs specialization $\rightarrow$ relational program specialization}
  \item{Section 5, para 4: The leaf node which contain $\rightarrow$ The leaf node which contains}
  \item{Figure 4 label: Partially constructed process tree $\rightarrow$ Partially constructed process graph}
  \item{Page 11, para 3: contain the single call $\rightarrow$ containing the single call}
  \item{Section 6, para 3: in the section 5 $\rightarrow$ in Section 5}
  \item{Section 6, para 1: for Prolog programming language $\rightarrow$ for the Prolog programming language}
  \item{Section 6, para 1: programs, written in miniKanren $\rightarrow$ programs written in miniKanren}
  \item{Section 6.1, para 1: the way the evalo is implemented $\rightarrow$ the way evalo is implemented}
  \item{Section 6.1.2, para 2: which is, in turn, has a $\rightarrow$ which, in turn, has a}
  \item{Section 6.1.3, para 1: summed up in the Table 1 $\rightarrow$ summarised in Table 1}
  \item{Section 6.1.3, para 2: ConsPD do not worsen performance $\rightarrow$ ConsPD does not worsen performance}
  \item{Section 6.2, para 2: Only terms of type integer can be summed up, multiplied or compared by less-than operator $\rightarrow$ Only terms of type integer can be added, multiplied or compared by the less-than operator.}
  \item{Section 6.2, para 2: If-then-else expression typechecks only if $\rightarrow$ The if-then-else expression typechecks only if}
  \item{Section 6.2, para 4: than the hand-written $\rightarrow$ than the hand-written one}
  \item{Section 6.2, para 5: performing average amount $\rightarrow$ performing an average amount}
  \item{Section 6.3, para 4: The similar thing happens $\rightarrow$ A similar thing happens}
  \item{Section 6.3, para 4: splits too much too early $\rightarrow$ splits much too early}
  \item{Section 6.3, para 5: This is an evidence of $\rightarrow$ This is evidence of}
  \item{Section 7: techniques for relational programming language MINIKANREN $\rightarrow$ techniques for the relational programming language MINIKANREN}
  \item{Acknowledgements: fruitful discussiona $\rightarrow$ fruitful discussions}
\end{itemize}