\section{Conclusion}

In this paper we discussed some issues which arise in the area of partial deduction techniques for relational programming language \mk.
We presented a novel approach to partial deduction --- conservative partial deduction --- which uses a heuristic to select a suitable relation call to unfold at each step of driving.
We compared this approach with the most sophisticated implementation of conjunctive partial deduction --- \ecce partial deduction system --- on 6 relations which solve 2 different problems.

Our specializer improved the execution time of all queries.
\ecce worsened the performance of all 4 implementations of the propositional evaluator relation, while improving the other queries.
Conservative partial deduction is more stable with regards to the order of relation calls than \ecce which is demonstrated by the similar performance of all 4 implementations of the evaluator of logic formulas.

Some queries to the same relation were improved more by ConsPD, while others --- by \ecce.
We conclude that there is still no one good technique which definitively speeds up every relational program.
More research is needed to develop models capable of predicting the performance of a relation which can be used in specialization.
Another direction for future research is exploring how specialization influences the execution order of a \mk program.