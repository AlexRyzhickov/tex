\documentclass[12pt]{article}
\usepackage[left=2cm,right=2cm,top=2cm,bottom=2cm,bindingoffset=0cm]{geometry}
\usepackage{hyperref}
\usepackage{fontspec}
\usepackage{polyglossia}
\setdefaultlanguage{russian}
\setmainfont[Mapping=tex-text]{CMU Serif}

\begin{document}
%% Весь этот текст можно удалить
%% ====== от сих =====
\centering {\LARGE Формальные языки}

{\Large домашнее задание до 23:59 05.10}
\bigskip

\begin{enumerate}
  \item 
  { Докажите следующие равенства. Можно воспользоваться свойствами или построить автоматы по исходным выражениям, детерминизировать и минимизировать их, а потом по ним построить обратно регулярные выражения --- можно вручную, можно с помощью любимого программного средства.  
    \begin{enumerate} 
      \item { $ (a | b)^{∗} ab (a | b)^{∗} | (a | b)^{∗} a | b^{∗} = (a | b)^{*} $ }
      \item { $ (a | b)^{*} (ab | ba) (a | b)^{∗} | a^{∗} | b^{∗} = (a | b)^{*}$ }
    \end{enumerate}
  }
  
  \item
  { Составьте регулярные выражения для следующих языков (подвыражения можно называть и использовать названия): 
    \begin{enumerate}
      \item
      { Кортежи, где элементами могут быть целые числа или идентификаторы. Пробелы не являются значимыми символами: между скобками и элементами, элементами и разделителями может быть произвольное число пробелов (символ пробела обозначайте как $\backslash s$). Используйте круглые скобки и запятые как разделители. 
        \begin{itemize} 
          \item \verb!(); (1); (1,1,2,3,5,8); (   ); ( 4, eight,   15,      42     )! --- корректные кортежи
          \item \verb!)(; (()); (1,2,3; (1;2;3); (1,2,); (1, 23  4, 5)! --- некорректные кортежи
        \end{itemize}
      }
      \item
      { Кортежи, где элементами могут быть целые числа или кортежи. 
      \begin{itemize} 
          \item \verb!(1, (), 23); ((1, 3), (4, (5,7), 6)); (13, 42)! --- корректные кортежи
      \end{itemize}
      }
    \end{enumerate}
  }

  \item
  {
    Проверить, является ли язык палиндромов $\{ \omega \omega^R \, | \, \omega \in \{ 0, 1\}^*  \} $ регулярным. Если является --- представить регулярное выражение, конечный автомат или регулярную грамматику, описывающую этот язык. Если нет --- привести доказательство.
  }

  \item 
  {
    Привести регулярные грамматики для языков идентификаторов и ключевых слов из предыдущего домашнего задания. Построить деревья вывода для двух нетривиальных цепочек из каждого языка
  }
\end{enumerate}

%% ===== и до сих =====
\end{document}
