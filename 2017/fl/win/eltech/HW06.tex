\documentclass[12pt]{article}
\usepackage[left=1.5cm,right=1.5cm,top=2cm,bottom=2cm,bindingoffset=0cm]{geometry}
\usepackage{hyperref}
\usepackage{fontspec}
\usepackage{polyglossia}
\usepackage{amssymb}
\setdefaultlanguage{russian}
\setmainfont[Mapping=tex-text]{CMU Serif}

\begin{document}
%% Весь этот текст можно удалить
%% ====== от сих =====
\begin{center}
{\LARGE Формальные языки}


{\Large домашнее задание до 23:59 09.11}
\end{center}

\bigskip

В этом задании отрабатываем преобразования КС грамматик и применяем алгоритмы синтаксического анализа. Каждое из заданий можно выполнить ``вручную'' или написать программу. Во втором случае ваш код должен удовлетворять требованиям ниже. 

\begin{enumerate}
  \item 
  {
    Привести в Нормальную Форму Хомского грамматику арифметических выражений (грамматика ниже).
    \begin{itemize}
      \item Добавьте в отчет промежуточные результаты преобразования на каждом шаге. 
    \end{itemize}
  }
  \item 
  {  
    Осуществить синтаксический анализ алгоритмом CYK для 2 арифметических выражений, содержащих не меньше 7 терминалов: одно выражение должно быть корректным арифметическим, другое --- нет. Для корректного выражения приведите дерево вывода. Используйте грамматику, полученную в прошлом задании.
    \begin{itemize}
      \item Добавьте в отчет таблицы, которые строит алгоритм CYK. Не забывайте делать вывод о выводимости цепочки. 
    \end{itemize}
  }
  \item 
  { 
    Построить LL(1) таблицу по грамматике арифметических выражений (грамматика ниже). 
    \begin{itemize}
      \item Не забываем про то, что нужно избавляться от левой рекурсии и делать левую факторизацию грамматики. Если в одной ячейке таблицы оказалось больше одной ячейки, значит, вы не преобразовали грамматику к подходящей форме.
      \item Добавьте в отчет таблицу LL(1), а также посчитанные для нетерминалов множества FIRST и FOLLOW. 
    \end{itemize}
  }
  \item
  {
    Осуществить синтаксический анализ алгоритмом LL(1) для 2 арифметических выражений, содержащих не меньше 7 терминалов: одно выражение должно быть корректным арифметическим, другое --- нет. Для корректного выражения приведите дерево вывода. Используйте грамматику, полученную в прошлом задании. 
    \begin{itemize}
      \item Добавьте в отчет "историю" стека --- то, в каком порядке в стек помещались символы грамматики (как мы делали на паре). 
    \end{itemize}
  }
\end{enumerate}

\bigskip

\begin{center} Грамматика арифметических выражений \end{center}

    $$
    \begin{array}{cccccccccc}
       &E & \rightarrow & E - T & \mid & T \\
       &T & \rightarrow & T * F & \mid & F \\
       &F & \rightarrow & ( E ) & \mid & D \\
       &D & \rightarrow & 0     & \mid & 1 & \mid & \dots & \mid & 9
    \end{array}
    $$

\newpage

\begin{center} Требования к коду \end{center}
\begin{itemize}
\item Ваша программа должна быть консольным приложением, принимающим на вход путь к файлу, содержащим грамматику, (и/или --- в зависимости от задачи) путь к файлу, содержащим входную строку. 
\item Результатом работы программы должно быть то, что требуется в задании --- со всеми промежуточными шагами. 
\item Код должен быть размещен на гитхабе, содержать инструкцию по сборке и запуску собранного приложения, 
\item Код должен собираться одним скриптом на чистой Ubuntu 16.04 или Windows 10. Все зависимости, в случае их отсутствия в системе, должны либо доставляться скриптом, либо перечисляться в явном виде.
\end{itemize}

\bigskip

\begin{center} Заготовка для LL(1) таблицы

(чтобы с версткой не мучиться)
\end{center}
\begin{center}
\begin{tabular}{ l || c | c || c | c | r }
  N & FIRST & FOLLOW &  &  & $ \$ $ \\ \hline  
    &       &        &  &  & \\ 
    &       &        &  &  & 

\end{tabular}  
\end{center}


%% ===== и до сих =====
\end{document}
