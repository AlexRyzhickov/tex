\documentclass{article}

\usepackage[left=2cm,right=2cm,top=2cm,bottom=2cm,bindingoffset=0cm]{geometry}
\usepackage{listings}
\usepackage{indentfirst}
\usepackage{verbatim}
\usepackage{amsmath, amsthm, amssymb}
\usepackage{stmaryrd}
\usepackage{graphicx}
\usepackage{euscript}

\usepackage[utf8]{inputenc}
\usepackage[english,russian]{babel}
\usepackage[T2A]{fontenc}

\lstdefinelanguage{llang}{
keywords={skip, do, while, read, write, if, then, else},
sensitive=true,
%%basicstyle=\small,
commentstyle=\scriptsize\rmfamily,
keywordstyle=\ttfamily\underbar,
identifierstyle=\ttfamily,
basewidth={0.5em,0.5em},
columns=fixed,
fontadjust=true,
literate={->}{{$\to$}}1
}

\lstset{
language=llang
}

\newcommand{\aset}[1]{\left\{{#1}\right\}}
\newcommand{\term}[1]{\mbox{\texttt{\bf{#1}}}}
\newcommand{\cd}[1]{\mbox{\texttt{#1}}}
\newcommand{\sembr}[1]{\llbracket{#1}\rrbracket}
\newcommand{\conf}[1]{\left<{#1}\right>}
\newcommand{\fancy}[1]{{\cal{#1}}}

\newcommand{\trule}[2]{\frac{#1}{#2}}
\newcommand{\crule}[3]{\frac{#1}{#2},\;{#3}}
\newcommand{\withenv}[2]{{#1}\vdash{#2}}
\newcommand{\trans}[3]{{#1}\xrightarrow{#2}{#3}}
\newcommand{\ctrans}[4]{{#1}\xrightarrow{#2}{#3},\;{#4}}
\newcommand{\llang}[1]{\mbox{\lstinline[mathescape]|#1|}}
\newcommand{\pair}[2]{\inbr{{#1}\mid{#2}}}
\newcommand{\inbr}[1]{\left<{#1}\right>}
\newcommand{\highlight}[1]{\color{red}{#1}}
\newcommand{\ruleno}[1]{\eqno[\textsc{#1}]}
\newcommand{\inmath}[1]{\mbox{$#1$}}
\newcommand{\lfp}[1]{fix_{#1}}
\newcommand{\gfp}[1]{Fix_{#1}}

\newcommand{\NN}{\mathbb N}
\newcommand{\ZZ}{\mathbb Z}

\begin{document}

%% Весь этот текст можно удалить
%% ====== от сих =====
\centering {\LARGE Формальные языки}

{\Large домашнее задание до 23:59 10.12}
\bigskip
\begin{enumerate}
  \item Адаптировать конкретный синтаксис под новые возможности абстрактного синтаксиса (1 балл)
  \begin{itemize}
    \item Если у вас какие-то из этих фич были реализованы, ничего страшного. Убедитесь, что ваши решения не ломаются с таким конкретным синтаксисом. 
  \end{itemize}
  \item Адаптировать парсер для нового конкретного синтаксиса языка L. Не забыть про тесты (3 балла)
    \begin{itemize}
        \item Синтаксический анализатор все еще должно быть возможно запустить как консольное приложение, требования такие же, как в прошлом задании.
     \end{itemize}
\end{enumerate}

\bigskip

\centering {\Large Абстрактный синтаксис языка L }
$$
X \mbox{ --- счетно-бесконечное множество переменных}
$$
$$
\otimes=\{\llang{+}, \llang{-}, \llang{*}, \llang{/}, \llang{\%}, \llang{==}, \llang{!=}, 
\llang{>}, \llang{>=}, \llang{<}, \llang{<=}, \llang{\&\&}, \llang{||}\}
$$

\begin{itemize}

\item Определения (функций): $\fancy{D}=\fancy{X}_{name} \; \fancy{X}_0 \dots \fancy{X}_k \leftarrow \fancy{S}$. $\fancy{X}_{name}$ --- имя функции;  $\fancy{X}_0 \dots \fancy{X}_k$ --- ее аргументы; $\fancy{S}$~--- тело.
\item Вызовы функций: $\fancy{C} = \fancy{X}_{name} \; \fancy{E}_0 \dots \fancy{E}_k$. \emph{Аргументами могут быть произвольные выражения.}
\item Выражения: $\fancy{E}=\fancy{C} \cup X\cup\NN\cup(\fancy{E}\otimes\fancy{E})$. \emph{Вызовы функций теперь могут быть использованы в выражениях.} В выражениях могут использоваться круглые скобки.


\item Операторы: 

$$
\begin{array}{rll}
  \fancy{S}=
            &\fancy{X}\;\llang{:=}\;\fancy{E}&\cup\\
            &\fancy{C}&\cup\\
            &\llang{write}\;\fancy{E}&\cup\\
            &\llang{read}\;\fancy{X}&\cup\\
            &\llang{while}\;\fancy{E}\;\llang{do}\;\fancy{S}&\cup\\
            &\llang{if}\;\fancy{E}\;\llang{then}\;\fancy{S}\;\llang{else}\;\fancy{S}&\cup\\
            &\llang{if}\;\fancy{E}\;\llang{then}\;\fancy{S}&\cup\\
            &\fancy{S}^*
            
\end{array}
$$
\item Программы: $\fancy{P}=(\fancy{D}^*, \fancy{S})$
\end{itemize}


\end{document}

