\documentclass[12pt]{article}
\usepackage[left=2cm,right=2cm,top=2cm,bottom=2cm,bindingoffset=0cm]{geometry}
\usepackage{fontspec}
\usepackage{polyglossia}
\setdefaultlanguage{russian}
\setmainfont[Mapping=tex-text]{CMU Serif}

\begin{document}
\begin{center} 
{\LARGE Формальные языки}

{\Large домашнее задание до 23:59 3.12}
\end{center}
\bigskip

Цель задания --- убедиться в понимании того, как работают LR-анализаторы. Можно выполнять ``вручную'', можно написать кусочек кода. Задание для каждой из грамматик сделать следующее: 
    
    \begin{itemize}
        \item Построить LR(0) автомат и LR(0) таблицу.
        \item Если не удалось, построить SLR(1) таблицу для той же грамматики.
        \item Если не удалось, построить LR(1) автомат и таблицу для той же грамматики.
        \item Если не удалось, подумать и написать, почему так вышло. 
        \item Если какую-нибудь таблицу построить все-таки удалось, промоделировать с ней разбор одной корректной и одной некорректной строки: предоставить историю изменения стека и дерево разбора.
    \end{itemize} 

Грамматики:

\begin{enumerate}
  \item $S \rightarrow  S ( S ) \, | \, \varepsilon$
  \item $ S \rightarrow  ( S ) \, | \, S S  \, | \, \varepsilon$
  \item Однозначная грамматика для арифметики (с операциями $-$ и $*$ с ествественными приоритетами и ассоциативностью)
  \item Грамматика для $\{ a^n b c^n \mid n \geq 1 \} \cup \{ a^n d c^{2n} \mid n \geq 1\}$ 
\end{enumerate}

  Автомат можно не рисовать в виде графа: достаточно указать, из каких LR-item-ов состоят состояния, и предоставить таблицу переходов между состояниями. 
\end{document}