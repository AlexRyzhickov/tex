\documentclass{article}

\usepackage[left=2cm,right=2cm,top=2cm,bottom=2cm,bindingoffset=0cm]{geometry}
\usepackage{listings}
\usepackage{indentfirst}
\usepackage{verbatim}
\usepackage{amsmath, amsthm, amssymb}
\usepackage{stmaryrd}
\usepackage{graphicx}
\usepackage{euscript}

\usepackage[utf8]{inputenc}
\usepackage[english,russian]{babel}
\usepackage[T2A]{fontenc}

\lstdefinelanguage{llang}{
keywords={skip, do, while, read, write, if, then, else, begin, end},
sensitive=true,
%%basicstyle=\small,
commentstyle=\scriptsize\rmfamily,
keywordstyle=\ttfamily\underbar,
identifierstyle=\ttfamily,
basewidth={0.5em,0.5em},
columns=fixed,
fontadjust=true,
literate={->}{{$\to$}}1
}

\lstset{
language=llang
}

\newcommand{\aset}[1]{\left\{{#1}\right\}}
\newcommand{\term}[1]{\mbox{\texttt{\bf{#1}}}}
\newcommand{\cd}[1]{\mbox{\texttt{#1}}}
\newcommand{\sembr}[1]{\llbracket{#1}\rrbracket}
\newcommand{\conf}[1]{\left<{#1}\right>}
\newcommand{\fancy}[1]{{\cal{#1}}}

\newcommand{\trule}[2]{\frac{#1}{#2}}
\newcommand{\crule}[3]{\frac{#1}{#2},\;{#3}}
\newcommand{\withenv}[2]{{#1}\vdash{#2}}
\newcommand{\trans}[3]{{#1}\xrightarrow{#2}{#3}}
\newcommand{\ctrans}[4]{{#1}\xrightarrow{#2}{#3},\;{#4}}
\newcommand{\llang}[1]{\mbox{\lstinline[mathescape]|#1|}}
\newcommand{\pair}[2]{\inbr{{#1}\mid{#2}}}
\newcommand{\inbr}[1]{\left<{#1}\right>}
\newcommand{\highlight}[1]{\color{red}{#1}}
\newcommand{\ruleno}[1]{\eqno[\textsc{#1}]}
\newcommand{\inmath}[1]{\mbox{$#1$}}
\newcommand{\lfp}[1]{fix_{#1}}
\newcommand{\gfp}[1]{Fix_{#1}}

\newcommand{\NN}{\mathbb N}
\newcommand{\ZZ}{\mathbb Z}

\begin{document}

%% Весь этот текст можно удалить
%% ====== от сих =====
\begin{center} {\LARGE Формальные языки} \end{center}

\begin{center} {\Large домашнее задание до 23:59 23.03} \end{center}
\bigskip

\begin{enumerate}
  \item Написать лексер для языка L (спецификация ниже), используя любимый генератор лексеров из семейства Lex (или любой другой понравившийся инструмент, только не парсер-комбинаторы, и не пишите лексер вручную). (8 баллов за полностью выполненное задание)
    \begin{itemize}
        \item Структуры данных для лексем должны однозначно их идентифицировать, а также содержать привязку к коду (в какой строке исходного кода и с какого по какой символ располагается лексема). 
        \item Составить набор тестов, демонстрирующий правильность работы полученного лексера (качество тестового покрытия важно!).
        \item Сделать консольное  приложение, принимающее на вход путь к файлу, содержащему программу на языке L, производящее лексический анализ и печатающее полученный поток лексем.
        \begin{itemize}
            \item Результатом работы лексического анализатора должен быть поток лексем, который печатается при помощи отдельной процедуры печати. 
        \end{itemize}
        \item Код должен быть размещен на гитхабе, собираться одним скриптом, содержать инструкцию по сборке и запуску собранного приложения, собираться на чистой Ubuntu 16.04 или Windows 10. Все зависимости, в случае их отсутствия в системе, должны либо доставляться скриптом, либо перечисляться в явном виде. 
     \end{itemize}
\end{enumerate}

\bigskip

\begin{center} {\Large Лексическая спецификация языка L } \end{center}

\medskip

Программы на языке L записываются символами ASCII. Ограничителями строк являются ASCII-символы CR, LF или 2 подряд идущих символа: CR LF. Пробельными символами являются символы-ограничители строк и символы пробела (SP), табуляции (HT) и перевода страницы (FF). Пробельные символы не имеют значения, для них не должно генерироваться лексем. 

Комментарии только однострочные. Комментарием считается суффикс строки до символа-ограничителя строк, начинающийся с "//".

Идентификаторы в языке --- произвольная последовательность символов (до первого пробельного символа), начинающаяся либо с буквы, либо символа подчеркивания (\_), которая не является ключевым словом. 

Ключевые слова языка: $\llang{if}, \llang{then}, \llang{else}, \llang{while}, \llang{do}, \llang{read}, \llang{write}, \llang{begin}, \llang{end}$

Литералы языка: числа с плавающей точкой, \llang{true}, \llang{false}. 

Операторы языка: $ \llang{+}, \llang{-}, \llang{*}, \llang{/}, \llang{\%}, \llang{==}, \llang{!=}, 
\llang{>}, \llang{>=}, \llang{<}, \llang{<=}, \llang{\&\&}, \llang{||}
$

Разделители языка: $(, \, ), \, ; $ 

Пример программы: 
$$
\llang{read}\; x \llang{;} \; \llang{if} \; y + 1 == x  \; \llang{then} \; \llang{write} \; y \; \llang{else} \; \llang{write} \; x \;  
$$

Результат лексического анализа должен быть в духе: 

\begin{verbatim}
KW_Read(0, 0, 3); Ident("x", 0, 5, 5); Colon(0, 6, 6); KW_If(0, 8, 9); Ident("y", 0, 11, 11);
Op(Plus, 0, 13, 13); Num(1, 0, 15, 15); Op(Eq, 0, 17, 18); Ident("x", 0, 20, 20); KW_Then(0, 22, 25); 
KW_Write(0, 27, 31); Ident("y", 0, 33, 33); KW_Else(0, 35, 38); KW_Write(0, 40, 44); Ident("x", 0, 46, 47);
\end{verbatim}

\end{document}

