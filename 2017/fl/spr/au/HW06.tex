\documentclass{article}

\usepackage[left=2cm,right=2cm,top=2cm,bottom=2cm,bindingoffset=0cm]{geometry}
\usepackage{listings}
\usepackage{indentfirst}
\usepackage{verbatim}
\usepackage{amsmath, amsthm, amssymb}
\usepackage{stmaryrd}
\usepackage{graphicx}
\usepackage{euscript}

\usepackage[utf8]{inputenc}
\usepackage[english,russian]{babel}
\usepackage[T2A]{fontenc}

\lstdefinelanguage{llang}{
keywords={skip, do, while, read, write, if, then, else, true, false, begin, end},
sensitive=true,
%%basicstyle=\small,
commentstyle=\scriptsize\rmfamily,
keywordstyle=\ttfamily\underbar,
identifierstyle=\ttfamily,
basewidth={0.5em,0.5em},
columns=fixed,
fontadjust=true,
literate={->}{{$\to$}}1
}

\lstset{
language=llang
}

\newcommand{\aset}[1]{\left\{{#1}\right\}}
\newcommand{\term}[1]{\mbox{\texttt{\bf{#1}}}}
\newcommand{\cd}[1]{\mbox{\texttt{#1}}}
\newcommand{\sembr}[1]{\llbracket{#1}\rrbracket}
\newcommand{\conf}[1]{\left<{#1}\right>}
\newcommand{\fancy}[1]{{\cal{#1}}}

\newcommand{\trule}[2]{\frac{#1}{#2}}
\newcommand{\crule}[3]{\frac{#1}{#2},\;{#3}}
\newcommand{\withenv}[2]{{#1}\vdash{#2}}
\newcommand{\trans}[3]{{#1}\xrightarrow{#2}{#3}}
\newcommand{\ctrans}[4]{{#1}\xrightarrow{#2}{#3},\;{#4}}
\newcommand{\llang}[1]{\mbox{\lstinline[mathescape]|#1|}}
\newcommand{\pair}[2]{\inbr{{#1}\mid{#2}}}
\newcommand{\inbr}[1]{\left<{#1}\right>}
\newcommand{\highlight}[1]{\color{red}{#1}}
\newcommand{\ruleno}[1]{\eqno[\textsc{#1}]}
\newcommand{\inmath}[1]{\mbox{$#1$}}
\newcommand{\lfp}[1]{fix_{#1}}
\newcommand{\gfp}[1]{Fix_{#1}}

\newcommand{\NN}{\mathbb N}
\newcommand{\ZZ}{\mathbb Z}

\begin{document}

%% Весь этот текст можно удалить
%% ====== от сих =====
\begin{center} {\LARGE Формальные языки}

{\Large домашнее задание до 23:59 06.04}
\end{center}
\bigskip

\begin{enumerate}
  \item Написать парсер для языка L (описание языка ниже), используя любимый способ писать парсеры. Не забыть про тесты (8 баллов).
    \begin{itemize}
        \item Можно использовать генераторы синтаксических анализаторов (yacc, bison, antlr, любой другой), лучше не использовать библиотеки парсер-комбинаторов (впрочем, это не возбраняется). 
        \item Ваш синтаксический анализатор должен принимать на вход то, что выдает ваш лексер --- те самые токены с позициями во входной строке; если для этого нужно править лексер --- правьте.
        \item Создайте консольное приложение для запуска синтаксического анализа.
        \begin{itemize}
            \item Консольное приложение обязательно должно принимать адрес файла со входной программой.
            \item Программа может быть многострочной.
            \item Консольное приложение должно запускать сначала лексер, потом парсер.
            \item Результатом синтаксического анализа должно быть абстрактное синтаксическое дерево --- напечатанное в человекочитаемом формате; можно в файл, но название файла должно быть связано с названием входного файла (можно использовать dot, можно написать свою функцию печати, можно воспользоваться другими подходами).
            \item Лексемы в дереве вывода должны отображаться со всей необходимой информацией.
            \item Парсеру на вход подается поток лексем с отфильтрованными комментариями.
            \item Парсер принимает программы в конкретном синтаксисе, описанном ниже, дерево строит соответствующее абстрактному синтаксису.
        \end{itemize}
     \end{itemize}
\end{enumerate}

\bigskip

\begin{center} {\Large Абстрактный синтаксис языка L } \end{center}
$$
X \mbox{ --- счетно-бесконечное множество переменных (идентификаторов)}
$$
$$
\otimes=\{\llang{+}, \llang{-}, \llang{*}, \llang{/}, \llang{\%}, \llang{==}, \llang{!=}, 
\llang{>}, \llang{>=}, \llang{<}, \llang{<=}, \llang{\&\&}, \llang{||}\}
$$

\begin{itemize}
\item Выражения: $\fancy{E}=X\cup\NN\cup(\fancy{E}\otimes\fancy{E})$. В выражениях могут использоваться круглые скобки.
\item Операторы: 

$$
\begin{array}{rll}
  \fancy{S}=&\llang{skip}&\cup\\
            &X\;\llang{:=}\;\fancy{E}&\cup\\
            &\fancy{S}\;\llang{;}\;\fancy{S}&\cup\\
            &\llang{write}\;\fancy{E}&\cup\\
            &\llang{read}\;\fancy{X}&\cup\\
            &\llang{while}\;\fancy{E}\;\llang{do}\;\fancy{S}&\cup\\
            &\llang{if}\;\fancy{E}\;\llang{then}\;\fancy{S}\;\llang{else}\;\fancy{S}
\end{array}
$$
\item Программы: $\fancy{P}=\fancy{S}$
\end{itemize}

\begin{center} {\Large Конкретный синтаксис языка L } \end{center}

Переменные $X$ являются идентификаторами.


Приоритет операторов (от высшего к низшему) представлен в таблице. Круглые скобки могут использоваться для задания приоритета. 

\begin{tabular}{ c | c | c | c }
  \multicolumn{4}{c}{Высший приоритет} \\ \hline
  \llang{*}    & \llang{/}  & \llang{\%} &            \\
  \llang{+}    & \llang{-}  &            &            \\
  \llang{>}    & \llang{>=} & \llang{<}  & \llang{<=} \\
  \llang{==}   & \llang{!=} &            &            \\
  \llang{\&\&} &            &            &            \\
  \llang{||}   &            &            &            \\ \hline
  \multicolumn{4}{c}{Низший приоритет} \\
\end{tabular}


\begin{itemize}
    \item Простые операторы выглядят согласно абстрактному синтаксису.
    \begin{itemize}
        \item $\llang{skip}$
        \item $\llang{:=}$
    \end{itemize}
    \item Чтение из входного потока и вывод в выходной требуют взятия аргументов в круглые скобки.
    \begin{itemize}
      \item $\llang{write} (\llang{1})$
      \item $\llang{read} (\llang{ident})$
    \end{itemize}
    \item {Условия для условного оператора и цикла берутся в круглые скобки. Тела могут быть многострочными, тогда они берутся в $\llang{begin}, \llang{end}$ скобки; если однострочные --- скобки не обязательны. }
    \begin{itemize}
        \item $\llang{if}\;(\llang{true})\;\llang{then}\;\llang{skip}\;\llang{else}\;\llang{begin}\;\llang{write} (\llang{x})\;\llang{;}\;\llang{z}\;\llang{:=}\;\llang{13}\;\llang{end}$
        \item  $\llang{while}\;(\llang{false})\;\llang{do}\;\llang{begin} \;\llang{skip}\;\llang{;}\;\llang{write} (\llang{x})\;\llang{;}\;\llang{z}\;\llang{:=}\;\llang{13}\;\llang{end}$
    \end{itemize}
\end{itemize}

В конце программы точка с запятой не ставится. 

Пример программы: 
$$
\llang{read}\; (x) \llang{;} \; \llang{if} \; (y + 1 == x)  \; \llang{then} \; \llang{write} \; (y) \; \llang{else} \; (* nothing \, to \, do \, here *) \, \llang{skip} 
$$


\end{document}

