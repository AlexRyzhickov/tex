\documentclass[12pt]{article}
\usepackage[left=2cm,right=2cm,top=2cm,bottom=2cm,bindingoffset=0cm]{geometry}
\usepackage{hyperref}
\usepackage{fontspec}
\usepackage{polyglossia}
\setdefaultlanguage{russian}
\setmainfont[Mapping=tex-text]{CMU Serif}

\begin{document}

\centering {\LARGE Формальные языки}

{\Large домашнее задание до 23:59 5.11}
\bigskip

\begin{enumerate}
  \item 
  {  Написать синтаксический анализатор для языка L, используя любую библиотеку парсер-комбинаторов (например, parsec; на страничке в википедии \url{https://en.wikipedia.org/wiki/Parser_combinator} --- в разделе External Links есть примеры других библиотек; можно взять любую другую хорошую). (8 баллов) 
  \begin{itemize}
    \item Добавьте в синтаксис языка закрывающие ``{скобки}" для условного оператора и цикла (\underline{if} $\mathcal{E}$ \underline{then} $\mathcal{S}$ \underline{else} $\mathcal{S}$ \underline{fi} и \underline{while} $\mathcal{E}$ \underline{do} $\mathcal{S}$ \underline{od}).
    \item Вход --- обычная строка, функции лексического анализа должны быть зашиты в парсер. Количество пробельных символов между лексемами не должно быть существенным. 
    \item Результатом синтаксического анализа должно быть дерево вывода — напечатанное в человекочитаемом формате.
    \item Предоставить набор тестов, демонстрирующий работоспособность парсера.
    \item Как и в предыдущих заданиях, все это должно быть приложением, ожидающим на вход имя файла с программой на языке L, должны быть скрипты сборки и запуска тестов.
  \end{itemize}
  }
  \item
  {
    Сделать pretty printer: функцию, которая, получая на вход дерево разбора, возвращает (красиво отформатированную) строчку с программой. (2 балла)
  }
  \item 
  {
    Провести оптимизацию арифметических и логических выражений: по построенному парсером \textbf{дереву разбора} построить новое, в котором нет лишних операций. Лишние операции это, например, домножения на 1, сложения с 0, домножения на 0, конъюнкция, где первый элемент --- ложь и т.д. Чем больше оптимизаций будет реализовано, тем лучше. 
    \begin{itemize}
      \item Количество баллов зависит от количества оптимизаций, поэтому желательно явно написать, какие оптимизации производятся.
      \item Результат оптимизаций --- дерево разбора. 
      \item Составить набор юнит-тестов, демонстрирующих, что каждая оптимизация работает, как задумано. 
      \item Добавить к вашему приложению возможность запуска анализатора в оптимизирующем режиме. Программа должна принимать на вход программу, визуализировать оптимизированное дерево и печатать оптимизированную программу (с помощью принтера из задания 2).
    \end{itemize}
  }
\end{enumerate}

\end{document}
