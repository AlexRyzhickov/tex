\documentclass[12pt]{article}
\usepackage[left=2cm,right=2cm,top=2cm,bottom=2cm,bindingoffset=0cm]{geometry}
\usepackage{fontspec}
\usepackage{polyglossia}
\setdefaultlanguage{russian}
\setmainfont[Mapping=tex-text]{CMU Serif}

\begin{document}
\centering {\LARGE Формальные языки}

{\Large домашнее задание до 23:59 24.10}
\bigskip

\enumerate
{
  \item
  
    Реализовать алгоритм синтаксического анализа графов: консольное приложение, принимающее 2 аргумента: пути к файлам с описанием грамматики и входным графом. Код и инструкция по запуску на гитхабе, скрипты сборки и запуска тестов. (8 баллов)
    \begin{itemize}
        \item \textbf{Вход:}
        \begin{itemize}
            \item Грамматика. Читается из файла, указанного в виде аргумента командной строки. Не обязана быть в 2NF. Для преобразования использовать результаты предыдущих домашних работ. Формат: 
            \begin{itemize}
                \item на строке не более одного правила
                \item нетерминалы --- заглавные буквы латинского алфавита
                \item терминалы --- строчные буквы латинского алфавита
                \item специальный символ для пустой строки --- eps
                \item разделитель левой и правой части правила --- \verb|->|
            \end{itemize}
            Пример:
            \begin{verbatim}
            S -> a S b
            S -> eps 
            \end{verbatim}
            \item Граф в формате DOT. Читается из файла, указанного в виде аргумента командной строки. Граф ориентированный, на ребре лежит имя токена в виде строки. Вершины --- целые числа. Пример:
            \begin{verbatim}
            digraph h
            {
               1 -> 2[label="a"]
               2 -> 3[label="c"]
               3 -> 1[label="a"]
            }
            \end{verbatim}
        \end{itemize}
        \item \textbf{Выход:} 
        \begin{itemize}
            \item Преобразованная в 2NF входная грамматика в том же формате, что и входная грамматика
            \item Тройки $(N_i,v_1,v_2)$, такие что во входном графе существует путь из $v_1$ в $v_2$, выводимый из $N_i$. Формат вывода: каждая тройка на новой строке, в круглых скобках, элементы разделены запятыми. Вокруг скобок пробелы не ставятся, справа от запятой 1 пробел. Вывод отсортирован в лексикографическом порядке. Пример:
            \begin{verbatim}
            (A, 1, 2)
            (A, 2, 4)
            (B, 3, 4)
            \end{verbatim}
        \end{itemize}
        \item \textbf{Тесты.} Набор файлов с грамматиками и графами.
    \end{itemize}
  
}
Дополнительные материалы.
\begin{itemize}
    \item Статья с описанием алгоритма (без модификаций для 2NF). Hellings J. Conjunctive context-free path queries. – 2014.
    \item Спецификация языка DOT: \url{http://www.graphviz.org/content/dot-language}
    \item Примеры библиотек для работы с графами. 
    \begin{itemize}
        \item .NET: YC.QuickGraph
        \item Java (JVM): JGraphT
        \item C++: Boost Graph Library
    \end{itemize}
\end{itemize}
\end{document}
%% It is just an empty TeX file.
%% Write your code here.