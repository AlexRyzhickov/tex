\documentclass[12pt]{article}
\usepackage[left=0cm,right=1.5cm,top=1cm,bottom=1cm,bindingoffset=0cm]{geometry}
\usepackage{hyperref}
\usepackage{fontspec}
\usepackage{polyglossia}
\usepackage{bbding}
\setdefaultlanguage{russian}
\setmainfont[Mapping=tex-text]{DejaVu Sans}
\usepackage{amsfonts}
\pagestyle{empty}

\begin{document}

{\Large Вариант 1}
\medskip

\begin{enumerate}
  \item Грамматика $S \rightarrow S a S b \, | \, \varepsilon$ является:
  \begin{enumerate}
      \item[(a)] праворекурсивной;
      \item[(b)] регулярной;
      \item[\Checkmark] однозначной.
  \end{enumerate}
  
  \item Какое из утверждений про алгоритм Коке-Янгера-Касами не является истинным?
  \begin{enumerate}
      \item[(a)] Сложность работы $O(n^3)$, где $n$ --- длина входной строки.
      \item[\Checkmark] Алгоритм не требует дополнительной памяти.
      \item[(c)] Алгооритм позволяет обрабатывать произвольные контекстно-свободные грамматики.
  \end{enumerate}

  \item Какое из следующих утверждений не является истинным?
  \begin{enumerate}
      \item[\Checkmark] Строки имеют одинаковые левосторонний и правосторонний вывод в однозначной грамматике.
      \item[(b)] LL(1)-анализатор является нисходящим.
      \item[(c)] LALR-анализатор является более мощным, чем SLR.
      \item[(d)] Ни для какого k неоднозначная грамматика не может быть LR(k).
  \end{enumerate}
\end{enumerate}

\medskip

{\Large Вариант 2}
\medskip

\begin{enumerate}
    \item Произвольный контекстно-свободный язык нельзя распознать при помощи:
    \begin{enumerate}
        \item[(a)] машины Тьюринга;
        \item[(b)] магазинного автомата;
        \item[\Checkmark] конечного автомата.
    \end{enumerate}
    
    \item Какое соотношение связывает количество состояний SLR(1)-автомата $(n)$, LR(1)-автомата $(m)$ и LALR(1)-автомата $(k)$ для следующей грамматики: $E \rightarrow (E) \, | \, a$?
    \begin{enumerate}
        \item[(a)] $n < m < k$.
        \item[(b)] $n = m = k$.
        \item[\Checkmark] $n = k < m$.
        \item[(d)] $n \geq k \geq m$.
    \end{enumerate}
    
    \item Какое утверждение о следующей грамматике арифметических выражений истинно?

$E \rightarrow E * F \, | \, F + E \, | \, F$

$F \rightarrow F - F \, | \, id $
    \begin{enumerate}
        \item[(a)] $*$ имеет больший приоритет, чем $+$.
        \item[\Checkmark] $-$ имеет больший приоритет, чем $*$.
        \item[(c)] $+$ и $-$ имеют одинаковый приоритет.
        \item[(d)] $+$ имеет больший приоритет, чем $*$.
    \end{enumerate}
\end{enumerate}

\newpage

{\Large Вариант 3}
\medskip

\begin{enumerate}
    \item Какие действия достаточно произвести, чтобы преобразовать произвольную контекстно-свободную грамматику к LL(1) грамматике?
    \begin{enumerate}
        \item[(a)] Удалить левую рекурсию.
        \item[(b)] Факторизовать грамматику.
        \item[(c)] Удалить левую рекурсию и факторизовать грамматику.
        \item[\Checkmark] Удаления левой рекурсии и факторизации не достаточно.
    \end{enumerate}
    
    \item Какое максимальное количество сверток может произвести восходящий синтаксический анализатор, если в грамматике нет эпсилон- и юнит- правил, во время разбора строки из $n$ терминалов?
    \begin{enumerate}
        \item[(a)] $n/2$.
        \item[\Checkmark] $n-1$.
        \item[(c)] $2n-1$.
        \item[(d)] $2^n$.
    \end{enumerate}
    
    \item Существенно неоднозначный язык:
    \begin{enumerate}
        \item[(a)] всегда конечен;
        \item[(b)] всегда регулярен;
        \item[\Checkmark] не может быть задан однозначной грамматикой;
        \item[(d)] не может быть задан  неоднозначной грамматикой.
    \end{enumerate}
\end{enumerate}


\medskip
{\Large Вариант 4}
\medskip

\begin{enumerate}
    \item Любой язык можно задать с помощью:
    \begin{enumerate}
        \item[(a)] конечного автомата;
        \item[(b)] магазинного автомата;
        \item[(c)] машины Тьюринга;
        \item[\Checkmark] ничего из вышеперечисленного.
    \end{enumerate}
    
    \item Нисходящий синтаксический анализатор, читая строку слева направо, строит ... вывод:
    \begin{enumerate}
        \item[\Checkmark] левосторонний;
        \item[(b)] правосторонний;
        \item[(c)] не специфицировано.
    \end{enumerate}
    
    \item Грамматика $S \rightarrow aSa \, | \, bS \, | \, c$ является грамматикой класса:
    \begin{enumerate}
        \item[(a)] LL(1), но не LR(1);
        \item[(b)] LR(1), но не LL(1);
        \item[\Checkmark] LL(1) и LR(1);
        \item[(d)] не LL(1) и не LR(1).
    \end{enumerate}
\end{enumerate}

~\\~

\medskip
{\Large Вариант 5}
\medskip

\begin{enumerate}
    \item Какой из нижеперечисленных синтаксических анализаторов является нисходящим?
    \begin{enumerate}
        \item[\Checkmark] Метод рекурсивного спуска.
        \item[(b)] Синтаксический анализ с учетом приоритетов операций.
        \item[(c)] LR(k).
        \item[(d)] LALR(k).
    \end{enumerate}
    
    \item Грамматика в Нормальной Форме Хомского не может:
    \begin{enumerate}
        \item[(a)] описывать пустой язык;
        \item[\Checkmark] содержать правила длины большей 2;
        \item[(c)] быть леворекурсивной;
        \item[(d)] быть неоднозначной.
    \end{enumerate}
    
    \item Язык $\{ a^n b^n c^n \, | \, n \geq 0 \} $ является:
    \begin{enumerate}
        \item[(a)] конечным
        \item[(b)] регулярным
        \item[(c)] контекстно-свободным 
        \item[\Checkmark] контекстно-зависимым 
    \end{enumerate}

\end{enumerate}

\pagebreak

\medskip
{\Large Вариант 1}
\medskip

\begin{enumerate}
    \item Отметьте \textbf{все} примеры алфавитов.
    \begin{enumerate}
        \item \{ \underline{if}, \underline{then}, \underline{else}, \underline{while}, \underline{do}, \underline{var}, \underline{write}, \underline{read}, \underline{skip}, 0, 1, x, y, z, +, *, =, :=, (, ), \&, | \}
        \item $\{ 0, 1 \}$
        \item \{ a, b, c, d, a, e, f\}
        \item  \{ 😁,😂, 😃, 😇, 😉, 😈, 😋, 😍, 😱 \}
        \item Все натуральные числа
        \item $\{ a_i \, | \, i \geq 0 \}$
        \item Все целые числа по модулю 5
    \end{enumerate}
    
    \item Грамматика $S \rightarrow ( S ) \, | \, S S \, | \, \varepsilon $:
    \begin{enumerate}
        \item описывает пустой язык; 
        \item регулярна;
        \item леворекурсивна;
        \item однозначна.
    \end{enumerate}


    \item Произвольный язык можно распознать при помощи: 
    \begin{enumerate}
        \item машины Тьюринга;
        \item магазинного автомата;
        \item конечного автомата;
        \item ничего из перечисленного.
    \end{enumerate}
\end{enumerate}

\pagebreak

\medskip
{\Large Вариант 2}
\medskip

\begin{enumerate}
    \item Отметьте \textbf{все} примеры алфавитов.
    \begin{enumerate}
        \item \{ (, ) \}
        \item \{ 😹, 😺, 😻, 😼, 😽, 😾, 😿, 🙀 \}
        \item Все рациональные числа
        \item \{ A \}
        \item \{ a, b, c, d, e, f, g, h, i, j, k, l, m, n, o, p, q, r, s, t, u, v, w, x, y, z \}
        \item Все сферы целочисленных диаметров, имеющие объем, не превышающий 100 кубических условных единиц
        \item \{ a, b, c, b , d, e, f\}
    \end{enumerate}    
            
    \item Грамматика $S \rightarrow ( S ) S \, | \, \varepsilon $:
    \begin{enumerate}
        \item описывает пустой язык; 
        \item регулярна;
        \item леворекурсивна;
        \item однозначна.
    \end{enumerate}

    \item Произвольный неограниченный язык можно распознать при помощи: 
    \begin{enumerate}
        \item машины Тьюринга;
        \item магазинного автомата;
        \item конечного автомата;
        \item ничего из перечисленного.
    \end{enumerate}
\end{enumerate}

\pagebreak

\medskip
{\Large Вариант 3}
\medskip

\begin{enumerate}
    \item Отметьте \textbf{все} примеры алфавитов.
    \begin{enumerate}
        \item \{ 0, 1, 2, 3, 4, 5, 6, 7, 8, 9, +, -, *, /, (, )\}
        \item $\{ \alpha, \beta, \gamma, \delta, \epsilon, \zeta, \eta, \theta, \iota, \kappa, \lambda, \mu, \nu, \xi, o, \pi, \rho, \sigma, \tau, \upsilon, \phi, \chi, \psi, \omega \} $
        \item \{  😁,😂, 😁,😂, 😁,😂 \}
        \item $\{ a_0, a_1, a_2, \dots \}$
        \item \{ $\cdot$ , --- \}
        \item Все комплексные числа
        \item Все квадраты с целочисленными сторонами, имеющие площадь, не превышающую 10 квадратных условных единиц
    \end{enumerate}

    \item Грамматика $S \rightarrow ( S ) \, | \, S S  $:
    \begin{enumerate}
        \item описывает пустой язык; 
        \item регулярна;
        \item не рекурсивна;
        \item однозначна.
    \end{enumerate}

    \item Произвольный контекстно-свободный язык можно распознать при помощи: 
    \begin{enumerate}
        \item машины Тьюринга;
        \item магазинного автомата;
        \item конечного автомата;
        \item ничего из перечисленного.
    \end{enumerate}

\end{enumerate}

\pagebreak

\medskip
{\Large Вариант 1}
\medskip

\begin{enumerate}
    \item Выберите вариант, правильно упорядоченный по возрастанию мощности классов языков.
    \begin{enumerate}
        \item Контекстно-свободные; регулярные; контекстно-зависимые языки.
        \item Регулярные; контекстно-зависимые; контекстно-свободные языки.
        \item Контекстно-зависимые; контекстно-свободные; регулярные языки.
        \item Регулярные; контекстно-свободные; контекстно-зависимые языки. 
    \end{enumerate}
    
    \item Выберите грамматику, эквивалентную грамматике $S \rightarrow ( S ) \, | \, S S \, | \, \varepsilon $:
    \begin{enumerate}
        \item $S \rightarrow ( S ) S \, | \, a $ 
        \item $S \rightarrow S ( S ) \, | \, \varepsilon $ 
        \item $a$ и $b$
        \item ни одна
    \end{enumerate}


    \item Грамматике $A \rightarrow a B; \, B \rightarrow b (C \, | \, E); \, C \rightarrow a B; \, E \rightarrow \varepsilon$б ($A$ --- стартовый нетерминал) соответствует регулярное выражение
    \begin{enumerate}
        \item $a b a b$
        \item $a ( b a )^* b$
        \item $(a b)^*$
        \item $b (a b)^* a$
    \end{enumerate}
\end{enumerate}

\medskip
{\Large Вариант 2}
\medskip

\begin{enumerate}
    \item Выберите вариант, правильно упорядоченный по убыванию мощности классов языков.
    \begin{enumerate}
        \item Контекстно-свободные; регулярные; контекстно-зависимые языки.
        \item Регулярные; контекстно-зависимые; контекстно-свободные языки.
        \item Контекстно-зависимые; контекстно-свободные; регулярные языки.
        \item Регулярные; контекстно-свободные; контекстно-зависимые языки. 
    \end{enumerate} 
            
    \item Выберите грамматику, эквивалентную грамматике $S \rightarrow ( S ) S \, | \, a $ :
    \begin{enumerate}
        \item $S \rightarrow ( S ) \, | \, S S \, | \, \varepsilon $
        \item $S \rightarrow S ( S ) \, | \, \varepsilon $ 
        \item $a$ и $b$
        \item ни одна
    \end{enumerate}


    \item Грамматике $A \rightarrow b B; \, B \rightarrow a (C \, | \, E); \, C \rightarrow b B; \, E \rightarrow \varepsilon$, ($A$ --- стартовый нетерминал) соответствует регулярное выражение
    \begin{enumerate}
        \item $b a b a$
        \item $a ( b a )^* b$
        \item $(b a)^*$
        \item $b (a b)^* a$
    \end{enumerate}
\end{enumerate}

\end{document}
