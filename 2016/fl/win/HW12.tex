t\documentclass[12pt]{article}
\usepackage[left=2cm,right=2cm,top=2cm,bottom=2cm,bindingoffset=0cm]{geometry}
\usepackage{fontspec}
\usepackage{polyglossia}
\setdefaultlanguage{russian}
\setmainfont[Mapping=tex-text]{CMU Serif}

\begin{document}
\centering {\LARGE Формальные языки}

{\Large домашнее задание до 23:59 2.12}
\bigskip

\enumerate
{
  \item Докажите или опровергните, что данный язык не является контекстно-свободным $\{ a^n b^m c^n d^m \, | \, n, m \geq 0 \}$ (3 балла)
  \item Напишите S-атрибутную грамматику для языка списков натуральных чисел (3 балла)
  \begin{itemize}
      \item Результат вычисления --- список, представленный в виде:
      \begin{itemize}
          \item $Nil$ обозначает пустой список
          \item $Cons(head, tail)$ обозначает непустой список, начинающийся на $head$, продолжающийся списком $tail$
      \end{itemize}
      \item Список заключается в квадратные скобки 
      \item Разделитель --- точка с запятой
      \item Списки: "[1;4;5]"; "[1]"; "[]"
      \begin{itemize}
          \item Результаты для них: $Cons(1, Cons(4, Cons(5, Nil)))$; $Cons(5, Nil)$; $Nil$
      \end{itemize}
      \item Не списки: "[1,4,-5]"; "[1;]"; "[a]"; "]["
      \item Промоделировать вычисления на примерах из задания
  \end{itemize}
  \item Напишите L-атрибутную грамматику для языка арифметических выражений в инфиксной записи над натуральными числами с операциями $+, -$ и скобками (3 балла)
  \begin{itemize}
      \item Результат --- целое число
      \item Законы арифметики должны быть соблюдены
      \item На всякий случай примеры
      \begin{itemize}
          \item $"1" \rightarrow 1$
          \item $"1+2+3" \rightarrow 6$
          \item $"1-2+3" \rightarrow 2$
          \item $"1-(2+3)" \rightarrow -4$
      \end{itemize}
      \item Промоделировать вычисления на примерах из задания
  \end{itemize}
}

 
\end{document}