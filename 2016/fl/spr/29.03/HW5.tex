\documentclass[12pt]{article}
\usepackage[left=2cm,right=2cm,top=2cm,bottom=2cm,bindingoffset=0cm]{geometry}
\usepackage{hyperref}
\usepackage{fontspec}
\usepackage{polyglossia}
\setdefaultlanguage{russian}
\setmainfont[Mapping=tex-text]{CMU Serif}

\begin{document}
%% Весь этот текст можно удалить
%% ====== от сих =====
\centering {\LARGE Формальные языки}

{\Large домашнее задание до 23:59 4.04}
\bigskip

\begin{enumerate}
  \item
  { Преобразовать в нормальную форму Хомского однозначную грамматику арифметических выражений, найти реализацию алгоритма CYK, запустить ее на полученной грамматике, трех различных корректных входах и трех некорректных, визуализировать полученные деревья вывода и записать сообщения об ошибках (прислать результаты работы парсера в pdf) (2 балла). 
  
    $E   {\to}   E + T | T  $ \\
    $T   {\to}   T * P | P  $ \\
    $P   {\to}   (E) | n $ \\
  }
  
  \item 
  {  Написать синтаксический анализатор для языка L, используя любимый генератор парсеров (семейство Yacc, Antlr, Bison, любой другой, но постарайтесь не выбирать парсер-комбинаторные библиотеки). Ваш синтаксический анализатор должен принимать на вход то, что выдает ваш лексер --- те самые токены с позициями во входной строке; если для этого нужно править лексер --- правьте. Составить набор тестов, демонстрирующий правильность работы полученного парсера. Сделать консольное (или вебовое) приложение, принимающее на вход программу на языке L, производящее лексический анализ, затем синтаксический анализ и \emph{визуализирующее} полученное дерево вывода. Прислать ссылку на репозиторий или код, сопровожденный инструкцией о том, как его собрать и запустить. (8 баллов за полностью выполненное задание)
  }
\end{enumerate}

%% ===== и до сих =====
\end{document}
