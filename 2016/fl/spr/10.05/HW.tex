\documentclass[12pt]{article}
\usepackage{geometry}
\usepackage{hyperref}
\usepackage{fontspec}
\usepackage{polyglossia}
\setdefaultlanguage{russian}
\setmainfont[Mapping=tex-text]{CMU Serif}

\begin{document}
\centering {\LARGE Формальные языки}

{\Large домашнее задание до 23:59 16.05}
\bigskip

\begin{enumerate}
  \item
  { Показать, что каждый из следующих языков не является контекстно-свободным.
  
  \begin{enumerate}
    \item $\{ a^i b^j c^k \, | \, i < j < k \}$
    \item $\{ a^n b^n c^i \, | \, i \leq n \}$
    \item $\{0^p \, | \, p $ --- простое$\}$
    \item $\{0^i 1^j \, | j = i^2 \}$
    \item $\{a^n b^n c^i \, | \, n \leq i \leq 2n\}$
    \item $\{w w^R w \, | \, w$ --- цепочка из нулей и единиц$\}$, т.е. множество цепочек, состоящих из цепочки w, за которой записаны ее обращение и она же еще раз, 
например 001100001
  \end{enumerate}
  }
  \item 
  { Придумать грамматики типа 1 или типа 0, задающие языки из первого задания; для каждой грамматики привести вывод какой-нибудь нитривиальной цепочки из языка.}
\end{enumerate}
\end{document}
