\documentclass[12pt]{article}
\usepackage[left=2cm,right=2cm,top=2cm,bottom=2cm,bindingoffset=0cm]{geometry}
\usepackage{hyperref}
\usepackage{fontspec}
\usepackage{polyglossia}
\setdefaultlanguage{russian}
\setmainfont[Mapping=tex-text]{CMU Serif}

\begin{document}
%% Весь этот текст можно удалить
%% ====== от сих =====
\centering {\LARGE Формальные языки}

{\Large домашнее задание до 23:59 21.03}
\bigskip

\begin{enumerate}
  \item
  { Докажите следующие классические свойства регулярных выражений ($e, f$ --- произвольные регулярные выражения): 
    \begin{enumerate}
      \item { $  \forall n \geq 1 $ верно $ (\varepsilon | e | ee | \dots | e^{n−1})(e^{n})^{*} = e^{*} $ }
      \item { $ (e^{∗} f)^{∗} e^{∗} = (e | f)^{*} $ }
      \item { $ \varepsilon | e (fe)^{∗} f = (ef)^{∗} $ }
    \end{enumerate}
  }
  
  \item 
  { Упростите следующие регулярные выражения: 
    \begin{enumerate} 
      \item { $ (a | b)^{∗} ab (a | b)^{∗} | (a | b)^{∗} a | b^{∗} $ }
      \item { $ (a | b)^{*} (ab | ba) (a | b)^{∗} | a^{∗} | b^{∗} $ }
    \end{enumerate}
  }
  
  \item
  { Составить регулярные выражения для следующих языков: 
    \begin{enumerate}
      \item 
      { Идентификаторы (алфавит: цифры, буквы английского алфавита в верхнем и нижнем регистре, нижнее подчеркивание)
        \begin{itemize}
          \item \verb;a, _a, A, a_b, iDeNt, _i_D_, __agent007__, _13, e2_e4; --- идентификаторы
          \item $\varepsilon$\verb;, 1, 1st, !myVar!; --- не идентификаторы
        \end{itemize}
      }
      \item 
      { Рациональные числа
        \begin{itemize} 
          \item \verb;0, +0, -0, 10, 0.0, 0.1e0, 0.10010e+1, 10e-123, .1, 2.; --- числа
          \item \verb;01, +.2, e, e., .,; $\varepsilon$ --- не числа
        \end{itemize}
      }
      \item
      { Списки целых чисел, разделенных точкой с запятой. Между скобками и числами, числами и разделителями может быть произвольное число пробелов (символ пробела обозначайте как $\backslash s$). 
        \begin{itemize} 
          \item \verb![], [1], [1;1;2;3;5;8], [  ], [ 4; 8;   15; 16   ; 23; 42]! --- корректные списки
          \item \verb!][, [[]], [1;2;3, [a], [1,2,3], [1;2;], [1; 23  4; 5]! --- некорректные списки
        \end{itemize}
      }
    \end{enumerate}
  }
  
  \item 
  { \begin{itemize}
      \item Выберите свой любимый язык программирования (не может совпадать с любимыми языками программирования ваших коллег); выбранный язык отметьте в табличке \href{https://goo.gl/wRlVOH}{https://goo.gl/wRlVOH}. 
      \item Найдите библиотеку для работы с конечными автоматами, написанную на этом языке, отметьте в табличке. 
      \item Для каждого из регулярных выражений из заданий 2 и 3 постройте НКА c $\varepsilon$-переходами, с помощью избранной библиотеки избавьтесь от $\varepsilon$-переходов, детерминизируйте автомат, минимизируйте его. На каждом из шагов визуализируйте автомат. Убедитесь, что полученный минимальный ДКА распознает требуемый язык. 
    \end{itemize}
  }
\end{enumerate}

%% ===== и до сих =====
\end{document}
